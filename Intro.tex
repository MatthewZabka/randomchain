Topology's tools have become very popular for analyzing data, and as one usually regards data as a random vector, some have attempted to apply randomness to topological ideas.  For example, Ginzburg and Pasechnik
\cite{ginzburg2017random} have investigated a random chain complex
with constant differential, while Zabka \cite{zabka2018random} has investigated
a random Bockstein operator.  Both of these papers have investigated their
topics in a strictly algebraic setting, and in this paper, we too shall
investigate a random chain complex in a strictly algebraic setting.

Chain complexes arise in topology as an algebraic measure in different
dimensions of the relationship between the cycles and boundaries of a
topological space. In particular, a chain complex defined on a space gives us a
way to calculate that space's homology groups.

Formally, a {\em chain complex $(C_n, \delta_n)$ } is a sequence of modules, denoted $C_n$, and a sequence of linear transformations $\delta_n : C_n \to C_{n-1}$ that satisfy the `boundary' condition
$\delta_{n-1}\delta_n = 0$ for all $n$. The $\delta_n$ are usually called the
boundary maps of the chain complex. An interested reader can see \cite{hatcher2002algebraic} for further details.

Let $q$ be a prime number and let $\Fq$ denote the field with $q$ elements. Consider the sequence of vector spaces $\Fqn{n_m}$ indexed by $m$ in the integers.  Let $A_m$ be a random sequence of $(n_{m-1})\times (n_m)$ matrices whose entries are chosen i.i.d. uniformly from $\Fq$, subject to the condition that the product of consecutive matrices is zero.  We then consider the random chain complex $(\Fqn{n_m},A_m)$. That is, we consider
\[
  \cdots \stackrel{A_{m+1}}{\lra} \Fqn{n_m} \stackrel{A_m}{\lra} \Fqn{n_{m-1}} 
  \stackrel{A_{m-1}}{\lra} \cdots 
\]
where $A_{i+1}A_i=0$ for every $i$. 

We have two main results. We first show that, as $q$ goes to infinity, homology is concentrated in dimension zero.  We then derive an explicit formula for the distribution of the Betti numbers.