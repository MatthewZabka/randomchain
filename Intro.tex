
The tools of algebraic topology have become very popular in the
analysis of large data sets~\cite{carlsson_topology_2009, ghrist_barcodes:_2008,
latala_persistent_2013}. Homological methods arising from
topology are scale-invariant, non-parametric, and importantly,
robust with respect to noise. This is increasingly important
for data arising from real-world applications, especially those
with low signal-to-noise ratio. All of these properties
leads one to investigate the properties of random topological
phenomena. The main goal of such theoretical considerations is to
develop a better understanding of noise and noise models, appropriate
for accurate data modelling in these contexts.

There have been a variety of attempts to randomize topological
constructions. Most famously, Erdos and Renyi introduced a model
for random graphs~\cite{erdos_random_1959, erdos_evolution_1960}. 
This work spawned an entire industry of probabilistic
models and tools used for understanding other random topological
and algebraic phenomenon. We continue this tradition by introducing
a new model for random chain complexes.

Chain complexes are algebraic constructions used to measure a variety of 
different properties. Their usefulness lies in providing a pathway
for homological algebra computations. They arise in a variety of contexts,
including group cohomology, Hoschild homology, de Rham cohomology, 
resolutions of commutative algebra and algebraic geometry, as well
as the main computational tool of algebraic topology. Specifically,
chain complexes measure the relationship between cycles and boundaries
of a topological space. This relationship underlies many topological
properties of interest, and is precisely what homology reveals. 

%Topology's tools have become very popular for analyzing data, and as one
%usually regards data as a random vector, some have attempted to apply
%randomness to topological ideas.  For example, Ginzburg and Pasechnik
%\cite{ginzburg2017random} have investigated a random chain complex with
%constant differential, while Zabka \cite{zabka2018random} has investigated a
%random Bockstein operator.  Both of these papers have investigated their topics
%in a strictly algebraic setting, and in this paper, we too shall investigate a
%random chain complex in a strictly algebraic setting.
%
%Chain complexes arise in topology as an algebraic measure in different
%dimensions of the relationship between the cycles and boundaries of a
%topological space. In particular, a chain complex defined on a space gives us a
%way to calculate that space's homology groups.

\subsection*{Main Results}

Let $R$ be a ring.  Formally, a {\em chain complex $(C_n, \delta_n)$ } is a
sequence of $R$-modules, denoted $C_n$, together with a sequence of linear
transformations 
\[
  \cdots \xrightarrow{\delta_{n+1}} C_n \xrightarrow{\delta_n}
  C_{n-1} \xrightarrow{\delta_{n-1}} \cdots
\]
such that $\delta_{n-1}\delta_n = 0$ for all $n$. The maps $\delta_n$ are called the boundary maps of the chain
complex, and the condition $\delta_{n-1} \delta_n = 0$ is known as the boundary
condition; see~\cite{hatcher2002algebraic} for further details.

%Formally, a {\em chain complex $(C_n, \delta_n)$ } is a sequence of modules,
%denoted $C_n$, and a sequence of linear transformations $\delta_n : C_n \to
%C_{n-1}$ that satisfy the `boundary' condition $\delta_{n-1}\delta_n = 0$ for
%all $n$. The $\delta_n$ are usually called the boundary maps of the chain
%complex. An interested reader can see \cite{hatcher2002algebraic} for further
%details.
%
%Let $q$ be a prime number and let $\Fq$ denote the field with $q$ elements.
%Consider the sequence of vector spaces $\Fqn{n_m}$ indexed by $m$ in the
%integers.  Let $A_m$ be a random sequence of $(n_{m-1})\times (n_m)$ matrices
%whose entries are chosen i.i.d. uniformly from $\Fq$, subject to the condition
%that the product of consecutive matrices is zero.  We then consider the random
%chain complex $(\Fqn{n_m},A_m)$. That is, we consider
%\[
%  \cdots \stackrel{A_{m+1}}{\lra} \Fqn{n_m} \stackrel{A_m}{\lra} \Fqn{n_{m-1}} 
%  \stackrel{A_{m-1}}{\lra} \cdots 
%\]
%where $A_{i+1}A_i=0$ for every $i$. 

Let $q$ be a prime number and let $\Fq$ denote the field with $q$ elements.
We build a random chain complex as follows (see Definition~\ref{defn:random_chain_cx}
for a precise statement). First, pick a \mc{We or not?}
sequence of integers $(n_m)$, where $m \in \N$. Next, we inductively
build random linear transformations $A_m : \Fqn{n_m} \lra \Fqn{n_{m-1}}$
for all $m$, subject to the contraint $A_{m-1} A_m = 0$. Fix the 
standard basis for $\Fqn{}$, so that it suffices to construct random
matrices $A_m \in M_{n_{m-1} \times n_m}(\Fqn{})$. We do so by 
chooising matrix entries i.i.d. from the uniform distribution on
$\Fqn{}$. 

%Consider the sequence of vector spaces $\Fqn{n_m}$ indexed by $m$ in the
%integers.  Let $A_m$ be a random sequence of $(n_{m-1})\times (n_m)$ matrices
%whose entries are chosen i.i.d. uniformly from $\Fq$, subject to the condition
%that the product of consecutive matrices is zero.  We then consider the random
%chain complex $(\Fqn{n_m},A_m)$. That is, we consider
%\[
%  \cdots \stackrel{A_{m+1}}{\lra} \Fqn{n_m} \stackrel{A_m}{\lra} \Fqn{n_{m-1}} 
%  \stackrel{A_{m-1}}{\lra} \cdots 
%\]
%where $A_{i+1}A_i=0$ for every $i$. 

We have two main results. We first show that, as $q$ goes to infinity, homology
is concentrated in dimension zero.  We then derive an explicit formula for the
distribution of the Betti numbers.


\subsection*{Related Work} There is another notion of random chain complex in the 
literature. In~\cite{ginzburg2017random}, Ginzburg and Pasechnik investigate
a random chain complex in which the vector spaces are 
