There have been a variety of attempts to randomize topological constructions.
Most famously, Erd\"os and R\'enyi introduced a model for random
graphs~\cite{erdos_random_1959, erdos_evolution_1960}.  This work spawned an
entire industry of probabilistic models and tools used for understanding other
random topological and algebraic phenomenon. These include various models for
random simplicial complexes, random networks, and many more. Further, this has
led to beautiful connections with statistical physics, for example through
percolation theory~\cite{broadbent_percolation_1957, bollobas_2006_percolation,
kesten_percolation_1982}.  Our ultimate goal is to understand higher
dimensional topological constructions arising in algebraic topology from a
random perspective. In this manuscript, we begin to address this goal with a
much simpler objective of understanding an algebraic construction on
topological spaces, known as a chain complex. 

Chain complexes are algebraic constructions used to measure a variety of
different properties. Their usefulness lies in providing a pathway for
homological algebra computations. They arise in a variety of contexts,
including commutative algebra, algebraic geometry, group cohomology, Hoschild homology, de Rham
cohomology, and of course algebraic topology.\mz{Do we need sources here?} \mc{Yes}
Specifically, chain
complexes measure the relationship between cycles and boundaries of a
topological space. This relationship uncovers many topological properties of
interest, and is precisely what homology reveals. Furthermore,
the singular chain complex of a topological space provides a canonical method
of associating a chain complex to a topological space.
%If the topological space has additional structure, like a CW or 
%simplicial decomposition, then the singular chain complex can be replaced
%by CW or simplicial chains, which are much easier to compute.

\mz{I don't know if we need the previous paragraph -- our paper is purely algebraic.}
\mc{I agree, I merged them.}


%Topology's tools have become very popular for analyzing data, and as one
%usually regards data as a random vector, some have attempted to apply
%randomness to topological ideas.  For example, Ginzburg and Pasechnik
%\cite{ginzburg2017random} have investigated a random chain complex with
%constant differential, while Zabka \cite{zabka2018random} has investigated a
%random Bockstein operator.  Both of these papers have investigated their topics
%in a strictly algebraic setting, and in this paper, we too shall investigate a
%random chain complex in a strictly algebraic setting.
%
%Chain complexes arise in topology as an algebraic measure in different
%dimensions of the relationship between the cycles and boundaries of a
%topological space. In particular, a chain complex defined on a space gives us a
%way to calculate that space's homology groups.


Let $R$ be a ring. A {\em chain complex $C_*=(C_m, \delta_m)$ with coefficients in
$R$} is a sequence of $R$-modules, denoted $C_m$, together with a sequence of
linear transformations 
\[
  \cdots \xrightarrow{\delta_{m+1}} C_m \xrightarrow{\delta_m}
  C_{m-1} \xrightarrow{\delta_{m-1}} \cdots
\]
such that $\delta_{m-1}\delta_m = 0$ for all $m \in \Z$.  The maps $\delta_m$
are called the boundary maps of the chain complex, and the equation
$\delta_{m-1} \delta_m = 0$ is known as the boundary condition;
see~\cite{cartan2016homological} for further details. 

The boundary condition $\delta_{m-1}\delta_m=0$ forces $\im \delta_m \subseteq \ker \delta_{m-1}$.
The {\em homology} of a chain complex measures the deviation of this containment
from equality:
\begin{equation*}
  \label{eqn:hom}
  H_m(C_*;R) = \frac{\ker \delta_{m-1}}{\im \delta_m} \, .
\end{equation*}
When the chain complex arises from singular chains on a topological
space, the homology is a very powerful tool in algebraic topology.
\cite{hatcher2002algebraic} \mz{I've put the Hatcher citation here instead --
do we need more?} \mc{After a second reading, the sentence
probably needs to be changed.}

We work over the field with $q$-elements $R = \Fqn{}$ and consider the chain
complex whose $R$-modules are given by finite dimensional vector spaces,  $C_m
= \Fqn{n_m}$, where each $n_m \in \mathbb{N}$. After fixing the standard
basis for $\Fqn{}$, the boundary
maps can be regarded as $n_{m-1}\times n_m $ matrices, which we denote by
$A_m$. Homology can then be understood in terms of dimension \mz{We had defined
the Betti numbers in terms of $A_m$ without defining the $A_m$. What I've
written is one possible fix.}
\begin{equation*}
  \beta_m := \dim_{\Fqn{}} \frac{\ker A_{m-1}}{\im A_m} \, ,
  \label{eqn:betti_numbers}
\end{equation*}
where $\beta_m$ is known as the $m^\mathrm{th}$ {\em Betti number}.

% \mc{expand this}
%Formally, a {\em chain complex $(C_n, \delta_n)$ } is a sequence of modules,
%denoted $C_n$, and a sequence of linear transformations $\delta_n : C_n \to
%C_{n-1}$ that satisfy the `boundary' condition $\delta_{n-1}\delta_n = 0$ for
%all $n$. The $\delta_n$ are usually called the boundary maps of the chain
%complex. An interested reader can see \cite{hatcher2002algebraic} for further
%details.
%
%Let $q$ be a prime number and let $\Fq$ denote the field with $q$ elements.
%Consider the sequence of vector spaces $\Fqn{n_m}$ indexed by $m$ in the
%integers.  Let $A_m$ be a random sequence of $(n_{m-1})\times (n_m)$ matrices
%whose entries are chosen i.i.d. uniformly from $\Fq$, subject to the condition
%that the product of consecutive matrices is zero.  We then consider the random
%chain complex $(\Fqn{n_m},A_m)$. That is, we consider
%\[
%  \cdots \stackrel{A_{m+1}}{\lra} \Fqn{n_m} \stackrel{A_m}{\lra} \Fqn{n_{m-1}} 
%  \stackrel{A_{m-1}}{\lra} \cdots 
%\]
%where $A_{i+1}A_i=0$ for every $i$. 

% \mc{Do we need an iterative construction like this? I don't think so:
% people work with unbounded chain complexes all the time. Equivalently,
% building a $\Z$-graded chain complex doesn't rely on an induction or anything.
% E.g.: take $C_n = R$ for all $n \in \Z$ and take all the maps to be zero. 
% This is a perfectly good chain complex that doesn't \emph{start} anywhere.
% This would allow us to define a more general object, even if we never work
% with it.}

\subsection*{Main Results}
Let $q$ be a prime number.  We
build a random chain complex with coefficients in $\Fqn{}$ as follows (see
Definition~\ref{defn:random_chain_cx} for a precise statement). Given a
sequence of non-negative integers $\{n_m\}$, where $m \in \N$, together with a probability
distribution $\varphi$ on $\Fqn{}$, we construct random
linear transformations 
\[
  A_m : \Fqn{n_m} \lra \Fqn{n_{m-1}} \, ,
\]
for all $m$. The transformations are subject to the constraint $A_{m-1} A_m =
0$, and should be chosen according to $\varphi$. The latter means the
following: After fixing the standard basis for $\Fqn{n_m}$, it suffices to
construct random $n_{m-1} \times n_m$ matrices $A_m$, satisfying $A_{m-1}A_m =
0$. We do so by choosing matrix entries i.i.d. from the distribution $\varphi$
on $\Fqn{}$. We then say that $(\Fqn{n_m}, A_m, \varphi)$ is a {\em random
chain complex}. 

We are primarily interested in the case when $\varphi$ is the discrete uniform distribution on $\Fqn{}$. In this case, we say
$(\Fqn{n_m}, A_m)$ is a {\em uniform random chain complex}.


Our first result is an explicit formula for the distribution
of the Betti numbers.
\begin{bigthm} 
  \label{thm:bettinum}
  Let $\beta_m$ be the $m$-th Betti number of a uniform random chain complex
  $(\Fq^{n_m} , A_m)$. Then
  \[    
    \bP[\beta_m = b] = \sum_{i_m=0}^{n_{m}} P_{i_m}^{m}(i_m-b)
    \sum_{i_{m-1}=0}^{n_{m-1}} P_{i_{m-1}}^{m-1}\left(n_{m} -i_m\right) \cdots
    \sum_{i_1 = 0}^{n_1} P_{i_1}^1\left(n_2 - i_2\right) P_{n_0}^0 \left(n_1 - i_1\right) \, ,
  \]
  where $P^m_k(r)$ is given in Eq.~\eqref{eqn:Pmkr}.
\end{bigthm}

As Theorem~\ref{thm:bettinum} gives a formula for
computing the distribution of the Betti numbers, it also leads to formulas for
other probabilistic properties of $\beta_m$, such as its moments and variance.

Our second main result show that, asymptotically, the $m$-th Betti number of a uniform random chain complex concentrates in a single value. Set
\begin{equation*}
  (n)_+ = \max(0,n) \, , % \mbox{ and } \, \, (n)_- = \min(0,n) \
\end{equation*}
to be the {\em positive part} of $n$.
Define
\begin{equation}
  B_m = (-n_{m+1} + (n_m - (n_{m-1} - (\cdots - (n_1 - n_0)_+ \cdots)_+ )_+
  )_+)_+ \, \, .
  \label{eqn:Nm}
\end{equation}

\begin{bigthm}
  \label{thm:qtoinfty}
  For a uniform random chain complex $(\Fqn{n_m},A_m)$ with $B_m$ defined as in Eq.~\eqref{eqn:Nm},
  \[
    \bP[\beta_m = B_m] \ra 1 
    \mbox{ as } q \ra \infty  \, .
  \]
\end{bigthm}

% \mc{The following remarks need to be re-written and substantiated.}
%\begin{remark}
%  \label{rem:smallest}
%  The number $N_m$ of Theorem~\ref{thm:qtoinfty} represents the smallest
%  possible value of $\beta_m$. Therefore, we interpret Theorem~\ref{thm:qtoinfty}
%  as the statement that the rank of the homology is minimal as 
%  $q \ra \infty$.
%\end{remark}

\begin{remark}
  \label{rem:monotone}
  As a special case of the above theorem, consider when $\{n_m\}$
  is constant or increasing. In this case, 
  $B_m = 0$, and the homology is trivial in probability as $q \ra \infty$. \mz{Should we refer to a cor here?} \mc{I don't think so. It should be obvious from the formula. Which cor did you have in mind?}
\end{remark}

\subsection*{Related Work} Others have considered different methods of applying
randomness to chain complexes. In~\cite{ginzburg2017random}, Ginzburg and
Pasechnik investigate
a different 
% \mz{Is it really that different? The map $D$ is just one of the nilpotent
% matrices from $V$ into $V$ -- they're just choosing their matrix according to a
% different distribution.  Best to talk about this.} 
notion of random chain
complex than the one described above.  Given a finite dimensional vector space
$V$ over $\Fqn{}$, they consider chain complexes of the form \[ \cdots
\stackrel{D}{\lra} V \stackrel{D}{\lra} V \stackrel{D}{\lra} \cdots \, \, , \]
for a randomly chosen linear operator $D$ such that $D^2 = 0$. They choose the
operator $D$ uniformly over all such possible choices. In particular, our construction
is distinct from theirs, since they use the same operator $D$
at each stage of the complex.

\mz{Is it important to reference their results, since they don't align with our own?}
\mc{No I don't think so. At least not as precisely. What do you think of the following?}
The first of their main results~\cite[Thm 2.1]{ginzburg2017random} states that
the rank of homology concentrates in the 
% $\bP[\beta_r > 0] \ra 0$ as $q \ra \infty$ for $r>1$. Furthermore, $\bP[\beta_0
% = 0] \ra 1$ ($n$-even) or $\bP[\beta_1 = 0] \ra 1$ ($n$-odd) as $q \ra \infty$.
%One interpretation of this result is that homology concentrates in the
lowest possible dimension as $q \ra \infty$.  While a direct comparison with
our results is not possible, we view Theorem~\ref{thm:qtoinfty} as a result in the
same fashion of concentration of minimal rank homology.

% Their second main result~\cite[Thm 2.2]{ginzburg2017random} shows that 
% the following limit exists and is bounded
% \[
%   0 < \lim_{n \ra \infty} \bP[\beta_r = j] < 1 \, ,
% \]
% for all $r$ and $j$. Based on explicit bounds, they provide analytic
% formulas for the quantity $\lim_{n\ra \infty}\bP[\beta_j=r]$ in terms
% of $r$ and $j$.\mz{Has something happened to the index here?}

In \cite{zabka2018random}, the second author introduced and studied the
properties of a random Bockstein operation. The Bockstein is an example of a
cohomology operation in algebraic topology, which can detect information about the topology of a
space that neither homology nor cohomology can detect. In homological algebra,
the Bockstein is a connecting homomorphism associated with a short exact
sequence of abelian groups, which are then used as the coefficients in a chain
complex. \mc{We need to connect this with our work/results.}

\mz{I'm happy to write the following subsections, but I'm not sure what should go there.}

\subsection*{Outline}
  The paper is organized as follows. In Section 2, we discuss preliminary results useful for 
  the combinatorial aspects of our results. We give a precise definition of a model for a 
  random chain complex in Section 3, as well as prove
  Theorem~\ref{thm:bettinum}. In Section 4, we complete the proof of Theorem~\ref{thm:qtoinfty}.

\subsection*{Conventions}
\mc{Do we have any conventions? What should go here? I don't remember now.}

\subsection*{Acknowledgments} The first author would like to thank
Peter Bubenik for helpful discussions.
