

There have been a variety of attempts to randomize topological
constructions. Most famously, Erdos and Renyi introduced a model
for random graphs~\cite{erdos_random_1959, erdos_evolution_1960}. 
This work spawned an entire industry of probabilistic
models and tools used for understanding other random topological
and algebraic phenomenon. We continue this tradition by introducing
a new model for random chain complexes.

Chain complexes are algebraic constructions used to measure a variety of 
different properties. Their usefulness lies in providing a pathway
for homological algebra computations. They arise in a variety of contexts,
including group cohomology, Hoschild homology, de Rham cohomology, 
resolutions of commutative algebra and algebraic geometry, as well
as the main computational tool of algebraic topology. Specifically,
chain complexes measure the relationship between cycles and boundaries
of a topological space. This relationship underlies many topological
properties of interest, and is precisely what homology reveals. 

%Topology's tools have become very popular for analyzing data, and as one
%usually regards data as a random vector, some have attempted to apply
%randomness to topological ideas.  For example, Ginzburg and Pasechnik
%\cite{ginzburg2017random} have investigated a random chain complex with
%constant differential, while Zabka \cite{zabka2018random} has investigated a
%random Bockstein operator.  Both of these papers have investigated their topics
%in a strictly algebraic setting, and in this paper, we too shall investigate a
%random chain complex in a strictly algebraic setting.
%
%Chain complexes arise in topology as an algebraic measure in different
%dimensions of the relationship between the cycles and boundaries of a
%topological space. In particular, a chain complex defined on a space gives us a
%way to calculate that space's homology groups.

\subsection*{Main Results}

Let $R$ be a ring.  Formally, a {\em chain complex $(C_n, \delta_n)$ with coefficients
in $R$} is a
sequence of $R$-modules, denoted $C_n$, together with a sequence of linear
transformations 
\[
  \cdots \xrightarrow{\delta_{n+1}} C_n \xrightarrow{\delta_n}
  C_{n-1} \xrightarrow{\delta_{n-1}} \cdots
\]
such that $\delta_{n-1}\delta_n = 0$ for all $n \in \Z$. The maps $\delta_n$ are called the boundary maps of the chain
complex, and the condition $\delta_{n-1} \delta_n = 0$ is known as the boundary
condition; see~\cite{hatcher2002algebraic} for further details.

%Formally, a {\em chain complex $(C_n, \delta_n)$ } is a sequence of modules,
%denoted $C_n$, and a sequence of linear transformations $\delta_n : C_n \to
%C_{n-1}$ that satisfy the `boundary' condition $\delta_{n-1}\delta_n = 0$ for
%all $n$. The $\delta_n$ are usually called the boundary maps of the chain
%complex. An interested reader can see \cite{hatcher2002algebraic} for further
%details.
%
%Let $q$ be a prime number and let $\Fq$ denote the field with $q$ elements.
%Consider the sequence of vector spaces $\Fqn{n_m}$ indexed by $m$ in the
%integers.  Let $A_m$ be a random sequence of $(n_{m-1})\times (n_m)$ matrices
%whose entries are chosen i.i.d. uniformly from $\Fq$, subject to the condition
%that the product of consecutive matrices is zero.  We then consider the random
%chain complex $(\Fqn{n_m},A_m)$. That is, we consider
%\[
%  \cdots \stackrel{A_{m+1}}{\lra} \Fqn{n_m} \stackrel{A_m}{\lra} \Fqn{n_{m-1}} 
%  \stackrel{A_{m-1}}{\lra} \cdots 
%\]
%where $A_{i+1}A_i=0$ for every $i$. 

Let $q$ be a prime number and let $\Fq$ denote the field with $q$ elements.
We build a random chain complex as follows (see Definition~\ref{defn:random_chain_cx}
for a precise statement). First, pick a \mc{We or not?}
sequence of integers $(n_m)$, where $m \in \N$. Next, we inductively
build random linear transformations 
\[
  A_m : \Fqn{n_m} \lra \Fqn{n_{m-1}} \, ,
\]
for all $m$, subject to the constraint $A_{m-1} A_m = 0$. By fixing the
standard basis for $\Fqn{}$, it suffices to construct random
matrices $n_{m-1} \times n_m$ matrices $A_m$. We do so by 
choosing matrix entries i.i.d. from the uniform distribution on
$\Fqn{}$. 

%Consider the sequence of vector spaces $\Fqn{n_m}$ indexed by $m$ in the
%integers.  Let $A_m$ be a random sequence of $(n_{m-1})\times (n_m)$ matrices
%whose entries are chosen i.i.d. uniformly from $\Fq$, subject to the condition
%that the product of consecutive matrices is zero.  We then consider the random
%chain complex $(\Fqn{n_m},A_m)$. That is, we consider
%\[
%  \cdots \stackrel{A_{m+1}}{\lra} \Fqn{n_m} \stackrel{A_m}{\lra} \Fqn{n_{m-1}} 
%  \stackrel{A_{m-1}}{\lra} \cdots 
%\]
%where $A_{i+1}A_i=0$ for every $i$. 

The boundary condition $A_{m-1}A_m=0$ forces $\im A_m \subseteq \ker A_{m-1}$.
The {\em homology} of a chain complex measures the deviation of this containment
from an equality. Since we work with field coefficients, this deviation can
be understood in terms of rank
\begin{equation}
  \beta_m := \mathrm{rank} \left(\ker A_{m-1} / \im A_m \right)\, .
  \label{eqn:betti_numbers}
\end{equation}

Our first main result shows that as $q \ra \infty$, the homology of a 
random chain complex tends to its minimal value.

\begin{bigthm}
  \label{thm:qtoinfty}
  Set $N_m = \max(0,n_m - \min(n_{m},n_{m-1}) - \min(n_m,n_{m+1}))$. 
  For a random chain complex $(\Fqn{n_m},A_m)$, we have
  \[
    \bP[\beta_m = N_m] \ra 1 
    \mbox{ as } q \ra \infty  \, .
  \]
\end{bigthm}

\begin{remark}
  \label{rem:smallest}
  The number $N_m$ of Theorem~\ref{thm:qtoinfty} represents the smallest
  possible value of $\beta_m$. Therefore, we interpret Theorem~\ref{thm:qtoinfty}
  as the statement that the rank of the homology is as small as possible as 
  $q \ra \infty$.
\end{remark}

\begin{remark}
  \label{rem:monotone}
  As a special case of the above theorem, consider when $\{n_m\}$
  is monotonic, e.g., either constant or increasing. In this case, 
  $N_m = 0$, and the homology is trivial in probability as $q \ra \infty$.
\end{remark}


Our second main result is an explicit formula for the distribution
of the Betti numbers.
\begin{bigthm} 
  \label{thm:bettinum}
  Let $\beta_j$ be the $j$-th Betti number of the random chain complex
  $(\Fq^{n} , A_m)$. Then
  \[    
    \bP[\beta_j = b] = \sum_{i_j=0}^{n_{j}} P_{i_j}^{j}(i_j-b)
    \sum_{i_{j-1}=0}^{n_{j-1}} P_{i_{j-1}}^{j-1}\left(n_{j} -i_j\right) \cdots
    \sum_{i_1 = 0}^{n_1} P_{i_1}^1\left(n_2 - i_2\right) P_{n_0}^0 \left(n_1 - i_1\right).
  \]
  where $P^l_k(r)$ is given in Eq.~\eqref{eqn:Pmkr}.
\end{bigthm}

\subsection*{Related Work} There are other notions of random chain complex
one could consider. In~\cite{ginzburg2017random}, Ginzburg and Pasechnik investigate
a different notion of random chain complex than the one described above.
Given a finite dimensional vector space $V$ over $\Fqn{}$, they consider chain
complexes of the form
\[
  \cdots \stackrel{D}{\lra} V \stackrel{D}{\lra} V \stackrel{D}{\lra} \cdots \, \, ,
\]
for a randomly chosen linear operator $D$ such that $D^2 = 0$. They choose the 
operator $D$ uniformly over all such possible choices. 

The first of their
main results~\cite[Thm 2.1]{ginzburg2017random} states that
$\bP[\beta_r > 0] \ra 0$ as $q \ra \infty$ for $r>1$. Furthermore,
$\bP[\beta_0 = 0] \ra 1$ ($n$-even) or $\bP[\beta_1 = 0] \ra 1$ ($n$-odd)
as $q \ra \infty$. One should interpret this result as the fact that
homology concentrates in the lowest possible dimension as $q \ra \infty$.
This result is a special case of Theorem~\ref{thm:qtoinfty} 
where $n_m \equiv n$ for all $m$ (c.f. Remark~\ref{rem:monotone}).

Their second main result~\cite[Thm 2.2]{ginzburg2017random} shows that 
the following limit exists and is bounded
\[
  0 < \lim_{n \ra \infty} \bP[\beta_r = j] < 1 \, ,
\]
for all $r$ and $j$. Based on explicit bounds, they provide analytic
formulas for the quantity $\lim_{n\ra \infty}\bP[\beta_j=r]$ in terms
of $r$ and $j$. In our situation, we interpret $n \ra \infty$ as the case
$n_m \ra \infty$ and consider $\beta_M$ for large $M$. In theory,
one can analyze this quantity using Theorem~\ref{thm:bettinum},
but we have not been able to obtain useful bounds.
