This section consists of several lemmas that count the number of elements in various sets related to finite vector spaces over $\mathbb{F}_q$.  We provide proofs for these lemmas, but an interested reader can see \cite{stanley2011enumerative} for further details. 
\begin{lemma}\label{NumkTup}
  The number of ordered, linearly independent $k$-tuples of vectors in $\Fqn{n}$  is
\[
(q^n-1)(q^n-q)(q^n-q^2)\cdots(q^n - q^{k-2})(q^n-q^{k-1}).
\]
\end{lemma}

\begin{proof}
Since first vector in the $k$-tuple may be any vector except for the zero vector, there are $q^n-1$ choices for the first vector.  For $1<m \leq k$, the $m$-th vector in the $k$-tuple may be any vector that is not a linear combination of the previously chosen $m-1$ vectors. So there are $q^n - q^{m-1}$ choices for the $m$-th vector.
\end{proof}

The number $\qbin{n}{k}$ defined in the next theorem is known as the $q$-binomial coefficient.

\begin{lemma}\label{NumkSub}
  The number of $k$-dimensional subspaces of $\Fqn{n}$ is
\[
\qbin{n}{k} = \frac{(q^n-1)(q^n-q)(q^n-q^2)\cdots(q^n - q^{k-2})(q^n-q^{k-1})}{(q^k-1)(q^k-q)(q^k-q^2)\cdots(q^k - q^{k-2})(q^k-q^{k-1})}.
\]
\end{lemma}


\begin{proof}
  Let $\qbin{n}{k}$ denote the number of $k$-dimensional subspaces of $\Fqn{n}$ and $N(q,k)$ be the number of ordered, linear independent $k$-tuples of vectors in $\Fqn{n}$.  Then Corollary \ref{NumkTup} gives
\begin{equation}\label{Nqk1}
N(q,k) = (q^n-1)(q^n-q)(q^n-q^2)\cdots(q^n - q^{k-2})(q^n-q^{k-1}).
\end{equation}
We may also find $N(q,k)$ another way: We can first choose a $k$-dimensional subspace and then choose the independent vectors in our $k$-tuple from the chosen subspace.  There are $\qbin{n}{k}$ $k$-dimensional subspaces of $\Fqn{n}$.  Then, there are $q^k-1$ choices for the first vector in the $k$-tuple, and, for $1<m \leq k$, there are $q^k - q^{m-1}$ vectors for the $m$-th vector in the $k$-tuple.  Thus
\begin{equation}\label{Nqk2}
N(q,k) = \qbin{n}{k}(q^k-1)(q^k-q)(q^k-q^2)\cdots(q^k - q^{k-2})(q^k-q^{k-1}).
\end{equation}
Equations (\ref{Nqk1}) and (\ref{Nqk2}) give the required result.
\end{proof}

Using Lemmas \ref{NumkSub} and \ref{Nqk1}, we can count the number of matrices with of a given rank. The following lemma will aid us in Section \ref{SecCondComp}.

\begin{lemma}\label{Num_mbyn_rankr}
The number of $m\times n$ matrices with entries in $\Fq$ of rank $r$ is given by
\begin{align*}
	&\qbin{m}{r}(q^n-1)(q^n-q)\cdots (q^n-q^{r-1})\\
   =& \qbin{n}{r}(q^m-1)(q^m-q)\cdots (q^m-q^{r-1})\\
   =& \frac{(q^m-1) (q^m-q) \cdots (q^m-q^{r-1})\cdot
            (q^n-1) (q^n-q) \cdots (q^n-q^{r-1})}
	 		      {(q^r-1)(q^r - q)(q^r - q^2)\cdots (q^r - q^{r-2})(q^r-q^{r-1})}
\end{align*}
\end{lemma}

\begin{proof}
  Let $W$ be a fixed $r$-dimensional subspace of $\Fqn{n}$.  Then the number of matrices whose column space is $W$ is given by the number of $r\times n$ matrices with rank $r$.  This number is given by Lemma \ref{NumkTup}. The number of $r$-dimensional subspaces of $\Fqn{n}$ is $\qbin{m}{r}$, as stated in Lemma \ref{NumkSub}.  The product of these is the number of $m\times n$ rank $r$ matrices.
\end{proof}

We now turn our attention to a sequence of random matrices.\mz{I've moved this stuff to Section 2. We can talk about this}

\begin{definition}\label{defPkmr}
Let $n_m$ be a sequence of natural numbers. Let $B_m$ be a sequence of random $(n_m) \times (n_{m-1})$ matrices whose entries are chosen i.i.d. uniformly from $\Fq$. Let $r$ be a non-negative integer.  Define 
\[
  P^m_k(r) := \mathbb{P} 
  \left[\rank(B_{m+1}) = r|B_{m}B_{m+1} = 0, \nul(B_{m}) = k \right] \, .
\]
\end{definition}


Lemma~\ref{Num_mbyn_rankr} gives us the following.

\begin{lemma}\label{lemPkmr} With $B_m$ defined as in Definition \ref{defPkmr}, we have that
\[
P^m_k(r) = \begin{cases}
  {\displaystyle \frac{\left(\prod_{j=0}^{r-1}\left(q^{n_{m+1}}-q^{j}\right)\right)
  \left(\prod_{j=0}^{r-1}\left(q^k - q^j \right) \right)}
  {q^{kn_{m+1}} \left(\prod_{j=0}^{r-1} \left(q^r-q^j\right)\right)}}
            					& \textrm{ if } k\neq 0, r \leq k \\
           0					& \textrm{ if }  r>k,\\
           1					& \textrm{ if } r = k = 0.
            \end{cases}
\]
\end{lemma}
\begin{proof}
Let $k= \nul(B_{m})$ and suppose $B_mB_{m+1} = 0$.  Then $B_{m+1}$ maps $\Fq^{n_{m+1}}$ into the kernel of $B_m$, and thus $\rank(B_{m+1}) \leq k$.  Therefore, if $r>k$, we have $P^m_k(r) = 0$.  Further, if $k=0$, then $\rank(B_{m+1}) = 0$, so $P^m_0(0) = 1$.

On the other hand, suppose $k\neq 0$ and $r\leq k$.  Then, $B_{m+1}$ represents a linear transformation from $\Fq^{n_{m+1}}$ into a $k$-dimensional subspace of $\Fq^{n_{m}}$. Thus, by changing basis, $B_{m+1}$ can be represented by a $(n_{m+1}) \times k$ matrix.  There are $q^{kn_{m+1}}$ such matrices, and, by Lemma \ref{Num_mbyn_rankr}, there are 
\[
 \displaystyle \frac{\left(\prod_{j=0}^{r-1}\left(q^{n_{m+1}}-q^{j}\right)\right)
  \left(\prod_{j=0}^{r-1}\left(q^k - q^j \right) \right)}
  {\prod_{j=0}^{r-1} \left(q^r-q^j\right)}
\]
such matrices of rank $r$.

\end{proof}