\documentclass{amsart}
%\usepackage[utf8]{inputenc}
\usepackage{amssymb, mathtools, amsthm}
\usepackage{graphicx}
\usepackage{hyperref}
\usepackage{microtype}


\newtheorem{theorem}{Theorem}[section]
\newtheorem{bigthm}{Theorem}
\renewcommand{\thebigthm}{\Alph{bigthm}}
\newtheorem{corollary}[theorem]{Corollary}
\newtheorem{proposition}[theorem]{Proposition}
\newtheorem{lemma}[theorem]{Lemma}
\newtheorem{definition}[theorem]{Definition}
\theoremstyle{remark}
\newtheorem{convention}[theorem]{Convention}
\newtheorem{remark}[theorem]{Remark}
\newtheorem{conjecture}[theorem]{Conjecture}
\newtheorem{question}[theorem]{Question}

\DeclareMathOperator{\rank}{\mathrm{rank}}
\DeclareMathOperator{\im}{\mathrm{im}}
%\DeclareMathOperator{\ker}{\mathrm{ker}}

%\usepackage{libertine,anyfontsize,libertinust1math}
\linespread{1.1}


\usepackage[usenames,dvipsnames]{color}
 
% comments
\newcommand\mc[1]{\textcolor{NavyBlue}{\textbf{Mike: }#1}}
\newcommand\mz[1]{\textcolor{OliveGreen}{\textbf{Matt: }#1}}

% abbreviations 
\newcommand{\qbin}[2]{\begin{bmatrix}{#1}\\ {#2}\end{bmatrix}_q}
\newcommand{\Fq}{\mathbb{F}_q}
\newcommand{\N}{\mathbb{N}}
\newcommand{\Z}{\mathbb{Z}}
\newcommand\Fqn[1]{\mathbb{F}_q^{#1}}
\newcommand{\nul}{\mathrm{nul}}
\newcommand{\bP}{\mathbb{P}}
\newcommand{\bE}{\mathbb{E}}
\newcommand{\ra}{\rightarrow}
\newcommand{\lra}{\longrightarrow}
\newcommand{\st}{\,|\,} % such that


\title{A Model for Random Chain Complexes}
\author{Michael J. Catanzaro \and and Matthew J. Zabka}
\thanks{M.J. Catanzaro, Iowa State University, \href{mjcatanz@iastate.edu}{mjcatanz@iastate.edu}}
\thanks{M. Zabka, Southwest Minnesota State University, \href{matthew.zabka@smsu.edu}{matthew.zabka@smsu.edu}}

\date{\today}
\begin{document}
\begin{abstract}
We introduce a model for random chain complexes over a finite
field. The randomness in our complex comes from choosing the entries in the
matrices that represent the boundary maps uniformly over $\mathbb{F}_q$,
conditioned on ensuring that the composition of consecutive boundary maps is
the zero map.  We then investigate the combinatorial and homological 
properties of this random chain complex.
\end{abstract}
\maketitle


\section{Introduction}

The tools of algebraic topology have become very popular in the
analysis of large data sets~\cite{carlsson_topology_2009, ghrist_barcodes:_2008,
latala_persistent_2013}. Homological methods arising from
topology are scale-invariant, non-parametric, and importantly,
robust with respect to noise. This is increasingly important
for data arising from real-world applications, especially those
with low signal-to-noise ratio. All of these properties
leads one to investigate the properties of random topological
phenomena. The main goal of such theoretical considerations is to
develop a better understanding of noise and noise models, appropriate
for accurate data modelling in these contexts.

There have been a variety of attempts to randomize topological
constructions. Most famously, Erdos and Renyi introduced a model
for random graphs~\cite{erdos_random_1959, erdos_evolution_1960}. 
This work spawned an entire industry of probabilistic
models and tools used for understanding other random topological
and algebraic phenomenon. We continue this tradition by introducing
a new model for random chain complexes.

Chain complexes are algebraic constructions used to measure a variety of 
different properties. Their usefulness lies in providing a pathway
for homological algebra computations. They arise in a variety of contexts,
including group cohomology, Hoschild homology, de Rham cohomology, 
resolutions of commutative algebra and algebraic geometry, as well
as the main computational tool of algebraic topology. Specifically,
chain complexes measure the relationship between cycles and boundaries
of a topological space. This relationship underlies many topological
properties of interest, and is precisely what homology reveals. 

%Topology's tools have become very popular for analyzing data, and as one
%usually regards data as a random vector, some have attempted to apply
%randomness to topological ideas.  For example, Ginzburg and Pasechnik
%\cite{ginzburg2017random} have investigated a random chain complex with
%constant differential, while Zabka \cite{zabka2018random} has investigated a
%random Bockstein operator.  Both of these papers have investigated their topics
%in a strictly algebraic setting, and in this paper, we too shall investigate a
%random chain complex in a strictly algebraic setting.
%
%Chain complexes arise in topology as an algebraic measure in different
%dimensions of the relationship between the cycles and boundaries of a
%topological space. In particular, a chain complex defined on a space gives us a
%way to calculate that space's homology groups.

\subsection*{Main Results}

Let $R$ be a ring.  Formally, a {\em chain complex $(C_n, \delta_n)$ } is a
sequence of $R$-modules, denoted $C_n$, together with a sequence of linear
transformations 
\[
  \cdots \xrightarrow{\delta_{n+1}} C_n \xrightarrow{\delta_n}
  C_{n-1} \xrightarrow{\delta_{n-1}} \cdots
\]
such that $\delta_{n-1}\delta_n = 0$ for all $n$. The maps $\delta_n$ are called the boundary maps of the chain
complex, and the condition $\delta_{n-1} \delta_n = 0$ is known as the boundary
condition; see~\cite{hatcher2002algebraic} for further details.

%Formally, a {\em chain complex $(C_n, \delta_n)$ } is a sequence of modules,
%denoted $C_n$, and a sequence of linear transformations $\delta_n : C_n \to
%C_{n-1}$ that satisfy the `boundary' condition $\delta_{n-1}\delta_n = 0$ for
%all $n$. The $\delta_n$ are usually called the boundary maps of the chain
%complex. An interested reader can see \cite{hatcher2002algebraic} for further
%details.
%
%Let $q$ be a prime number and let $\Fq$ denote the field with $q$ elements.
%Consider the sequence of vector spaces $\Fqn{n_m}$ indexed by $m$ in the
%integers.  Let $A_m$ be a random sequence of $(n_{m-1})\times (n_m)$ matrices
%whose entries are chosen i.i.d. uniformly from $\Fq$, subject to the condition
%that the product of consecutive matrices is zero.  We then consider the random
%chain complex $(\Fqn{n_m},A_m)$. That is, we consider
%\[
%  \cdots \stackrel{A_{m+1}}{\lra} \Fqn{n_m} \stackrel{A_m}{\lra} \Fqn{n_{m-1}} 
%  \stackrel{A_{m-1}}{\lra} \cdots 
%\]
%where $A_{i+1}A_i=0$ for every $i$. 

Let $q$ be a prime number and let $\Fq$ denote the field with $q$ elements.
We build a random chain complex as follows (see Definition~\ref{defn:random_chain_cx}
for a precise statement). First, pick a 
sequence of integers $(n_m)$, where $m \in \N$. Next, we inductively
build random linear transformations $A_m : \Fqn{n_m} \lra \Fqn{n_{m-1}}$
for all $m$, subject to the contraint $A_{m-1} A_m = 0$. Fix the 
standard basis for $\Fqn{}$, so that it suffices to construct random
matrices $A_m \in M_{n_{m-1} \times n_m}(\Fqn{})$. We do so by 
chooising matrix entries i.i.d. from the uniform distribution on
$\Fqn{}$. 

%Consider the sequence of vector spaces $\Fqn{n_m}$ indexed by $m$ in the
%integers.  Let $A_m$ be a random sequence of $(n_{m-1})\times (n_m)$ matrices
%whose entries are chosen i.i.d. uniformly from $\Fq$, subject to the condition
%that the product of consecutive matrices is zero.  We then consider the random
%chain complex $(\Fqn{n_m},A_m)$. That is, we consider
%\[
%  \cdots \stackrel{A_{m+1}}{\lra} \Fqn{n_m} \stackrel{A_m}{\lra} \Fqn{n_{m-1}} 
%  \stackrel{A_{m-1}}{\lra} \cdots 
%\]
%where $A_{i+1}A_i=0$ for every $i$. 

We have two main results. We first show that, as $q$ goes to infinity, homology
is concentrated in dimension zero.  We then derive an explicit formula for the
distribution of the Betti numbers.


\subsection*{Related Work} There is another notion of random chain complex in the 
literature. In~\cite{ginzburg2017random}, Ginzburg and Pasechnik investigate
a random chain complex in which the vector spaces are 

%---------------------------------------------------
\section{Preliminaries}
This section consists of several lemmas that count the number of elements in
various sets related to finite vector spaces over $\mathbb{F}_q$.  We provide
proofs for these lemmas, and the interested reader can
see~\cite{stanley2011enumerative} for further details. 
\begin{lemma}\label{NumkTup}
  The number of ordered, linearly independent $k$-tuples of vectors in $\Fqn{n}$  is
\[
\prod_{j=0}^{k-1} \left(q^n - q^j\right) = 
(q^n-1)(q^n-q)(q^n-q^2)\cdots(q^n - q^{k-2})(q^n-q^{k-1}).
\]
\end{lemma}

\begin{proof}
Since first vector in the $k$-tuple may be any vector except for the zero vector, there are $q^n-1$ choices for the first vector.  For $1<m \leq k$, the $m$-th vector in the $k$-tuple may be any vector that is not a linear combination of the previously chosen $m-1$ vectors. So there are $q^n - q^{m-1}$ choices for the $m$-th vector.
\end{proof}

The number $\qbin{n}{k}$ defined in the next theorem is known as the $q$-binomial coefficient.

\begin{lemma}\label{NumkSub}
  The number of $k$-dimensional subspaces of $\Fqn{n}$ is
\[
%\qbin{n}{k} = \frac{(q^n-1)(q^n-q)(q^n-q^2)\cdots(q^n - q^{k-2})(q^n-q^{k-1})}{(q^k-1)(q^k-q)(q^k-q^2)\cdots(q^k - q^{k-2})(q^k-q^{k-1})}.
  \qbin{n}{k} = \prod_{j=0}^{k-1} \frac{q^n-q^j}{q^k-q^j} \, .
\]
\end{lemma}


\begin{proof}
  Let $\qbin{n}{k}$ denote the number of $k$-dimensional subspaces of $\Fqn{n}$
  and $N(q,k)$ be the number of ordered, linear independent $k$-tuples of
  vectors in $\Fqn{n}$.  Then Lemma~\ref{NumkTup} gives
  \begin{equation}
    \label{Nqk1}
%    N(q,k) = (q^n-1)(q^n-q)(q^n-q^2)\cdots(q^n - q^{k-2})(q^n-q^{k-1}).
    N(q,k) = \prod_{j=0}^{k-1} q^n-q^j \, .
  \end{equation}
  We may also find $N(q,k)$ another way: first choose a $k$-dimensional
  subspace and then choose the independent vectors in our $k$-tuple from the
  chosen subspace.  There are $\qbin{n}{k}$ $k$-dimensional subspaces of
  $\Fqn{n}$.  There are $q^k-1$ choices for the first vector in the
  $k$-tuple, and, for $1<m \leq k$, there are $q^k - q^{m-1}$ vectors for the
  $m$-th vector in the $k$-tuple.  Thus
  \begin{equation}\label{Nqk2}
  %  N(q,k) = \qbin{n}{k}(q^k-1)(q^k-q)(q^k-q^2)\cdots(q^k - q^{k-2})(q^k-q^{k-1}).
    N(q,k) = \qbin{n}{k}\prod_{j=0}^{k-1} q^k - q^j \, .
  \end{equation}
  Equations~\eqref{Nqk1}~and~\eqref{Nqk2} give the desired result.
\end{proof}

Using Lemmas~\ref{NumkTup}~and~\ref{NumkSub}, we can count the number of matrices
with of a given rank. %The following lemma will aid us in Section
%\ref{SecCondComp}.

\begin{lemma}\label{Num_mbyn_rankr}
The number of $m\times n$ matrices of rank $r$ with entries in $\Fq$ is given by
\begin{equation*}
%	&\qbin{m}{r}(q^n-1)(q^n-q)\cdots (q^n-q^{r-1})\\
%   =& \qbin{n}{r}(q^m-1)(q^m-q)\cdots (q^m-q^{r-1})\\
%   =& \frac{(q^m-1) (q^m-q) \cdots (q^m-q^{r-1})\cdot
%            (q^n-1) (q^n-q) \cdots (q^n-q^{r-1})}
%      {(q^r-1)(q^r - q)(q^r - q^2)\cdots (q^r - q^{r-2})(q^r-q^{r-1})} \\
%      =& \prod_{j=0}^{r-1} \frac{(q^m-q^j) (q^n - q^j)}{q^r - q^j} \, .
      \prod_{j=0}^{r-1} \frac{(q^m-q^j) (q^n - q^j)}{q^r - q^j} \, .
\end{equation*}
\end{lemma}
\mc{Redo this statement and proof. This proof uses 'homogeneity' of
  rank $r$ subspaces of vector space over finite fields, i.e., all
  that matters is the rank, not the actual subspace. Should we point
this out?}
\begin{proof}
  Let $W$ be a fixed $r$-dimensional subspace of $\Fqn{n}$.  The number of
  matrices whose column space is $W$ is given by the number of $r\times n$
  matrices with rank $r$.  This number is given by Lemma \ref{NumkTup}. The
  number of $r$-dimensional subspaces of $\Fqn{n}$ is $\qbin{m}{r}$, as stated
  in Lemma \ref{NumkSub}.  The product of these is the number of $m\times n$
  rank $r$ matrices.
\end{proof}

%We now turn our attention to a sequence of random matrices.

\begin{definition}\label{defPkmr}
Let $n_m$ be a sequence of natural numbers. Let $B_m$ be a sequence of random $(n_m) \times (n_{m-1})$ matrices whose entries are chosen i.i.d. uniformly from $\Fq$. Let $r$ be a non-negative integer.  Define 
\[
  P^m_k(r) := \mathbb{P} 
  \left[\rank(B_{m+1}) = r|B_{m}B_{m+1} = 0, \nul(B_{m}) = k \right] \, .
\]
\end{definition}

%Lemma~\ref{Num_mbyn_rankr} gives us the following.

\begin{lemma}\label{lemPkmr} With $B_m$ defined as in Definition \ref{defPkmr}, we have that
  \begin{equation}
    P^m_k(r) = 
    \begin{cases}
      {\displaystyle 
      q^{-kn_{m+1}}\prod_{j=0}^{r-1} \frac{(q^{n_{m+1}}-q^j) (q^k - q^j)}{q^r - q^j} }
%	\frac{\left(\prod_{j=0}^{r-1}\left(q^{n_{m+1}}-q^{j}\right)\right)
%	\left(\prod_{j=0}^{r-1}\left(q^k - q^j \right) \right)}
%      {q^{kn_{m+1}} \left(\prod_{j=0}^{r-1} \left(q^r-q^j\right)\right)}}
      & \textrm{ if } k\neq 0, r \leq k, \\
           0	& \textrm{ if }  r>k,\\
           1	& \textrm{ if } r = k = 0.
	 \end{cases}
	 \label{eqn:Pmkr}
       \end{equation}
\end{lemma}
\begin{proof}
Let $k= \nul(B_{m})$ and suppose $B_mB_{m+1} = 0$.  Then $B_{m+1}$ maps $\Fq^{n_{m+1}}$ into the kernel of $B_m$, and thus $\rank(B_{m+1}) \leq k$.  Therefore, if $r>k$, we have $P^m_k(r) = 0$.  Further, if $k=0$, then $\rank(B_{m+1}) = 0$, so $P^m_0(0) = 1$.

On the other hand, suppose $k\neq 0$ and $r\leq k$.  Then, $B_{m+1}$ represents a linear transformation from $\Fq^{n_{m+1}}$ into a $k$-dimensional subspace of $\Fq^{n_{m}}$. Thus, by changing basis, $B_{m+1}$ can be represented by a $(n_{m+1}) \times k$ matrix.  There are $q^{kn_{m+1}}$ such matrices, and, by Lemma \ref{Num_mbyn_rankr}, there are 
\[
% \displaystyle \frac{\left(\prod_{j=0}^{r-1}\left(q^{n_{m+1}}-q^{j}\right)\right)
%  \left(\prod_{j=0}^{r-1}\left(q^k - q^j \right) \right)}
%  {\prod_{j=0}^{r-1} \left(q^r-q^j\right)}
  \prod_{j=0}^{r-1} \frac{(q^{n_{m+1}}-q^j) ( q^k - q^j)}{q^r-q^j}
\]
such matrices of rank $r$.
\end{proof}

%---------------------------------------------------
\section{The Homology of a Random Chain Complex}\label{SecCondComp}
%Recall that a chain complex consists of a pair of sequences $(C_n, \delta_n)$,
%where the $C_n$ are appropriate spaces (vector spaces, groups, modules, etc.)
%and the $\delta_n$ are maps, $\delta_n: C_n \to C_{n-1}$ such that
%$\delta_{n-1}\delta_n = 0$.  
%
%Let $q$ be a prime number and let $\Fq$ denote the field with $q$ elements. Let
%$n_m$ be a sequence of natural numbers. Consider the sequence of finite vector
%spaces $\Fqn{n_m}$ indexed by $m$.  We iteratively construct a sequence $A_m$
%of boundary maps from $\Fq^{n_m}$ into $\Fq^{n_{m-1}}$.

\begin{definition} 
  \label{defn:random_chain_cx}
  Let $n_m$ be a sequence of natural numbers and let $A_0:\Fq^{n_0} \to 0$ be the zero map. 
  Suppose for $m>0$ that
  $A_{m}:\Fq^{n_m} \to \Fq^{n_{m-1}}$ is known. Let $A_{m+1}$ be the random linear
  transformation given by the random $(n_{m})\times (n_{m+1})$ matrix whose entries are
  chosen i.i.d. uniformly from $\Fq$, subject to the condition that $A_{m+1}A_{m}
  = 0$.
  A {\bf random chain complex} is a sequence of pairs $(\Fqn{n_m},A_m)$ for $m \in \Z$.
\end{definition}

\mc{What do you think of the following more general definition?}

\begin{definition}
  Let $n_m$ be a sequence of natural numbers indexed by $m \in \Z$. 
  A {\bf random chain complex} is a sequence of pairs $(\Fqn{n_m},A_m)$
  such that each $A_m: \Fqn{n_{m+1}} \ra \Fqn{n_m}$ is a 
  random matrix whose entries are chosen i.i.d. uniformly from $\Fqn{}$,
  subject to the condition $A_{m+1}A_m = 0$, for all $m \in \Z$.
\end{definition}

\mc{We could even get crazier, and we might want to for our CW complexes
  stuff. Define a {\em model for a random chain complex} to be the above
  together with a probability distribution on $\Fqn{}$. Then the above 
  is the model with the uniform distribution. This might be more interesting
  topologically since we might want to do things like Bernoulli with a small
  parameter, so that our spaces aren't uniformly complicated. Maybe
  we want to build spaces whose attaching maps tends to have smaller degree, rather
  than uniform on $\Fqn{}$. Importantly, we wouldn't
  use anything other than the uniform distribution here, but still, we would
be the first to make the definition.}





We wish to investigate the probabilistic properties of the homology of a random
chain complex.  We are primarily interested in the distribution of the Betti
numbers $\beta_m = \ker A_m - \im A_{m+1}$.
%:= \dim(H_m(A_\ast; \Fqn{n}))$.  Recall that
%$H_m(A_\ast;\Fqn{n}) = \ker (A_{m})/ A_{m+1}(\Fqn{n})$, so if $k= \ker(A_m)$,
%then $\beta_m = k - \rank(A_{m+1})$.  We are therefore interested in the
%probabilistic properties of $\rank(A_{m+1})$ given that $A_{m}A_{m+1} = 0$ and
%that $\nul(A_{m}) = k$. 


\begin{remark}
As the $A_m$ of a random chain complex are uniformly distributed, Definition
\ref{defPkmr} immediately gives us that
\[
  P^m_k(r) = \mathbb{P}\left[\beta_m = k-r \st \nul(A_{m}) = k\right].
\]
\end{remark}

\mz{If we know anything about $\lim_{q\to\infty} \Fq$, we should state it here.}


We are now ready to state our first theorem.


\begin{lemma}
  \label{lem:condqtoinfty}
Let $\beta_m$ be the $m$-th Betti number of a random chain complex.  If $k\leq n_{m+1}$, then
\[
\mathbb{P}\left[\beta_m= 0 \st \nul(A_{m}) = k \right] \to 1 \textrm{ as } q\to\infty.
\]
\end{lemma}
\mz{I had to fix a bunch of the indexing here.  Please double check!}
\begin{proof}
We have
	\begin{align*}
	\mathbb{P}\left[\beta_m = 0 \st A_{m}A_{m+1} = 0, \nul(A_{m}) = k \right]
    =& P^m_k(k)\\
    =& q^{-kn_{m+1}}\prod_{j=0}^{k-1} \frac{(q^{n_m+1}-q^j) (q^k - q^j)}{q^k - q^j}  \\
		=& \frac{1}{q^{kn_{m+1}}} \prod_{j=0}^{k-1}(q^{n_{m+1}} - q^j) \\
		%=& \frac{\prod_{j=0}^{k-1}(q^{n_{m+1}} - q^j)}{q^{n_{m+1}k}}\\
		=& \frac{q^{kn_{m+1}}}{q^{kn_{m+1}}}\prod_{j=0}^{k-1}(1-q^{j-n_{m+1}})\\
		=& \prod_{j=0}^{k-1} (1-q^{j-n_{m+1}}).
	\end{align*}
Since the index $j$ is less than $k$, and since $k$ is less than $n_{m+1}$, this product tends to $1$ as $q\to\infty$.  
\end{proof}

\mz{Exposition here regarding if $k > n_{m+1}$, then the smallest possible rank of $H_m$ is greater than $0$? Maybe an example?}

\begin{lemma}
  \label{lem:condqtoinfty2}
Let $\beta_m$ be the $m$-th Betti number of a random chain complex.  If $k > n_{m+1}$, then
\[
\mathbb{P}\left[\beta_m= k - n_{m+1} \st \nul(A_{m}) = k \right] \to 1 \textrm{ as } q\to\infty.
\]
\end{lemma}

\begin{proof}
Since $k$ must be less than or equal to $n_m$, we have that $n_{m+1} < k \leq n_m$.  So
	\begin{align*}
	\mathbb{P}\left[\beta_m = k - n_{m+1} \st A_{m}A_{m+1} = 0, \nul(A_{m}) = k \right]
    =& P^m_k(n_{m+1})\\
    =& q^{-kn_{m+1}}\prod_{j=0}^{n_{m+1}-1} \left(\frac{(q^{n_{m+1}} - q^j)(q^k - q^j)}{q^{n_{m+1}} - q^j}  \right)  \\
    =&  \prod_{j=0}^{n_{m+1}-1}(1-q^{j-k}).
	\end{align*}
Since the index $j$ is less than to $n_{m+1}$, which is less than $k$, this product goes to $1$ as $q$ goes to infinity.
\end{proof}

\mc{I've made this a separate lemma for now. Maybe we only need this, and not the previous
two lemmas at all. Or maybe not.}

\begin{lemma}
  Fix $m$ and $k$. Then 
  \[
    \lim_{q \ra \infty} P^m_k(r) = 
      \begin{cases}
        1 & \mbox{ if } r = \min(k,n_{m+1}) \, , \\
        0 & \mbox { else.}
      \end{cases}
    \]
\end{lemma}

\begin{proof}
  Reproduce the two previous proofs, starting from $P^m_k(k)$ and $P^m_k(n_{m+1})$.
\end{proof}



We are now in a position to prove Theorem~\ref{thm:qtoinfty}.

\begin{proof}[Proof of Theorem~\ref{thm:qtoinfty}]
  Consider the law of total probability
  \begin{equation}
    \label{eqn:totalbm}
    \bP[\beta_m = j] = \sum_{k=0}^{n_m} \bP[\beta_m = j \st \nul(A_m) = k]
    \bP[\nul(A_m) = k] \, .
  \end{equation}
   Since $0 \leq \nul(A_m) \leq n_m$, we have that $\bP[\nul(A_m) = k]$ is
   finite for all $k$ and for all primes $q$.  It suffices to show that 
   $\bP[\beta_m =N_m \st \nul(A_m) = k] \ra 1$ as $q \ra \infty$.
   
   By Lemma~\ref{lem:condqtoinfty}, every
   term of Eq.~\eqref{eqn:totalbm} with $k\leq n_{m+1}$ tend to 0 as $q \ra \infty$.
   On the other hand, if $k>n_{m+1}$, there are two cases to consider.
   If $n_m \leq n_{m+1}$, then $n_m \leq n_{m+1} < k$, which is a contradiction.
   Otherwise, $n_m > n_{m+1}$, in which case $N_m = 0$. Hence, we have
   \begin{align*}
     \bP[\beta_m = 0 \st \nul(A_m) = k] &= q^{-k n_{m+1}} \prod_{j=0}^{k-1}q^{n_{m+1}}-q^j \\
     &= \prod_{j=0}^{k-1} 1 - q^{j-n_{m+1}} \, ,
   \end{align*}
   which tends to 1 as $q \ra \infty$.
 \end{proof}




  



The previous theorem immediately leads to two corollaries.  
\mc{Fix the following two Corollaries.}

\begin{corollary}\label{condtozero}
Let $\beta_m$ be the $m$-th Betti number of a random chain complex. Let $b$ be a positive integer that is less than or equal to $k$. Then 
\[
\mathbb{P}[\beta_m = b \st \nul(A_{m}) = k ] \to 0 \textrm{ as } q\to\infty.
\]
\end{corollary}
\begin{proof}
We have that
	\begin{eqnarray*}
	&&\mathbb{P}[\beta_m = b \st A_{m}A_{m+1} = 0, \nul(A_{m}) = k ]\\
    &=& 1 - \sum_{j\neq b}\mathbb{P}[\beta_0 = j \st A_{m}A_{m+1} = 0, \nul(A_{m}) = k ]\\
    &\leq& 1 - \mathbb{P}[\beta_m = 0 \st A_{m}A_{m+1} = 0, \dim\ker(A_{m}) = k ].
	\end{eqnarray*}
By Lemma~\ref{lem:condqtoinfty}, this goes to $0$ as $q$ goes to infinity.
\end{proof}

\begin{corollary}
Let $\beta_m$ be the $m$-th Betti number of a random chain complex.  Then 
\[
\mathbb{E}[\beta_m \st \nul(A_{m}) = k ] \to  N_m\textrm{ as } q\to\infty.
\]
\begin{proof}
The conditional expectation of the $m$-th Betti number is given by
	\begin{eqnarray*}
	& & \mathbb{E}[\beta_m \st A_{m}A_{m+1} = 0, \dim\ker(A_{m}) = k ]\\
	&=& \sum_{b=1}^{n_m} b \mathbb{P}[\beta_m = b \st A_{m}A_{m+1} = 0, \dim\ker(A_{m}) = k ] \, .
	\end{eqnarray*}
By Corollary \ref{condtozero}, all terms with $b< n$ in this sum tend to 
$0$ as $q$ tends to infinity. 

On the other hand, when $b=n$, 
\[
n\mathbb{P}(\beta_m=0\st A_{m}A_{m+1} = 0, \dim\ker(A_{m}) = k ),
\]
which goes to $n$ has $q$ tends to infinity by Lemma~\ref{lem:condqtoinfty}.
\end{proof}
\end{corollary}

We next derive explicit formulas for the distributions of the Betti numbers of a random chain complex. We first need a lemma

\begin{lemma}\label{lemProbranks}
Let $(\Fq^{n_m}, A_m)$ be a random chain complex with $A_0:\Fq^{n_0}\to 0$.  Then
\begin{align*}
 &  \mathbb{P}\left[\rank(A_{m+1}) = n_{m+1} - k\right]\\
=& 	\sum_{i_m=0}^{n_m} P_{i_m}^{m}\left(n_{m+1} -k\right)
	\sum_{i_{m-1}=0}^{n_{m-1}} P_{i_{m-1}}^{m-1}\left(n_m - i_m\right)
		\cdots
	\sum_{i_1 = 0}^{n_1} P_{i_1}^1\left(n_2 - i_2\right) P_{n_0}^0 \left(n_1 - i_1\right) 
\end{align*}
\end{lemma}
\begin{proof}
The proof is by induction on $m$. For the basis step $m=1$, we have
\begin{align*}
\mathbb{P}\left[\rank(A_1) = n_1 - k\right]
	=& \sum_{i_0=0}^{n_0}\mathbb{P}\left [\rank(A_1)= n_1 - k\st\nul (A_0) = i_0\right]
		\mathbb{P}\left[\nul(A_0) = i_0 \right]\\
	=& \mathbb{P}\left [\rank(A_1)= n_1 - k\st\nul (A_0) = n_0\right]\\
	=& P_{n_0}^0(n_1 - k).
\end{align*}
The first equality follows by the Law of Total Probability, and the second equality follows because $A_0$ is the zero map.

For the inductive step, suppose that
\begin{align*}
 &  \mathbb{P}\left[\rank(A_{m}) = n_{m} - i_{m}\right]\\
=& 	\sum_{i_{m-1}=0}^{n_{m-1}} P_{i_{m-1}}^{m-1}\left(n_{m} -i_{m}\right)
	\sum_{i_{m-2}=0}^{n_{m-2}} P_{i_{m-2}}^{m-2}\left(n_{m-1} - i_{m-1}\right)
		\cdots
	\sum_{i_1 = 0}^{n_1} P_{i_1}^1\left(n_2 - i_2\right) P_{n_0}^0 \left(n_1 - i_1\right).
\end{align*}
We must show that
\begin{align*}
 &  \mathbb{P}\left[\rank(A_{m+1}) = n_{m+1} - k\right]\\
=& 	\sum_{i_m=0}^{n_m} P_{i_m}^{m}\left(n_{m+1} -k\right)
	\sum_{i_{m-1}=0}^{n_{m-1}} P_{i_{m-1}}^{m-1}\left(n_m - i_m\right)
		\cdots
	\sum_{i_1 = 0}^{n_1} P_{i_1}^1\left(n_2 - i_2\right) P_{n_0}^0 \left(n_1 - i_1\right) 
\end{align*}
Note that
\begin{align*}
 &	\mathbb{P}\left[\rank(A_{m+1}) = n_{m+1} - k\right]\\
=& 	\sum_{i_{m}}^{n_{m}} \mathbb{P}\left[ \rank(A_{m+1}) = n_{m+1} - k  
\st \nul(A_m) = i_m\right] \mathbb{P}\left[\nul(A_m) = i_m\right]\\
=&  \sum_{i_{m}}^{n_{m}} P_{i_m}^{n_m}(n_{m+1} - k)
		\mathbb{P}\left[n_m - \rank(A_m) = i_m\right]\\
=&	\sum_{i_{m}}^{n_{m}} P_{i_m}^{n_m}(n_{m+1} - k)
		\mathbb{P}\left[\rank(A_m) = n_m- i_m\right].
\end{align*}
The desired result now follows by the induction hypothesis.
\end{proof}

\begin{theorem} Let $\beta_j$ be the $j$-th Betti number of the random chain complex $(\Fq^{n} , A_m)$. Then
	\[    
    \bP[\beta_j = b] = \sum_{i_j=0}^{n_{j}} P_{i_j}^{j}(i_j-b)
    \sum_{i_{j-1}=0}^{n_{j-1}} P_{i_{j-1}}^{j-1}\left(n_{j} -i_j\right)
		\cdots
	\sum_{i_1 = 0}^{n_1} P_{i_1}^1\left(n_2 - i_2\right) P_{n_0}^0 \left(n_1 - i_1\right).
    \]
\end{theorem}
\mc{re-write proof.}
\begin{proof}
We have

\begin{align*}
  \mathbb{P}[\beta_{m} = b] 
   &= \mathbb{P}[\nul(A_{m}) - \rank(A_{m+1}) = b]\\
   &= \mathbb{P}[\rank(A_{m+1}) = \nul(A_{m}) - b]\\
   &= \sum_{k=0}^{n_{m}}\mathbb{P}[\rank(A_{m+1}) = k - b\st \nul(A_{m})=k] \mathbb{P}[\nul(A_{m}) = k]\\
   &= \sum_{k=0}^{n_{m}} P_{k}^{m}(k-b)\mathbb{P}[n_{m+1} - \rank(A_{m}) = k]\\
   &= \sum_{k=0}^{n_{m}} P_{k}^{m}(k-b)\mathbb{P}[\rank(A_{m+1}) = n_{m} - k]\\ 
\end{align*}
By Lemma \ref{lemProbranks} we have
\begin{align*}
 & \sum_{k=0}^{n_{m}} P_{k}^{m}(k-b)\mathbb{P}[\rank(A_{m+1}) = n_{m} - k] \\
=& \sum_{k=0}^{n_{m}} P_{k}^{m}(k-b)\sum_{i_m=0}^{n_m} P_{i_m}^{m}\left(n_{m+1} -k\right)
		\cdots
	\sum_{i_1 = 0}^{n_1} P_{i_1}^1\left(n_2 - i_2\right) P_{n_0}^0 \left(n_1 - i_1\right), 
\end{align*}
as desired.
\end{proof}

%-------------------------------------------------
\section{Proof of Theorem B}
In this section, we analyze Theorem~\ref{thm:bettinum} under the 
limit $q \ra \infty$.

\begin{proposition}
  \label{prop:oneseq}
Let $I_m:= \{0,1,\ldots, n_m\}$ and let $I^{(j)}:= I_1\times\cdots \times I_j$.
Then for every natural number $j$, there exists exactly one $\mathbf{i}^\ast =
(i_1^\ast,\ldots, i_j^\ast)$ in $I^{(j)}$ such that 
\[
  P_{i_{j-1}^\ast}^{j-1}(n_j-i_j^\ast)\cdots
  P_{i_1^\ast}^1(n_2-i_2^\ast)P_{n_0}^0(n_1-i_1^\ast) \to 1 \, ,
\]
as $q\to\infty$. In particular, if we set $i_0^\ast = n_0$, for each $\ell$ in
$\{1,2,\ldots, j\}$, we have $i_\ell^\ast = (n_\ell - i_{\ell - 1}^\ast)_+$.
\end{proposition}

\begin{proof}
The proof is by induction on $j$.

Base step ($j=1$). By Lemma~\ref{lem:Pqtoinfty}, we have $P_{n_0}^0 (n_1 -
i_1^\ast) \to 1$ as $q\to\infty$ if and only if $n_1 - i_1^\ast = \min(n_0,
n_1)$.  That is, $i_1^\ast = (n_1 - n_o)_+ = (n_1 - i_0^\ast)_+$.

Inductive step. Assume there exists exactly one $(i_1^\ast,\ldots ,
i_{j-1}^\ast)$ in $I^{(j-1)}$, with $i_\ell = (n_\ell - i_{\ell-1}^\ast)_+$ for
$\ell$ in $\{1,2,\ldots, j-1\}$, such that 
\[
P_{i_{j-2}^\ast}^{j-2}(n_{j-1} - i_{j-1}^\ast) %P_{i_{j-3}^\ast}^{j-3}(n_{j-2} - i_{j-2}^\ast) 
\cdots P_{i_1^\ast}^1 (n_2 - i_2^\ast) P_{n_0}^0 (n_1 - i_1^\ast) \to 1 \, ,
\]
as $q\to\infty$.  By Lemma~\ref{lem:Pqtoinfty}, $P_{i_{j-1}^\ast}^{j-1}(n_j -
i_j^\ast) \to 1$ as $q \to \infty$ if and only if $n_j - i_j^\ast =
\min(i_{j-1}^\ast, n_j)$. That is, $i_j^\ast = (n_j - i_{j-1}^\ast)_+$.  
For $\mathbf{i} = (i_1^\ast,\ldots, i_{j-1}^\ast, i_j^\ast)$ in $I^{(j)}$, we
have
\[
P_{i_{j-1}^\ast}^{j-1}(n_j-i_j^\ast)P_{i_{j-2}^\ast}^{j-2}(n_{j-1}-i_{j-1}^\ast)\cdots P_{i_1^\ast}^1(n_2-i_2^\ast)P_{n_0}^0(n_1-i_1^\ast) \to 1
\]
as $q\to\infty$, as desired.
\end{proof}

\mc{As a corollary, we could put what happens to the rank as $q \ra \infty$. This 
is what I think is true. Let me know what you think.}

Proposition~\ref{prop:oneseq} has a number of immediate consequences.

\begin{corollary}
  Let $(\Fqn{n_m},A_m)$ be a uniform random chain complex. Then
  \[
    \bP[\rank(A_m) = n_m-(n_{m-1} -(n_{m-2} -( \cdots -(n_1-n_0)_+ \cdots )_+ )_+ )_+]
    \ra 1 \,
  \]
  as $q \ra \infty$.
\end{corollary}

\begin{proof}
  This follows immediately from Proposition~\ref{prop:oneseq} and Theorem~\ref{thm:qtoinfty}.
\end{proof}

\begin{proof}[Proof of Theorem~\ref{thm:qtoinfty}]
  By Theorem~\ref{thm:bettinum}, it is sufficient to show
  \[
    P^m_{i_m^\ast}(i_m^\ast-b) P_{i^\ast_{m-1}}^{m-1}\left(n_{m} -i_m^\ast\right) \cdots
    P_{i_1^\ast}^1\left(n_2 - i_2^\ast\right) P_{n_0}^0 \left(n_1 - i_1^\ast\right) \to 1 
  \]
  as $q \ra \infty$ for a single sequence $\mathbf{i}^\ast=(i_0^\ast, \ldots,
  i_m^\ast)$ and a single
  value of $b$. After choosing $\mathbf{i}^\ast$ as in Proposition~\ref{prop:oneseq},
  the value of $b$ is easily determined from Lemma~\ref{lem:Pqtoinfty} to be
	\begin{align*}
	b 	&= i_m^\ast - \min(i_m^\ast, n_{m+1})\\
		&= (-n_{m+1} + i_m^\ast)_+\\
		&= (-n_{m+1} + (n_m -i_{m-1}^\ast)_+)_+\\
		&= (-n_{m+1} + (n_m - (n_{m-1} - (\cdots (n_1 - n_0)_+ \cdots)_+)_+)_+ )_+\\
        &= B_m \, . \qedhere
	\end{align*}
\end{proof}

\mz{Do we want a corollary here for the case where $n_m$ is monotone increasing?}
\mc{I added the following corollary and a short proof. What do you think?}

\begin{corollary}
  \label{cor:inc}
  If $\{n_m\}$ is a monotone increasing sequence, then 
  \[
    \lim_{q \ra \infty} \bP[\beta_m = 0] = 1 \, .
  \]
\end{corollary}

\begin{proof}
  By direct inspection, we have
  \[
    (n_m - (n_{m-1} - ( \cdots (n_1 - n_0)_+ \cdots )_+)_+)_+ \leq n_m \, ,
  \]
  and hence $B_m = 0$.
\end{proof}

\begin{corollary}
  The $t$-th moments of the random variable $\beta_m$ satisfy
  \[
    \lim_{q \ra \infty} \mathbb{E}\left[\beta_m^t \right] = B_m^t \, .
  \]
\end{corollary}
% \begin{proof}
	% \begin{align*}
	% \lim_{q \ra \infty} \mathbb{E}\left[\beta_m^t \right]
	% &= \lim_{q \ra \infty} \sum_{b=0}^{n_m} b^t \bP\left[\beta_m = b \right]\\
	% &= \sum_{b=0}^{n_m} b^t \lim_{q \ra \infty} \bP\left[\beta_m = b \right]\\
	% &= B_m^t
	% \end{align*}
% \end{proof}



%--------------------------------------------------
%\section{Conjectures}
%\mc{Re-write this section. Include plots as evidence from previous results and
conjectures. Should this be an appendix?}

This section consists of a bunch of miscellaneous facts and conjectures that
either we should include or not, but either way, I think we should understand
them. In particular, I think these will be useful things to know to get a good
grasp on what's really going on. 

\begin{conjecture}
 Consider the following.
  \begin{enumerate}
    \item Fix $r$. As a function of $k$, $P^m_k(r)$ is decreasing on its support.
    \item Fix $k$. As a function of $r$, $P^m_k(r)$ is increasing on its support.
    \item $P^m_k(r)$ has a maximum when $k = r$, as either variable changes.
    \item $P^m_{r+1}(r) > P^m_r(r-1)$.
  \end{enumerate}
\end{conjecture}

\begin{question} 
  What is the max of $\bP[\beta_1=b]$, as a function of $b$? For
  what $b$ is the max attained? The same questions for $\bP[\beta_2=b]$.
  \mc{I think our new stuff shows that $\bP[\beta_1=b]$ attains its maximum
    at $b = N_m$. Of course, this is in the limit as $q \ra \infty$, but maybe
    there's an argument that says for fixed $q$, $\bP[\beta_1=b]$ is a 
    decreasing function of $b$, and therefore it attains its maximum at
    the minimal value of $b$. And there should be nothing special to $\beta_1$
  here, and should hold for $\beta_k$ in general.}
\end{question}

\begin{question}
  What if $n_0 = w$ and $n>0 = 2w$? Intuitively, we would expect these numbers
  to not depend on the degree of the chain complex. Something similar should
  be true for any sequence of $(n_i)$ which converge.
\end{question}

\begin{question}
  What does this formula say if there exists $N >0$ so that $n_i=0$ for 
  all $i > N$? Specifically, $\bP[\beta_i=0] = 1$ and $\bP[\beta_i > 0]= 0$ 
  for $i>N$.
\end{question}

Using the same logic as the previous theorem, I'm {\em pretty sure} we have
the following.

%\begin{conjecture}
%  \[
%    \bP[\rank A_m = r_m] = \sum_{i_{m-1}=0}^{n_{m-1}} P^{m-1}_{i_{m-1}}(r_m)
%    \sum_{i_{m-2} = 0}^{n_{m-2}} P^{m-2}_{i_{m-2}}(n_{m-1}- i_{m-1})
%    \cdots \sum_{i_1 = 0}^{n_2}P^1_{i_1}(n_2-i_2) P^0_{n_0}(n_1-i_1)
%  \]
%\end{conjecture}


\begin{question} What distribution does the random variable
  $\rank(A_m)$ follow? What are its moments?
\end{question}

\begin{conjecture}
  Attempt to answer all of the following in cases:
  (i) $n_i$ is eventually constant, (ii) $n_i \sim o(i)$.
  \begin{enumerate}
    \item $\bP[\beta_m = 0] \ra ?$ as $m \ra \infty$.
    \item $\bP[\beta_m = j] \ra ?$ as $m \ra \infty$, $j>0$.
    \item $\bE[\beta_m] = ?$ as $m \ra \infty$.
  \end{enumerate}
\end{conjecture}



\bibliography{master}
\bibliographystyle{plain}
  

\end{document}
