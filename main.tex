\documentclass{amsart}
\usepackage[utf8]{inputenc}
\usepackage{amssymb, mathtools, amsthm}
\usepackage{graphicx}

\newtheorem{theorem}{Theorem}[section]
\newtheorem{corollary}[theorem]{Corollary}
\newtheorem{lemma}[theorem]{Lemma}
\newtheorem{conjecture}[theorem]{Conjecture}
\newtheorem{question}[theorem]{Question}

\DeclareMathOperator{\rank}{\mathrm{rank}}
\DeclareMathOperator{\im}{\mathrm{im}}
%\DeclareMathOperator{\ker}{\mathrm{ker}}



\usepackage[usenames,dvipsnames]{color}
 
% comments
\newcommand\mc[1]{\textcolor{NavyBlue}{\textbf{Mike: }#1}}
\newcommand\mz[1]{\textcolor{OliveGreen}{\textbf{Matt: }#1}}

% abbreviations 
\newcommand{\qbin}[2]{\begin{bmatrix}{#1}\\ {#2}\end{bmatrix}_q}
\newcommand{\Fq}{\mathbb{F}_q}
\newcommand\Fqn[1]{\mathbb{F}_q^{#1}}
\newcommand{\nul}{\mathrm{nul}}
\newcommand{\bP}{\mathbb{P}}
\newcommand{\bE}{\mathbb{E}}
\newcommand{\ra}{\rightarrow}
\newcommand{\lra}{\longrightarrow}
\newcommand{\st}{\,|\,} % such that


\title{A Model for Random Chain Complexes}
\author{Michael J. Catanzaro and Matthew Zabka}
\date{\today}
\begin{document}
\begin{abstract}
We introduce a random chain complex over a finite
field. The randomness in our complex comes from choosing the entries in the
matrices that represent the boundary maps uniformly over $\mathbb{F}_q$,
conditioned on ensuring that the composition of consecutive boundary maps is
the zero map.  We then investigate the combinatorial and homological 
properties of this random chain complex.
\end{abstract}
\maketitle


\section{Introduction}

The tools of algebraic topology have become very popular in the
analysis of large data sets~\cite{carlsson_topology_2009, ghrist_barcodes:_2008,
latala_persistent_2013}. Homological methods arising from
topology are scale-invariant, non-parametric, and importantly,
robust with respect to noise. This is increasingly important
for data arising from real-world applications, especially those
with low signal-to-noise ratio. All of these properties
leads one to investigate the properties of random topological
phenomena. The main goal of such theoretical considerations is to
develop a better understanding of noise and noise models, appropriate
for accurate data modelling in these contexts.

There have been a variety of attempts to randomize topological
constructions. Most famously, Erdos and Renyi introduced a model
for random graphs~\cite{erdos_random_1959, erdos_evolution_1960}. 
This work spawned an entire industry of probabilistic
models and tools used for understanding other random topological
and algebraic phenomenon. We continue this tradition by introducing
a new model for random chain complexes.

Chain complexes are algebraic constructions used to measure a variety of 
different properties. Their usefulness lies in providing a pathway
for homological algebra computations. They arise in a variety of contexts,
including group cohomology, Hoschild homology, de Rham cohomology, 
resolutions of commutative algebra and algebraic geometry, as well
as the main computational tool of algebraic topology. Specifically,
chain complexes measure the relationship between cycles and boundaries
of a topological space. This relationship underlies many topological
properties of interest, and is precisely what homology reveals. 

%Topology's tools have become very popular for analyzing data, and as one
%usually regards data as a random vector, some have attempted to apply
%randomness to topological ideas.  For example, Ginzburg and Pasechnik
%\cite{ginzburg2017random} have investigated a random chain complex with
%constant differential, while Zabka \cite{zabka2018random} has investigated a
%random Bockstein operator.  Both of these papers have investigated their topics
%in a strictly algebraic setting, and in this paper, we too shall investigate a
%random chain complex in a strictly algebraic setting.
%
%Chain complexes arise in topology as an algebraic measure in different
%dimensions of the relationship between the cycles and boundaries of a
%topological space. In particular, a chain complex defined on a space gives us a
%way to calculate that space's homology groups.

\subsection*{Main Results}

Let $R$ be a ring.  Formally, a {\em chain complex $(C_n, \delta_n)$ } is a
sequence of $R$-modules, denoted $C_n$, together with a sequence of linear
transformations 
\[
  \cdots \xrightarrow{\delta_{n+1}} C_n \xrightarrow{\delta_n}
  C_{n-1} \xrightarrow{\delta_{n-1}} \cdots
\]
such that $\delta_{n-1}\delta_n = 0$ for all $n$. The maps $\delta_n$ are called the boundary maps of the chain
complex, and the condition $\delta_{n-1} \delta_n = 0$ is known as the boundary
condition; see~\cite{hatcher2002algebraic} for further details.

%Formally, a {\em chain complex $(C_n, \delta_n)$ } is a sequence of modules,
%denoted $C_n$, and a sequence of linear transformations $\delta_n : C_n \to
%C_{n-1}$ that satisfy the `boundary' condition $\delta_{n-1}\delta_n = 0$ for
%all $n$. The $\delta_n$ are usually called the boundary maps of the chain
%complex. An interested reader can see \cite{hatcher2002algebraic} for further
%details.
%
%Let $q$ be a prime number and let $\Fq$ denote the field with $q$ elements.
%Consider the sequence of vector spaces $\Fqn{n_m}$ indexed by $m$ in the
%integers.  Let $A_m$ be a random sequence of $(n_{m-1})\times (n_m)$ matrices
%whose entries are chosen i.i.d. uniformly from $\Fq$, subject to the condition
%that the product of consecutive matrices is zero.  We then consider the random
%chain complex $(\Fqn{n_m},A_m)$. That is, we consider
%\[
%  \cdots \stackrel{A_{m+1}}{\lra} \Fqn{n_m} \stackrel{A_m}{\lra} \Fqn{n_{m-1}} 
%  \stackrel{A_{m-1}}{\lra} \cdots 
%\]
%where $A_{i+1}A_i=0$ for every $i$. 

Let $q$ be a prime number and let $\Fq$ denote the field with $q$ elements.
We build a random chain complex as follows (see Definition~\ref{defn:random_chain_cx}
for a precise statement). First, pick a 
sequence of integers $(n_m)$, where $m \in \N$. Next, we inductively
build random linear transformations $A_m : \Fqn{n_m} \lra \Fqn{n_{m-1}}$
for all $m$, subject to the contraint $A_{m-1} A_m = 0$. Fix the 
standard basis for $\Fqn{}$, so that it suffices to construct random
matrices $A_m \in M_{n_{m-1} \times n_m}(\Fqn{})$. We do so by 
chooising matrix entries i.i.d. from the uniform distribution on
$\Fqn{}$. 

%Consider the sequence of vector spaces $\Fqn{n_m}$ indexed by $m$ in the
%integers.  Let $A_m$ be a random sequence of $(n_{m-1})\times (n_m)$ matrices
%whose entries are chosen i.i.d. uniformly from $\Fq$, subject to the condition
%that the product of consecutive matrices is zero.  We then consider the random
%chain complex $(\Fqn{n_m},A_m)$. That is, we consider
%\[
%  \cdots \stackrel{A_{m+1}}{\lra} \Fqn{n_m} \stackrel{A_m}{\lra} \Fqn{n_{m-1}} 
%  \stackrel{A_{m-1}}{\lra} \cdots 
%\]
%where $A_{i+1}A_i=0$ for every $i$. 

We have two main results. We first show that, as $q$ goes to infinity, homology
is concentrated in dimension zero.  We then derive an explicit formula for the
distribution of the Betti numbers.


\subsection*{Related Work} There is another notion of random chain complex in the 
literature. In~\cite{ginzburg2017random}, Ginzburg and Pasechnik investigate
a random chain complex in which the vector spaces are 

%---------------------------------------------------
\section{Preliminaries}
This section consists of several lemmas that count the number of elements in
various sets related to finite vector spaces over $\mathbb{F}_q$.  We provide
proofs for these lemmas, and the interested reader can
see~\cite{stanley2011enumerative} for further details. 
\begin{lemma}\label{NumkTup}
  The number of ordered, linearly independent $k$-tuples of vectors in $\Fqn{n}$  is
\[
\prod_{j=0}^{k-1} \left(q^n - q^j\right) = 
(q^n-1)(q^n-q)(q^n-q^2)\cdots(q^n - q^{k-2})(q^n-q^{k-1}).
\]
\end{lemma}

\begin{proof}
Since first vector in the $k$-tuple may be any vector except for the zero vector, there are $q^n-1$ choices for the first vector.  For $1<m \leq k$, the $m$-th vector in the $k$-tuple may be any vector that is not a linear combination of the previously chosen $m-1$ vectors. So there are $q^n - q^{m-1}$ choices for the $m$-th vector.
\end{proof}

The number $\qbin{n}{k}$ defined in the next theorem is known as the $q$-binomial coefficient.

\begin{lemma}\label{NumkSub}
  The number of $k$-dimensional subspaces of $\Fqn{n}$ is
\[
%\qbin{n}{k} = \frac{(q^n-1)(q^n-q)(q^n-q^2)\cdots(q^n - q^{k-2})(q^n-q^{k-1})}{(q^k-1)(q^k-q)(q^k-q^2)\cdots(q^k - q^{k-2})(q^k-q^{k-1})}.
  \qbin{n}{k} = \prod_{j=0}^{k-1} \frac{q^n-q^j}{q^k-q^j} \, .
\]
\end{lemma}


\begin{proof}
  Let $\qbin{n}{k}$ denote the number of $k$-dimensional subspaces of $\Fqn{n}$
  and $N(q,k)$ be the number of ordered, linear independent $k$-tuples of
  vectors in $\Fqn{n}$.  Then Lemma~\ref{NumkTup} gives
  \begin{equation}
    \label{Nqk1}
%    N(q,k) = (q^n-1)(q^n-q)(q^n-q^2)\cdots(q^n - q^{k-2})(q^n-q^{k-1}).
    N(q,k) = \prod_{j=0}^{k-1} q^n-q^j \, .
  \end{equation}
  We may also find $N(q,k)$ another way: first choose a $k$-dimensional
  subspace and then choose the independent vectors in our $k$-tuple from the
  chosen subspace.  There are $\qbin{n}{k}$ $k$-dimensional subspaces of
  $\Fqn{n}$.  There are $q^k-1$ choices for the first vector in the
  $k$-tuple, and, for $1<m \leq k$, there are $q^k - q^{m-1}$ vectors for the
  $m$-th vector in the $k$-tuple.  Thus
  \begin{equation}\label{Nqk2}
  %  N(q,k) = \qbin{n}{k}(q^k-1)(q^k-q)(q^k-q^2)\cdots(q^k - q^{k-2})(q^k-q^{k-1}).
    N(q,k) = \qbin{n}{k}\prod_{j=0}^{k-1} q^k - q^j \, .
  \end{equation}
  Equations~\eqref{Nqk1}~and~\eqref{Nqk2} give the desired result.
\end{proof}

Using Lemmas~\ref{NumkTup}~and~\ref{NumkSub}, we can count the number of matrices
with of a given rank. %The following lemma will aid us in Section
%\ref{SecCondComp}.

\begin{lemma}\label{Num_mbyn_rankr}
The number of $m\times n$ matrices of rank $r$ with entries in $\Fq$ is given by
\begin{equation*}
%	&\qbin{m}{r}(q^n-1)(q^n-q)\cdots (q^n-q^{r-1})\\
%   =& \qbin{n}{r}(q^m-1)(q^m-q)\cdots (q^m-q^{r-1})\\
%   =& \frac{(q^m-1) (q^m-q) \cdots (q^m-q^{r-1})\cdot
%            (q^n-1) (q^n-q) \cdots (q^n-q^{r-1})}
%      {(q^r-1)(q^r - q)(q^r - q^2)\cdots (q^r - q^{r-2})(q^r-q^{r-1})} \\
%      =& \prod_{j=0}^{r-1} \frac{(q^m-q^j) (q^n - q^j)}{q^r - q^j} \, .
      \prod_{j=0}^{r-1} \frac{(q^m-q^j) (q^n - q^j)}{q^r - q^j} \, .
\end{equation*}
\end{lemma}
\mc{Redo this statement and proof. This proof uses 'homogeneity' of
  rank $r$ subspaces of vector space over finite fields, i.e., all
  that matters is the rank, not the actual subspace. Should we point
this out?}
\begin{proof}
  Let $W$ be a fixed $r$-dimensional subspace of $\Fqn{n}$.  The number of
  matrices whose column space is $W$ is given by the number of $r\times n$
  matrices with rank $r$.  This number is given by Lemma \ref{NumkTup}. The
  number of $r$-dimensional subspaces of $\Fqn{n}$ is $\qbin{m}{r}$, as stated
  in Lemma \ref{NumkSub}.  The product of these is the number of $m\times n$
  rank $r$ matrices.
\end{proof}

%We now turn our attention to a sequence of random matrices.

\begin{definition}\label{defPkmr}
Let $n_m$ be a sequence of natural numbers. Let $B_m$ be a sequence of random $(n_m) \times (n_{m-1})$ matrices whose entries are chosen i.i.d. uniformly from $\Fq$. Let $r$ be a non-negative integer.  Define 
\[
  P^m_k(r) := \mathbb{P} 
  \left[\rank(B_{m+1}) = r|B_{m}B_{m+1} = 0, \nul(B_{m}) = k \right] \, .
\]
\end{definition}

%Lemma~\ref{Num_mbyn_rankr} gives us the following.

\begin{lemma}\label{lemPkmr} With $B_m$ defined as in Definition \ref{defPkmr}, we have that
  \begin{equation}
    P^m_k(r) = 
    \begin{cases}
      {\displaystyle 
      q^{-kn_{m+1}}\prod_{j=0}^{r-1} \frac{(q^{n_{m+1}}-q^j) (q^k - q^j)}{q^r - q^j} }
%	\frac{\left(\prod_{j=0}^{r-1}\left(q^{n_{m+1}}-q^{j}\right)\right)
%	\left(\prod_{j=0}^{r-1}\left(q^k - q^j \right) \right)}
%      {q^{kn_{m+1}} \left(\prod_{j=0}^{r-1} \left(q^r-q^j\right)\right)}}
      & \textrm{ if } k\neq 0, r \leq k, \\
           0	& \textrm{ if }  r>k,\\
           1	& \textrm{ if } r = k = 0.
	 \end{cases}
	 \label{eqn:Pmkr}
       \end{equation}
\end{lemma}
\begin{proof}
Let $k= \nul(B_{m})$ and suppose $B_mB_{m+1} = 0$.  Then $B_{m+1}$ maps $\Fq^{n_{m+1}}$ into the kernel of $B_m$, and thus $\rank(B_{m+1}) \leq k$.  Therefore, if $r>k$, we have $P^m_k(r) = 0$.  Further, if $k=0$, then $\rank(B_{m+1}) = 0$, so $P^m_0(0) = 1$.

On the other hand, suppose $k\neq 0$ and $r\leq k$.  Then, $B_{m+1}$ represents a linear transformation from $\Fq^{n_{m+1}}$ into a $k$-dimensional subspace of $\Fq^{n_{m}}$. Thus, by changing basis, $B_{m+1}$ can be represented by a $(n_{m+1}) \times k$ matrix.  There are $q^{kn_{m+1}}$ such matrices, and, by Lemma \ref{Num_mbyn_rankr}, there are 
\[
% \displaystyle \frac{\left(\prod_{j=0}^{r-1}\left(q^{n_{m+1}}-q^{j}\right)\right)
%  \left(\prod_{j=0}^{r-1}\left(q^k - q^j \right) \right)}
%  {\prod_{j=0}^{r-1} \left(q^r-q^j\right)}
  \prod_{j=0}^{r-1} \frac{(q^{n_{m+1}}-q^j) ( q^k - q^j)}{q^r-q^j}
\]
such matrices of rank $r$.
\end{proof}

%---------------------------------------------------
\section{The Homology of a Conditional Random Chain Complex}\label{SecCondComp}
Let $q$ be a prime number and let $\Fq$ denote the field with $q$ elements. Let $n_m$ be a sequence of natural numbers. Consider the sequence of finite vector spaces $\Fqn{n_m}$ indexed by $m$ in the integers.  Let $A_m$ be a random sequence of $(n_{m-1})\times (n_m)$ matrices whose entries are chosen i.i.d. uniformly from $\Fq$, subject to the condition that the product of consecutive matrices is zero.  We then consider the random chain complex $(\Fqn{n_m},A_m)$. That is, we consider
\[
  \cdots \stackrel{A_{m+1}}{\lra} \Fqn{n_m} \stackrel{A_m}{\lra} \Fqn{n_{m-1}} 
  \stackrel{A_{m-1}}{\lra} \cdots 
\]
where $A_{i+1}A_i=0$ for every $i$. 

Having already constructed $A_{m}$, we may suppose that its kernel has dimension $k$.  Then the set of maps from $\Fqn{n}$ to $\ker A_{m}$ are in one-to-one correspondence with the set of $n\times k$ matrices with entries in $\Fqn{n}$, of which there are $q^{nk}$.
\mc{Is this our proof that it doesn't actually depend on which
vector subspace the kernel is? All that matters is its dimension?}

Next, we wish to investigate the $m$-th Betti number, $\beta_m := \dim(H_m(A_\ast; \Fqn{n}))$.  Recall that $H_m(A_\ast;\Fqn{n}) = \ker (A_{m})/ A_{m+1}(\Fqn{n})$, so $\beta_m = k - \rank(A_{m+1})$.  We are therefore interested in the probabilistic properties of $\rank(A_{m+1})$ given that $A_{m}A_{m+1} = 0$ and that $\dim\ker(A_{m}) = k$. 

Let $r \leq k$ be a non-negative integer and let 
\[
  P^m_k(r) := \mathbb{P} 
  \left[\rank(A_{m+1}) = r|A_{m}A_{m+1} = 0, \nul(A_{m}) = k \right] \, .
\]

Note that the above is equivalent to
\[
P^m_k(r) = \mathbb{P}\left[\beta_m = k-r | A_{m}A_{m+1} = 0, \nul(A_{m}) = k\right] \, .
\]

\mc{This depends on every matrix being uniformly distributed, which
  follows from the uniform distribution on $\Fq$. We should mention or 
remark this.}
Lemma~\ref{Num_mbyn_rankr} gives us the following.
\begin{lemma}
\[
P^m_k(r) = \begin{cases}
  {\displaystyle \frac{\left(\prod_{j=0}^{r-1}\left(q^{n_{m+1}}-q^{j}\right)\right)
  \left(\prod_{j=0}^{r-1}\left(q^k - q^j \right) \right)}
  {q^{kn_{m+1}} \left(\prod_{j=0}^{r-1} \left(q^r-q^j\right)\right)}}
            					& \textrm{ if } k\neq 0,\\
           0					&  \textrm{ if }  r>k. 
            \end{cases}
\]
\end{lemma}
\mz{We've defined $P_k(r)$ several ways above -- this needs to be fixed.}\\
\mc{I think we've fixed this now. Right?}                        


\mc{The following theorem is no longer true in general. It fails if
  $n_{m+1} < k$, but it is still true if $n_m = n_{m+1}$. One possible
  fix is to just impose $k \leq n_{m+1}$. This proposed fix may be 
  unnatural. In any case, there should be 
some condition on $k$ in the statement, since its a bound variable.}

\begin{theorem} 
Let $\beta_m$ be the $m$-th Betti number in the conditional chain complex.  Then 
\[
\mathbb{P}\left[\beta_m=0| A_{m}A_{m+1} = 0, \nul(A_{m}) = k \right] \to 1 \textrm{ as } q\to\infty.
\]
\end{theorem}

\begin{proof}\label{Condptoinfty}
We have
	\begin{eqnarray*}
	&&\mathbb{P}\left[\beta_m = 0|A_{m}A_{m+1} = 0, \nul(A_{m}) = k \right]\\ 
    &=& P^m_k(k)\\
    &=& \frac{\prod_{j=0}^{k-1}(q^{n_{m+1}}-q^{j})
		\prod_{j=0}^{k-1}(q^k - q^j )}
		{q^{kn_{m+1}} \prod_{j=0}^{k-1} (q^k-q^j)}\\
		&=& \frac{\prod_{j=0}^{k-1}(q^{n_{m+1}} - q^j)}
		{q^{kn_{m+1}}} \\
		%&=& \frac{\prod_{j=0}^{k-1}(q^{n_{m+1}} - q^j)}{q^{n_{m+1}k}}\\
		%&=& \frac{q^{kn_{m+1}}}{q^{kn_{m+1}}}\prod_{j=0}^{k-1}(1-q^{j-n_{m+1}})\\
		&=& \prod_{j=0}^{k-1} (1-q^{j-n_{m+1}}),
	\end{eqnarray*}
which tends to 1 as $q\to\infty$.  
\end{proof}

The previous theorem immediately leads to two corollaries.  

\begin{corollary}\label{condtozero}
Let $b$ be a positive integer that is less than or equal to $k$. Then 
\[
\mathbb{P}[\beta_m = b| A_{m}A_{m+1} = 0, \nul(A_{m}) = k ] \to 0 \textrm{ as } q\to\infty.
\]
\end{corollary}
\begin{proof}
We have that
	\begin{eqnarray*}
	&&\mathbb{P}[\beta_m = b \st A_{m}A_{m+1} = 0, \nul(A_{m}) = k ]\\
    &=& 1 - \sum_{j\neq b}\mathbb{P}[\beta_0 = j \st A_{m}A_{m+1} = 0, \nul(A_{m}) = k ]\\
    &\leq& 1 - \mathbb{P}(\beta_m = 0|A_{m}A_{m+1} = 0, \dim\ker(A_{m}) = k ).
	\end{eqnarray*}
By Theorem~\ref{Condptoinfty}, this goes to $0$ as $q$ goes to infinity.
\end{proof}

\begin{corollary}
$\mathbb{E}[\beta_m | A_{m}A_{m+1} = 0, \dim\ker(A_{m}) = k ] \to n$ as $q\to\infty$.
\begin{proof}
The conditional expectation of the $m$-th Betti number is given by
	\begin{eqnarray*}
	& & \mathbb{E}[\beta_m | A_{m}A_{m+1} = 0, \dim\ker(A_{m}) = k ]\\
	&=& \sum_{b=1}^n b \mathbb{P}(\beta_m = b | A_{m}A_{m+1} = 0, \dim\ker(A_{m}) = k ) \, .
	\end{eqnarray*}
By Corollary \ref{condtozero}, all terms with $b< n$ in this sum tend to 
$0$ as $q$ goes to infinity. 

On the other hand, when $b=n$, 
\[
n\mathbb{P}(\beta_m=0| A_{m}A_{m+1} = 0, \dim\ker(A_{m}) = k ),
\]
which goes to $n$ has $q$ goes to infinity by Theorem~\ref{Condptoinfty}.
\end{proof}
\end{corollary}

%------------------------------------------------
\section{A Random Chain Complex}
\mc{This section will be removed. Before we remove it, let's be 
certain we can't do anything with the final formula.}
In the previous section, we built a chain complex iteratively. That is, we
studied the properties of attaching an appropriate map to an already
constructed complex of length $m-1$. \mz{Rewrite this}

In this section, we assume we have already have a chain complex.  That is, we want to know the probabilistic properties of $H_m(A_\ast)$ knowing only that $A_{m-1}A_m=0$. \mz{Rewrite this too} That is, we wish to investigate
\[
\mathbb{P}[\rank(A_m) = r| A_{m}A_{m+1} = 0].
\] 

As in Section \ref{SecCondComp}, for $k\neq 0$ set
\[
P_k(r) :=  \mathbb{P}[\rank(A_{m+1})=r|A_{m}A_{m+1} = 0, \dim\ker (A_{m}) = k].
\]
and set $P_k(r) = 0$ whenever $r<k$.
Note that this does not depend on $m$. Then
	\begin{eqnarray*}
	&&\mathbb{P}[\rank(A_{m+1}) = r| A_{m}A_{m+1} = 0]\\
	&=& \sum_{k=0}^n \mathbb{P}[\dim\ker (A_{m}) = k|A_{m}A_{m+1} = 0]
    	\cdot P_k(r)  \textrm{ (law of tot. prob) }\\
	&=& \sum_{k=0}^n \mathbb{P}[\dim\ker (A_{m}) = k| \Fqn{n} \xrightarrow{A_{m+1}} \ker(A_{m})]
    	\cdot P_k(r)\\
	&=& \sum_{k=0}^n \mathbb{P}[\rank(A_{m}) = n-k | \Fqn{n} \xrightarrow{A_{m+1}} \ker(A_{m})]
    	\cdot P_k(r)\\
    &=& \sum_{k=0}^n \frac{\mathbb{P}[\rank(A_{m}) = n-k]
    \cdot \mathbb{P}(\Fqn{n} \xrightarrow{A_{m+1}} \ker(A_{m}) |\dim\ker(A_{m}) = k)}
    {\mathbb{P}(\Fqn{n} \xrightarrow{A_{m+1}} \ker(A_{m}))}
        \cdot P_k(r).
   \end{eqnarray*}
The last equality holds by Bayes' Rule. \mz{The magic should happen in the denominator of this fraction. It can be pulled out in front of the sum.  Everything in the numerator can be figured out using the lemmas in the counting section -- in fact, it seems to be constant over $m$.  See below:}

\[
  \frac{ \sum_{k=0}^n \mathbb{P}(\rank(A_{m}) = n-k) \cdot \mathbb{P}(\Fqn{n} \xrightarrow{A_{m+1}} \ker(A_{m}) |\dim\ker(A_{m}) = k)
    \cdot P_k(r)}
    {\mathbb{P}(\Fqn{n} \xrightarrow{A_{m+1}} \ker(A_{m}))}
\]

%\section{(Mike's probably wrong) Homology calculation}
%Let us step through the computation of the homology of a random
%chain complex degree by degree. In what follows, we assume a random
%chain complex $(\Fqn{n},A_*)$ exists.
%
%We ask for $\mathbb{P} [\beta_j= b_j]$, for each $j$. I'm kind of
%sloppy in sticking with either $\nul$ or $\rank$ in the following
%so I'm sure I use rank + nullity a lot without explicitly saying it.
%You've been warned...
%
%\subsection{$\beta_0$}
%
%\begin{align*}
%  \mathbb{P}[\beta_0 = b_0] &= \mathbb{P}[\rank(A_1) = n - b_0] \\
%   &= \frac{1}{q^{n^2}} \qbin{n}{n-b_0} \, (q^n-1) (q^n - q) \cdots (q^n - q^{n-b_0-1}) \, ,
%\end{align*}
%by Lemma~\ref{Num_mbyn_rankr}.
%
%\mc{Alternatively, we have $\bP[\beta_0=b_0] = 
%\bP[\beta_0=b_0 \, | \, \nul(A_0) = n]$, vacuously since
%$A_0$ is the zero map. But then
%$\bP[\beta_0 = b_0] = \bP_{n}(n-b_0)$, which may be obvious.
%Since it's late, I will only make this substitution in the 
%last conjecture.}
%
%  
%\subsection{$\beta_1$}  
%  
%\begin{align*}
%  	\mathbb{P}[\beta_1 = b_1] 
%  	&= \sum_{b=0}^n \mathbb{P} [ \beta_1 = b_1 \, | \, \beta_0=b] \cdot \mathbb{P}[\beta_0 = b] \\
%  	&= \sum_{b=0}^n \mathbb{P}[ \rank(A_2) = b-b_1 \, | \, \rank(A_1)=n-b] \cdot \mathbb{P}[ \rank(A_1) = n-b] \\
%  	&= \sum_{b=0}^n \mathbb{P}[ \rank(A_2) = b-b_1 \, | \, \nul(A_1)=b]  \cdot \mathbb{P}[ \rank(A_1) = n-b] \\
%  	&= \sum_{b=0}^n P_b(b-b_1) \bP[\beta_0=b] \\
%    &= \sum_{b=0}^n P_b(b-b_1) P_n(n-b)\, ,
%\end{align*}  
%where the last factor of the last equal sign is computed in the previous
%subsection.
%
%\subsection{$\beta_2$}
%
%\begin{align*}
%  	\bP[\beta_2 = b_2] 
%  	&= \sum_{c=0}^n \bP[\beta_2=b_2 \, | \, \nul(A_2) = c] \bP[\nul(A_2) = c] \\
%  	&= \sum_{c=0}^n \bP[\rank(A_3) = c-b_2 \, | \, \nul(A_2) = c] \bP[\nul(A_2) = c] \\
%  	&= \sum_{c=0}^n P_c(c-b_2) \bP[\rank(A_2) = n-c]
%\end{align*}
%
%Now we must compute this last factor $\bP[\rank(A_2) = n-c]$, so we 
%expand again:
%
%\begin{align*}
%  	\bP[\rank(A_2) = n-c] 
%    &= \sum_{b=0}^n \bP[\rank(A_2) = n-c \, | \,\nul(A_1) = b] \, \bP[\nul(A_1) = b] \\
%  	&=\sum_{b=0}^n P_b(n-c) \bP[\beta_0=b]
%\end{align*}
%
%Collecting these two, we get
%\[
%  \bP[\beta_2 = b_2] = \sum_{c=0}^n P_c(c-b_2) \sum_{b=0}^n P_b(n-c)
%  \bP[\beta_0=b]
%\]
%
%\subsection{$\beta_3$}
%So I think we understand the general story now. But just to be sure,
%let's do one more.
%
%\begin{align*}
%  	\bP[\beta_3 = b_3] 
%    &= \sum_{d=0}^n \bP[\beta_3=b_3 \, | \, \nul(A_3) = d] \, \bP[\nul(A_3) = d] \\
%  	&= \sum_{d=0}^n P_d(d-b_3) \bP[\rank(A_3) = n-d] \\
%  	&= \sum_{d=0}^n P_d(d-b_3) \, \sum_{c=0}^n P_c(n-d) \, \bP[\rank(A_2) = n-c] \\
%  	&= \sum_{d=0}^n P_d(d-b_3) \, \sum_{c=0}^n P_c(n-d) \, \sum_{b=0}^n P_b(n-c) \bP[\beta_0=b] \\
%\end{align*}
%
%\subsection{$\beta_j$}
%
%So here's the conjecture:
%
%\begin{align*}
%	\bP[\beta_j = b] 
%    &= \sum_{i_1=0}^n P_{i_1}(i_1-b) \sum_{i_2=0}^n P_{i_2}(n-i_1) \cdots \sum_{i_j=0}^n P_{i_j}(n-i_{j-1}) \bP[\beta_0=i_j] \\
% 	&= \sum_{i_1=0}^n P_{i_1}(i_1-b) \sum_{i_2=0}^n P_{i_2}(n-i_1) \cdots \sum_{i_j=0}^n P_{i_j}(n-i_{j-1}) P_n(n-i_j) \\
%\end{align*}
% 
%\mc{This formula may not be useful in any practical way. But it definitely
%`completes' the story in some sense. We should still do expectations
%and conditional probabilites.}

%---------------------------------------------------
\section{Appendix of Figures}
Here we put some figures.

The colors in the figures all follow the following scheme:

$\beta_1$ is blue, $\beta_2$ is red, $\beta_3$ is green, $\beta_4$ is orange,
$\beta_5$ is brown, $\beta_6$ is yellow, $\beta_7$ is pink.

I have removed all the figures from this file and I'll add them soon.

%\begin{figure}[h]
%%\centering
%  \includegraphics[width=\textwidth]{betti_plot.png}
%  \caption{This is a plot of $j$ versus $\log(\bP[\beta_i=j])$. 
%  $\beta_1$ is red, $\beta_2$ is blue, $\beta_3$ is green, 
%and $\beta_4$ is orange.}
%\end{figure}
%
%\begin{figure}[h]
%\centering
%\includegraphics[width=\textwidth]{betti_plotp2n5.png}
%\caption{$p=2$, $n=5$. This is a plot of $j$ versus $\log(\bP[\beta_i=j])$. 
%  $\beta_1$ is red, $\beta_2$ is blue, $\beta_3$ is green, 
%and $\beta_4$ is orange.}
%\end{figure}
%
%\begin{figure}[h]
%\centering
%\includegraphics[width=\textwidth]{betti_plotp3n7.PNG}
%\caption{$p=3$, $n=7$. This is a plot of $j$ versus $\log(\bP[\beta_i=j])$. 
%  $\beta_1$ is red, $\beta_2$ is blue, $\beta_3$ is green, 
%and $\beta_4$ is orange.}
%\end{figure}

%-------------------------------------------------
\mc{Re-write this section. Include plots as evidence from previous results and
conjectures. Should this be an appendix?}

This section consists of a bunch of miscellaneous facts and conjectures that
either we should include or not, but either way, I think we should understand
them. In particular, I think these will be useful things to know to get a good
grasp on what's really going on. 

\begin{conjecture}
 Consider the following.
  \begin{enumerate}
    \item Fix $r$. As a function of $k$, $P^m_k(r)$ is decreasing on its support.
    \item Fix $k$. As a function of $r$, $P^m_k(r)$ is increasing on its support.
    \item $P^m_k(r)$ has a maximum when $k = r$, as either variable changes.
    \item $P^m_{r+1}(r) > P^m_r(r-1)$.
  \end{enumerate}
\end{conjecture}

\begin{question} 
  What is the max of $\bP[\beta_1=b]$, as a function of $b$? For
  what $b$ is the max attained? The same questions for $\bP[\beta_2=b]$.
  \mc{I think our new stuff shows that $\bP[\beta_1=b]$ attains its maximum
    at $b = N_m$. Of course, this is in the limit as $q \ra \infty$, but maybe
    there's an argument that says for fixed $q$, $\bP[\beta_1=b]$ is a 
    decreasing function of $b$, and therefore it attains its maximum at
    the minimal value of $b$. And there should be nothing special to $\beta_1$
  here, and should hold for $\beta_k$ in general.}
\end{question}

\begin{question}
  What if $n_0 = w$ and $n>0 = 2w$? Intuitively, we would expect these numbers
  to not depend on the degree of the chain complex. Something similar should
  be true for any sequence of $(n_i)$ which converge.
\end{question}

\begin{question}
  What does this formula say if there exists $N >0$ so that $n_i=0$ for 
  all $i > N$? Specifically, $\bP[\beta_i=0] = 1$ and $\bP[\beta_i > 0]= 0$ 
  for $i>N$.
\end{question}

Using the same logic as the previous theorem, I'm {\em pretty sure} we have
the following.

%\begin{conjecture}
%  \[
%    \bP[\rank A_m = r_m] = \sum_{i_{m-1}=0}^{n_{m-1}} P^{m-1}_{i_{m-1}}(r_m)
%    \sum_{i_{m-2} = 0}^{n_{m-2}} P^{m-2}_{i_{m-2}}(n_{m-1}- i_{m-1})
%    \cdots \sum_{i_1 = 0}^{n_2}P^1_{i_1}(n_2-i_2) P^0_{n_0}(n_1-i_1)
%  \]
%\end{conjecture}


\begin{question} What distribution does the random variable
  $\rank(A_m)$ follow? What are its moments?
\end{question}

\begin{conjecture}
  Attempt to answer all of the following in cases:
  (i) $n_i$ is eventually constant, (ii) $n_i \sim o(i)$.
  \begin{enumerate}
    \item $\bP[\beta_m = 0] \ra ?$ as $m \ra \infty$.
    \item $\bP[\beta_m = j] \ra ?$ as $m \ra \infty$, $j>0$.
    \item $\bE[\beta_m] = ?$ as $m \ra \infty$.
  \end{enumerate}
\end{conjecture}



\bibliography{master}
\bibliographystyle{plain}
  

\end{document}
