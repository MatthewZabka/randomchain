\documentclass{amsart}
%\usepackage[utf8]{inputenc}
\usepackage{amssymb, mathtools, amsthm}
\usepackage{graphicx}
\usepackage{hyperref}
\usepackage{microtype}


\newtheorem{theorem}{Theorem}[section]
\newtheorem{corollary}[theorem]{Corollary}
\newtheorem{lemma}[theorem]{Lemma}
\newtheorem{definition}[theorem]{Definition}
\newtheorem{remark}[theorem]{Remark}
\newtheorem{conjecture}[theorem]{Conjecture}
\newtheorem{question}[theorem]{Question}
\newtheorem{bigthm}{Theorem}
\renewcommand{\thebigthm}{\Alph{bigthm}}

\DeclareMathOperator{\rank}{\mathrm{rank}}
\DeclareMathOperator{\im}{\mathrm{im}}
%\DeclareMathOperator{\ker}{\mathrm{ker}}

\usepackage{libertine,anyfontsize,libertinust1math}
\linespread{1.1}


\usepackage[usenames,dvipsnames]{color}
 
% comments
\newcommand\mc[1]{\textcolor{NavyBlue}{\textbf{Mike: }#1}}
\newcommand\mz[1]{\textcolor{OliveGreen}{\textbf{Matt: }#1}}

% abbreviations 
\newcommand{\qbin}[2]{\begin{bmatrix}{#1}\\ {#2}\end{bmatrix}_q}
\newcommand{\Fq}{\mathbb{F}_q}
\newcommand{\N}{\mathbb{N}}
\newcommand{\Z}{\mathbb{Z}}
\newcommand\Fqn[1]{\mathbb{F}_q^{#1}}
\newcommand{\nul}{\mathrm{nul}}
\newcommand{\bP}{\mathbb{P}}
\newcommand{\bE}{\mathbb{E}}
\newcommand{\ra}{\rightarrow}
\newcommand{\lra}{\longrightarrow}
\newcommand{\st}{\,|\,} % such that


\title{A Model for Random Chain Complexes}
\author{Michael J. Catanzaro and Matthew Zabka}
\date{\today}
\begin{document}
\begin{abstract}
We introduce a model of random chain complexes over a finite
field. The randomness in our complex comes from choosing the entries in the
matrices that represent the boundary maps uniformly over $\mathbb{F}_q$,
conditioned on ensuring that the composition of consecutive boundary maps is
the zero map.  We then investigate the combinatorial and homological 
properties of this random chain complex.
\end{abstract}
\maketitle


\mc{Do we want to choose $n_m$ randomly?}
\section{Introduction}
There have been a variety of attempts to randomize topological constructions.
Most famously, Erd\"os and R\'enyi introduced a model for random
graphs~\cite{erdos_random_1959}.  This work spawned an
entire industry of probabilistic models and tools used for understanding other
random topological and algebraic phenomenon. These include various models for
random simplicial complexes, random networks, and many
more~\cite{erdos_evolution_1960, linial_random_2017}. Further, this has
led to beautiful connections with statistical physics, for example through
percolation theory~\cite{bollobas_2006_percolation, broadbent_percolation_1957,
kesten_percolation_1982}.  Our ultimate goal is to understand higher
dimensional topological constructions arising in algebraic topology from a
random perspective. In this manuscript, we begin to address this goal with the
much simpler objective of understanding an algebraic construction commonly associated
with topological spaces, known as a chain complex. 

\mc{I re-worded the previous and following sentences.}

Chain complexes are used to measure a
variety of different algebraic, geoemtric, and topological properties Their usefulness lies in
providing a pathway for homological algebra computations. They arise in a
variety of contexts, including commutative algebra, algebraic geometry, group
cohomology, Hoschild homology, de Rham cohomology, and of course algebraic
topology~\cite{bott2013differential, brown2012cohomology,
hartshorne2013algebraic, hatcher2002algebraic, hochschild1945cohomology}.
Specifically, chain complexes measure the relationship between cycles and
boundaries of a topological space. This relationship uncovers many topological
properties of interest, and is precisely what homology reveals. Furthermore,
the singular chain complex of a topological space provides a canonical method
of associating a chain complex to a topological space.
%If the topological space has additional structure, like a CW or 
%simplicial decomposition, then the singular chain complex can be replaced
%by CW or simplicial chains, which are much easier to compute.


%Topology's tools have become very popular for analyzing data, and as one
%usually regards data as a random vector, some have attempted to apply
%randomness to topological ideas.  For example, Ginzburg and Pasechnik
%\cite{ginzburg2017random} have investigated a random chain complex with
%constant differential, while Zabka \cite{zabka2018random} has investigated a
%random Bockstein operator.  Both of these papers have investigated their topics
%in a strictly algebraic setting, and in this paper, we too shall investigate a
%random chain complex in a strictly algebraic setting.
%
%Chain complexes arise in topology as an algebraic measure in different
%dimensions of the relationship between the cycles and boundaries of a
%topological space. In particular, a chain complex defined on a space gives us a
%way to calculate that space's homology groups.


Let $R$ be a ring. A {\em chain complex $C_*=(C_m, \delta_m)$ with coefficients in
$R$} is a sequence of $R$-modules, denoted $C_m$, together with a sequence of
linear transformations 
\[
  \cdots \xrightarrow{\delta_{m+1}} C_m \xrightarrow{\delta_m}
  C_{m-1} \xrightarrow{\delta_{m-1}} \cdots
\]
such that $\delta_{m-1}\delta_m = 0$ for all $m \in \Z$.  The maps $\delta_m$
are called the boundary maps of the chain complex, and the equation
$\delta_{m-1} \delta_m = 0$ is known as the boundary condition;
see~\cite{cartan2016homological} for further details. 

The boundary condition $\delta_{m-1}\delta_m=0$ forces $\im \delta_m \subseteq \ker \delta_{m-1}$.
The {\em homology} of a chain complex measures the deviation of this containment
from equality:
\begin{equation*}
  \label{eqn:hom}
  H_m(C_*;R) = \frac{\ker \delta_{m-1}}{\im \delta_m} \, .
\end{equation*}
When the chain complex arises by taking singular chains on a topological
space, homology can be a very powerful tool in algebraic topology~\cite{hatcher2002algebraic}. 
\mc{Reworded the previous sentence.}

We work over the field with $q$-elements $R = \Fqn{}$ and consider the chain
complex whose $R$-modules are given by finite dimensional vector spaces,  $C_m
= \Fqn{n_m}$, where each $n_m \in \mathbb{N}$. After fixing the standard
basis for $\Fqn{}$, the boundary
maps can be regarded as $n_{m-1}\times n_m $ matrices, which we denote by
$A_m$. Homology can then be understood in terms of dimension
\begin{equation*}
  \beta_m := \dim_{\Fqn{}} \frac{\ker A_{m-1}}{\im A_m} \, ,
  \label{eqn:betti_numbers}
\end{equation*}
where $\beta_m$ is known as the $m^\mathrm{th}$ {\em Betti number}.

% \mc{expand this}
%Formally, a {\em chain complex $(C_n, \delta_n)$ } is a sequence of modules,
%denoted $C_n$, and a sequence of linear transformations $\delta_n : C_n \to
%C_{n-1}$ that satisfy the `boundary' condition $\delta_{n-1}\delta_n = 0$ for
%all $n$. The $\delta_n$ are usually called the boundary maps of the chain
%complex. An interested reader can see \cite{hatcher2002algebraic} for further
%details.
%
%Let $q$ be a prime number and let $\Fq$ denote the field with $q$ elements.
%Consider the sequence of vector spaces $\Fqn{n_m}$ indexed by $m$ in the
%integers.  Let $A_m$ be a random sequence of $(n_{m-1})\times (n_m)$ matrices
%whose entries are chosen i.i.d. uniformly from $\Fq$, subject to the condition
%that the product of consecutive matrices is zero.  We then consider the random
%chain complex $(\Fqn{n_m},A_m)$. That is, we consider
%\[
%  \cdots \stackrel{A_{m+1}}{\lra} \Fqn{n_m} \stackrel{A_m}{\lra} \Fqn{n_{m-1}} 
%  \stackrel{A_{m-1}}{\lra} \cdots 
%\]
%where $A_{i+1}A_i=0$ for every $i$. 

% \mc{Do we need an iterative construction like this? I don't think so:
% people work with unbounded chain complexes all the time. Equivalently,
% building a $\Z$-graded chain complex doesn't rely on an induction or anything.
% E.g.: take $C_n = R$ for all $n \in \Z$ and take all the maps to be zero. 
% This is a perfectly good chain complex that doesn't \emph{start} anywhere.
% This would allow us to define a more general object, even if we never work
% with it.}

\subsection*{Main Results}
Let $q$ be a prime number.  We
build a random chain complex with coefficients in $\Fqn{}$ as follows (see
Definition~\ref{defn:random_chain_cx} for a precise statement). Given a
sequence of non-negative integers $\{n_m\}$, where $m \in \N$, together with a probability
distribution $\varphi$ on $\Fqn{}$, we construct random
linear transformations 
\[
  A_m : \Fqn{n_m} \lra \Fqn{n_{m-1}} \, ,
\]
for all $m$. The transformations are subject to the constraint $A_{m-1} A_m =
0$, and should be chosen according to $\varphi$. The latter means the
following: After fixing the standard basis for $\Fqn{n_m}$, it suffices to
construct random $n_{m-1} \times n_m$ matrices $A_m$, satisfying $A_{m-1}A_m =
0$. We do so by choosing matrix entries i.i.d. from the distribution $\varphi$
on $\Fqn{}$. We then say that $(\Fqn{n_m}, A_m, \varphi)$ is a {\em random
chain complex}. 

We are primarily interested in the case when $\varphi$ is the discrete uniform
distribution on $\Fqn{}$. In this case, we drop $\varphi$ from the notation and
say that $(\Fqn{n_m}, A_m)$ is a {\em uniform random chain complex}. We also
restrict our attention to bounded below chain complexes (see Remark~\ref{rem:bdd}). 


Our first result is an explicit formula for the distribution
of the Betti numbers.
\begin{bigthm} 
  \label{thm:bettinum}
  Let $\beta_m$ be the $m$-th Betti number of a uniform random chain complex
  $(\Fq^{n_m} , A_m)$. Then
  \[    
    \bP[\beta_m = b] = \sum_{i_m=0}^{n_{m}} P_{i_m}^{m}(i_m-b)
    \sum_{i_{m-1}=0}^{n_{m-1}} P_{i_{m-1}}^{m-1}\left(n_{m} -i_m\right) \cdots
    \sum_{i_1 = 0}^{n_1} P_{i_1}^1\left(n_2 - i_2\right) P_{n_0}^0 \left(n_1 - i_1\right) \, ,
  \]
  where $P^m_k(r)$ is given in Eq.~\eqref{eqn:Pmkr}.
\end{bigthm}

As Theorem~\ref{thm:bettinum} gives a formula for
computing the distribution of the Betti numbers, it also leads to formulas for
other probabilistic properties of $\beta_m$, such as its moments and variance.

Our second main result show that, asymptotically, the $m$-th Betti number of a uniform random chain complex concentrates in a single value. Set
\begin{equation*}
  (n)_+ = \max(0,n) \, , % \mbox{ and } \, \, (n)_- = \min(0,n) \
\end{equation*}
to be the {\em positive part} of $n$.
Define
\begin{equation}
  B_m = (-n_{m+1} + (n_m - (n_{m-1} - (\cdots - (n_1 - n_0)_+ \cdots)_+ )_+
  )_+)_+ \, \, .
  \label{eqn:Nm}
\end{equation}

\begin{bigthm}
  \label{thm:qtoinfty}
  For a uniform random chain complex $(\Fqn{n_m},A_m)$ with $B_m$ defined as in Eq.~\eqref{eqn:Nm},
  \[
    \bP[\beta_m = B_m] \ra 1 
    \mbox{ as } q \ra \infty  \, .
  \]
\end{bigthm}

% \mc{The following remarks need to be re-written and substantiated.}
%\begin{remark}
%  \label{rem:smallest}
%  The number $N_m$ of Theorem~\ref{thm:qtoinfty} represents the smallest
%  possible value of $\beta_m$. Therefore, we interpret Theorem~\ref{thm:qtoinfty}
%  as the statement that the rank of the homology is minimal as 
%  $q \ra \infty$.
%\end{remark}

\begin{remark}
  \label{rem:monotone}
  As a special case of the above theorem, consider when $\{n_m\}$
  is constant or increasing. In this case, 
  $B_m = 0$, and the homology is trivial in probability as $q \ra \infty$ 
  (see Corollary~\ref{cor:inc}). 
\end{remark}

\subsection*{Related Work} Others have considered different methods of applying
randomness to chain complexes. In~\cite{ginzburg2017random}, Ginzburg and
Pasechnik investigate a different notion of a random chain complex than the one
we have described above.  Given a finite dimensional vector space $V$ over
$\Fqn{}$, they consider chain complexes of the form \[ \cdots
\stackrel{D}{\lra} V \stackrel{D}{\lra} V \stackrel{D}{\lra} \cdots \, \, , \]
for a randomly chosen linear operator $D$ such that $D^2 = 0$. They choose the
operator $D$ uniformly over all such possible choices. In particular, our
construction is distinct from theirs, since they use the same operator $D$ at
each stage of the complex.
The first of their main results~\cite[Thm 2.1]{ginzburg2017random} states that
the rank of homology concentrates in the lowest possible dimension as $q \ra \infty$.  
In the special case when $n_m \equiv n$ is constant, we also obtain minimal rank
homology (see Remark~\ref{rem:monotone}).

% Their second main result~\cite[Thm 2.2]{ginzburg2017random} shows that 
% the following limit exists and is bounded
% \[
%   0 < \lim_{n \ra \infty} \bP[\beta_r = j] < 1 \, ,
% \]
% for all $r$ and $j$. Based on explicit bounds, they provide analytic
% formulas for the quantity $\lim_{n\ra \infty}\bP[\beta_j=r]$ in terms
% of $r$ and $j$.\mz{Has something happened to the index here?}

The second author has introduced and studied the
properties of a random Bockstein operation~\cite{zabka2018random}. 
% The Bockstein is an example of a
% cohomology operation in algebraic topology, which can detect information about the topology of a
% space that neither homology nor cohomology can detect. 
In homological algebra,
the Bockstein is a connecting homomorphism associated with a short exact
sequence of abelian groups, which are then used as the coefficients in a chain
complex. Given a random boundary operator of a chain complex, the distribution of compatible
random Bockstein operations is given in~\cite[Thm 5.2]{zabka2018random}.

\subsection*{Outline}
  The paper is organized as follows. In Section 2, we discuss preliminary results useful for 
  the combinatorial aspects of our results. We give a precise definition of a model for a 
  random chain complex in Section 3, as well as prove
  Theorem~\ref{thm:bettinum}. In Section 4, we complete the proof of Theorem~\ref{thm:qtoinfty}.

% \subsection*{Conventions}


  \subsection*{Acknowledgments} The first author would like to thank Peter Bubenik for helpful discussions.

%---------------------------------------------------
\section{Preliminaries}
This section consists of several lemmas that count the number of elements in various sets related to finite vector spaces over $\mathbb{F}_q$.  We provide proofs for these lemmas, but an interested reader can see \cite{stanley2011enumerative} for further details. 
\begin{lemma}\label{NumkTup}
  The number of ordered, linearly independent $k$-tuples of vectors in $\Fqn{n}$  is
\[
(q^n-1)(q^n-q)(q^n-q^2)\cdots(q^n - q^{k-2})(q^n-q^{k-1}).
\]
\end{lemma}

\begin{proof}
Since first vector in the $k$-tuple may be any vector except for the zero vector, there are $q^n-1$ choices for the first vector.  For $1<m \leq k$, the $m$-th vector in the $k$-tuple may be any vector that is not a linear combination of the previously chosen $m-1$ vectors. So there are $q^n - q^{m-1}$ choices for the $m$-th vector.
\end{proof}

The number $\qbin{n}{k}$ defined in the next theorem is known as the $q$-binomial coefficient.

\begin{lemma}\label{NumkSub}
  The number of $k$-dimensional subspaces of $\Fqn{n}$ is
\[
\qbin{n}{k} = \frac{(q^n-1)(q^n-q)(q^n-q^2)\cdots(q^n - q^{k-2})(q^n-q^{k-1})}{(q^k-1)(q^k-q)(q^k-q^2)\cdots(q^k - q^{k-2})(q^k-q^{k-1})}.
\]
\end{lemma}


\begin{proof}
  Let $\qbin{n}{k}$ denote the number of $k$-dimensional subspaces of $\Fqn{n}$ and $N(q,k)$ be the number of ordered, linear independent $k$-tuples of vectors in $\Fqn{n}$.  Then Corollary \ref{NumkTup} gives
\begin{equation}\label{Nqk1}
N(q,k) = (q^n-1)(q^n-q)(q^n-q^2)\cdots(q^n - q^{k-2})(q^n-q^{k-1}).
\end{equation}
We may also find $N(q,k)$ another way: We can first choose a $k$-dimensional subspace and then choose the independent vectors in our $k$-tuple from the chosen subspace.  There are $\qbin{n}{k}$ $k$-dimensional subspaces of $\Fqn{n}$.  Then, there are $q^k-1$ choices for the first vector in the $k$-tuple, and, for $1<m \leq k$, there are $q^k - q^{m-1}$ vectors for the $m$-th vector in the $k$-tuple.  Thus
\begin{equation}\label{Nqk2}
N(q,k) = \qbin{n}{k}(q^k-1)(q^k-q)(q^k-q^2)\cdots(q^k - q^{k-2})(q^k-q^{k-1}).
\end{equation}
Equations (\ref{Nqk1}) and (\ref{Nqk2}) give the required result.
\end{proof}

Using Lemmas \ref{NumkSub} and \ref{Nqk1}, we can count the number of matrices with of a given rank. The following lemma will aid us in Section \ref{SecCondComp}.

\begin{lemma}\label{Num_mbyn_rankr}
The number of $m\times n$ matrices with entries in $\Fq$ of rank $r$ is given by
\begin{align*}
	&\qbin{m}{r}(q^n-1)(q^n-q)\cdots (q^n-q^{r-1})\\
   =& \qbin{n}{r}(q^m-1)(q^m-q)\cdots (q^m-q^{r-1})\\
   =& \frac{(q^m-1) (q^m-q) \cdots (q^m-q^{r-1})\cdot
            (q^n-1) (q^n-q) \cdots (q^n-q^{r-1})}
	 		      {(q^r-1)(q^r - q)(q^r - q^2)\cdots (q^r - q^{r-2})(q^r-q^{r-1})}
\end{align*}
\end{lemma}

\begin{proof}
  Let $W$ be a fixed $r$-dimensional subspace of $\Fqn{n}$.  Then the number of matrices whose column space is $W$ is given by the number of $r\times n$ matrices with rank $r$.  This number is given by Lemma \ref{NumkTup}. The number of $r$-dimensional subspaces of $\Fqn{n}$ is $\qbin{m}{r}$, as stated in Lemma \ref{NumkSub}.  The product of these is the number of $m\times n$ rank $r$ matrices.
\end{proof}

We now turn our attention to a sequence of random matrices.\mz{I've moved this stuff to Section 2. We can talk about this}

\begin{definition}\label{defPkmr}
Let $n_m$ be a sequence of natural numbers. Let $B_m$ be a sequence of random $(n_m) \times (n_{m-1})$ matrices whose entries are chosen i.i.d. uniformly from $\Fq$. Let $r$ be a non-negative integer.  Define 
\[
  P^m_k(r) := \mathbb{P} 
  \left[\rank(B_{m+1}) = r|B_{m}B_{m+1} = 0, \nul(B_{m}) = k \right] \, .
\]
\end{definition}


Lemma~\ref{Num_mbyn_rankr} gives us the following.

\begin{lemma}\label{lemPkmr} With $B_m$ defined as in Definition \ref{defPkmr}, we have that
\[
P^m_k(r) = \begin{cases}
  {\displaystyle \frac{\left(\prod_{j=0}^{r-1}\left(q^{n_{m+1}}-q^{j}\right)\right)
  \left(\prod_{j=0}^{r-1}\left(q^k - q^j \right) \right)}
  {q^{kn_{m+1}} \left(\prod_{j=0}^{r-1} \left(q^r-q^j\right)\right)}}
            					& \textrm{ if } k\neq 0, r \leq k \\
           0					& \textrm{ if }  r>k,\\
           1					& \textrm{ if } r = k = 0.
            \end{cases}
\]
\end{lemma}
\begin{proof}
Let $k= \nul(B_{m})$ and suppose $B_mB_{m+1} = 0$.  Then $B_{m+1}$ maps $\Fq^{n_{m+1}}$ into the kernel of $B_m$, and thus $\rank(B_{m+1}) \leq k$.  Therefore, if $r>k$, we have $P^m_k(r) = 0$.  Further, if $k=0$, then $\rank(B_{m+1}) = 0$, so $P^m_0(0) = 1$.

On the other hand, suppose $k\neq 0$ and $r\leq k$.  Then, $B_{m+1}$ represents a linear transformation from $\Fq^{n_{m+1}}$ into a $k$-dimensional subspace of $\Fq^{n_{m}}$. Thus, by changing basis, $B_{m+1}$ can be represented by a $(n_{m+1}) \times k$ matrix.  There are $q^{kn_{m+1}}$ such matrices, and, by Lemma \ref{Num_mbyn_rankr}, there are 
\[
 \displaystyle \frac{\left(\prod_{j=0}^{r-1}\left(q^{n_{m+1}}-q^{j}\right)\right)
  \left(\prod_{j=0}^{r-1}\left(q^k - q^j \right) \right)}
  {\prod_{j=0}^{r-1} \left(q^r-q^j\right)}
\]
such matrices of rank $r$.

\end{proof}
%---------------------------------------------------
\section{The Homology of a Random Chain Complex}\label{SecCondComp}
Recall that a chain complex consists of a pair of sequences $(C_n, \delta_n)$,
where the $C_n$ are appropriate spaces (vector spaces, groups, modules, etc.)
and the $\delta_n$ are maps, $\delta_n: C_n \to C_{n-1}$ such that
$\delta_{n-1}\delta_n = 0$.  

Let $q$ be a prime number and let $\Fq$ denote the field with $q$ elements. Let
$n_m$ be a sequence of natural numbers. Consider the sequence of finite vector
spaces $\Fqn{n_m}$ indexed by $m$.  We iteratively construct a sequence $A_m$
of boundary maps from $\Fq^{n_m}$ into $\Fq^{n_{m-1}}$.

Let $A_0:\Fq^{n_0} \to 0$ be the zero map. Suppose for $m>0$ that
$A_{m}:\Fq^{n_m} \to \Fq^{n_{m-1}}$ is known. Let $A_{m+1}$ be the random map
that is given by the random $(n_{m})\times (n_{m+1})$ matrix whose entries are
chosen i.i.d. uniformly from $\Fq$, subject to the condition that $A_{m+1}A_{m}
= 0$.

\begin{definition} 
  \label{defn:random_chain_cx}
  Let $n_m$ be a sequence of natural numbers. A \textbf{random chain complex}
  is a pair $(\Fqn{n_m},A_m)$, where $A_m$ is an iteratively defined sequence
  of random linear transformations given by random $(n_{m-1})\times (n_m)$
  matrices, as defined above.
\end{definition}

We wish to investigate the probabilistic properties of the homology of a random
chain complex.  We are primarily interested in the distribution of the Betti
numbers $\beta_m := \dim(H_m(A_\ast; \Fqn{n}))$.  Recall that
$H_m(A_\ast;\Fqn{n}) = \ker (A_{m})/ A_{m+1}(\Fqn{n})$, so if $k= \ker(A_m)$,
then $\beta_m = k - \rank(A_{m+1})$.  We are therefore interested in the
probabilistic properties of $\rank(A_{m+1})$ given that $A_{m}A_{m+1} = 0$ and
that $\nul(A_{m}) = k$. 


\begin{remark}
As the $A_m$ of a random chain complex are uniformly distributed, Definition
\ref{defPkmr} immediately gives us that
\[
  P^m_k(r) = \mathbb{P}\left[\beta_m = k-r | \nul(A_{m}) = k\right].
\]
\end{remark}

\mz{If we know anything about $\lim_{q\to\infty} \Fq$, we should state it here.}


We are now ready to state our first theorem.

\mz{Letting $k\leq n_{m+1}$ seems fine to me, but what about Definition \ref{defPkmr}?}

\begin{theorem}\label{Condptoinfty}
Let $\beta_m$ be the $m$-th Betti number of a random chain complex.  If $k\leq n_{m+1}$, then
\[
\mathbb{P}\left[\beta_m=0| \nul(A_{m}) = k \right] \to 1 \textrm{ as } q\to\infty.
\]
\end{theorem}

\begin{proof}
We have
	\begin{eqnarray*}
	&&\mathbb{P}\left[\beta_m = 0|A_{m}A_{m+1} = 0, \nul(A_{m}) = k \right]\\ 
    &=& P^m_k(k)\\
    &=& \frac{\prod_{j=0}^{k-1}(q^{n_{m+1}}-q^{j})
		\prod_{j=0}^{k-1}(q^k - q^j )}
		{q^{kn_{m+1}} \prod_{j=0}^{k-1} (q^k-q^j)}\\
		&=& \frac{\prod_{j=0}^{k-1}(q^{n_{m+1}} - q^j)}
		{q^{kn_{m+1}}} \\
		%&=& \frac{\prod_{j=0}^{k-1}(q^{n_{m+1}} - q^j)}{q^{n_{m+1}k}}\\
		&=& \frac{q^{kn_{m+1}}}{q^{kn_{m+1}}}\prod_{j=0}^{k-1}(1-q^{j-n_{m+1}})\\
		&=& \prod_{j=0}^{k-1} (1-q^{j-n_{m+1}}),
	\end{eqnarray*}
which tends to 1 as $q\to\infty$.  
\end{proof}

The previous theorem immediately leads to two corollaries.  

\begin{corollary}\label{condtozero}
Let $\beta_m$ be the $m$-th Betti number of a random chain complex. Let $b$ be a positive integer that is less than or equal to $k$. Then 
\[
\mathbb{P}[\beta_m = b| \nul(A_{m}) = k ] \to 0 \textrm{ as } q\to\infty.
\]
\end{corollary}
\begin{proof}
We have that
	\begin{eqnarray*}
	&&\mathbb{P}[\beta_m = b \st A_{m}A_{m+1} = 0, \nul(A_{m}) = k ]\\
    &=& 1 - \sum_{j\neq b}\mathbb{P}[\beta_0 = j \st A_{m}A_{m+1} = 0, \nul(A_{m}) = k ]\\
    &\leq& 1 - \mathbb{P}(\beta_m = 0|A_{m}A_{m+1} = 0, \dim\ker(A_{m}) = k ).
	\end{eqnarray*}
By Theorem~\ref{Condptoinfty}, this goes to $0$ as $q$ goes to infinity.
\end{proof}

\begin{corollary}
Let $\beta_m$ be the $m$-th Betti number of a random chain complex.  Then 
\[
\mathbb{E}[\beta_m | \nul(A_{m}) = k ] \to n \textrm{ as } q\to\infty.
\]
\begin{proof}
The conditional expectation of the $m$-th Betti number is given by
	\begin{eqnarray*}
	& & \mathbb{E}[\beta_m | A_{m}A_{m+1} = 0, \dim\ker(A_{m}) = k ]\\
	&=& \sum_{b=1}^n b \mathbb{P}(\beta_m = b | A_{m}A_{m+1} = 0, \dim\ker(A_{m}) = k ) \, .
	\end{eqnarray*}
By Corollary \ref{condtozero}, all terms with $b< n$ in this sum tend to 
$0$ as $q$ goes to infinity. 

On the other hand, when $b=n$, 
\[
n\mathbb{P}(\beta_m=0| A_{m}A_{m+1} = 0, \dim\ker(A_{m}) = k ),
\]
which goes to $n$ has $q$ goes to infinity by Theorem~\ref{Condptoinfty}.
\end{proof}
\end{corollary}

We next derive explicit formulas for the distributions of the Betti numbers of a random chain complex. We first need a lemma

\begin{lemma}\label{lemProbranks}
Let $(\Fq^{n_m}, A_m)$ be a random chain complex with $A_0:\Fq^{n_0}\to 0$.  Then
\begin{align*}
 &  \mathbb{P}\left[\rank(A_{m+1}) = n_{m+1} - k\right]\\
=& 	\sum_{i_m=0}^{n_m} P_{i_m}^{m}\left(n_{m+1} -k\right)
	\sum_{i_{m-1}=0}^{n_{m-1}} P_{i_{m-1}}^{m-1}\left(n_m - i_m\right)
		\cdots
	\sum_{i_1 = 0}^{n_1} P_{i_1}^1\left(n_2 - i_2\right) P_{n_0}^0 \left(n_1 - i_1\right) 
\end{align*}
\end{lemma}
\begin{proof}
The proof is by induction on $m$. For the basis step $m=1$, we have
\begin{align*}
\mathbb{P}\left[\rank(A_1) = n_1 - k\right]
	=& \sum_{i_0=0}^{n_0}\mathbb{P}\left [\rank(A_1)= n_1 - k|\nul (A_0) = i_0\right]
		\mathbb{P}\left[\nul(A_0) = i_0 \right]\\
	=& \mathbb{P}\left [\rank(A_1)= n_1 - k|\nul (A_0) = n_0\right]\\
	=& P_{n_0}^0(n_1 - k).
\end{align*}
The first equality follows by the Law of Total Probability, and the second equality follows because $A_0$ is the zero map.

For the inductive step, suppose that
\begin{align*}
 &  \mathbb{P}\left[\rank(A_{m}) = n_{m} - i_{m}\right]\\
=& 	\sum_{i_{m-1}=0}^{n_{m-1}} P_{i_{m-1}}^{m-1}\left(n_{m} -i_{m}\right)
	\sum_{i_{m-2}=0}^{n_{m-2}} P_{i_{m-2}}^{m-2}\left(n_{m-1} - i_{m-1}\right)
		\cdots
	\sum_{i_1 = 0}^{n_1} P_{i_1}^1\left(n_2 - i_2\right) P_{n_0}^0 \left(n_1 - i_1\right).
\end{align*}
We must show that
\begin{align*}
 &  \mathbb{P}\left[\rank(A_{m+1}) = n_{m+1} - k\right]\\
=& 	\sum_{i_m=0}^{n_m} P_{i_m}^{m}\left(n_{m+1} -k\right)
	\sum_{i_{m-1}=0}^{n_{m-1}} P_{i_{m-1}}^{m-1}\left(n_m - i_m\right)
		\cdots
	\sum_{i_1 = 0}^{n_1} P_{i_1}^1\left(n_2 - i_2\right) P_{n_0}^0 \left(n_1 - i_1\right) 
\end{align*}
Note that
\begin{align*}
 &	\mathbb{P}\left[\rank(A_{m+1}) = n_{m+1} - k\right]\\
=& 	\sum_{i_{m}}^{n_{m}} \mathbb{P}\left[ \rank(A_{m+1}) = n_{m+1} - k  \right|\nul(A_m) = i_m]
		\mathbb{P}\left[\nul(A_m) = i_m\right]\\
=&  \sum_{i_{m}}^{n_{m}} P_{i_m}^{n_m}(n_{m+1} - k)
		\mathbb{P}\left[n_m - \rank(A_m) = i_m\right]\\
=&	\sum_{i_{m}}^{n_{m}} P_{i_m}^{n_m}(n_{m+1} - k)
		\mathbb{P}\left[\rank(A_m) = n_m- i_m\right].\\
\end{align*}
The desired result now follows by the induction hypothesis.
\end{proof}

\begin{theorem} Let $\beta_j$ be the $j$-th Betti number of the random chain complex $(\Fq^{n} , A_m)$. Then
	\[    
    \bP[\beta_j = b] = \sum_{k=0}^{n_{m}} P_{k}^{m}(k-b)\sum_{i_m=0}^{n_m} P_{i_m}^{m}\left(n_{m+1} -k\right)
		\cdots
	\sum_{i_1 = 0}^{n_1} P_{i_1}^1\left(n_2 - i_2\right) P_{n_0}^0 \left(n_1 - i_1\right).
    \]
\end{theorem}
\begin{proof}
We have

\begin{align*}
  \mathbb{P}[\beta_{m} = b] 
   &= \mathbb{P}[\nul(A_{m}) - \rank(A_{m+12}) = b]\\
   &= \mathbb{P}[\rank(A_{m+1}) = \nul(A_{m}) - b]\\
   &= \sum_{k=0}^{n_{m}}\mathbb{P}[\rank(A_{m+1}) = k - b| \nul(A_{m})=k] \mathbb{P}[\nul(A_{m}) = k]\\
   &= \sum_{k=0}^{n_{m}} P_{k}^{m}(k-b)\mathbb{P}[n_{m+1} - \rank(A_{m}) = k]\\
   &= \sum_{k=0}^{n_{m}} P_{k}^{m}(k-b)\mathbb{P}[\rank(A_{m+1}) = n_{m} - k]\\ 
\end{align*}
By Lemma \ref{lemProbranks} we have
\begin{align*}
 & \sum_{k=0}^{n_{m}} P_{k}^{m}(k-b)\mathbb{P}[\rank(A_{m+1}) = n_{m} - k] \\
=& \sum_{k=0}^{n_{m}} P_{k}^{m}(k-b)\sum_{i_m=0}^{n_m} P_{i_m}^{m}\left(n_{m+1} -k\right)
		\cdots
	\sum_{i_1 = 0}^{n_1} P_{i_1}^1\left(n_2 - i_2\right) P_{n_0}^0 \left(n_1 - i_1\right), 
\end{align*}
as desired.
\end{proof}

%-------------------------------------------------
\section{Conjectures}
\mc{I think this section will be completely removed now with no conjectures.
I still point out what has and hasn't been done.}

This section consists of a bunch of miscellaneous facts and conjectures that
either we should include or not, but either way, I think we should understand
them. In particular, I think these will be useful things to know to get a good
grasp on what's really going on. 

\begin{conjecture}
 Consider the following.
  \begin{enumerate}
    \item Fix $r$. As a function of $k$, $P^m_k(r)$ is decreasing on its support.
    \item Fix $k$. As a function of $r$, $P^m_k(r)$ is increasing on its support.
    \item $P^m_k(r)$ has a maximum when $k = r$, as either variable changes.
    \item $P^m_{r+1}(r) > P^m_r(r-1)$.
  \end{enumerate}
\end{conjecture}

\mc{Answer: We haven't answered this although its probably pretty easy to prove.}


\begin{question} 
  What is the max of $\bP[\beta_1=b]$, as a function of $b$? For
  what $b$ is the max attained? The same questions for $\bP[\beta_2=b]$.
\end{question}
  \mc{Answer: I think our stuff shows that $\bP[\beta_1=b]$ attains its maximum
    at $b = N_m$. Of course, this is in the limit as $q \ra \infty$, but maybe
    there's an argument that says for fixed $q$, $\bP[\beta_1=b]$ is a 
    decreasing function of $b$, and therefore it attains its maximum at
    the minimal value of $b$. And there should be nothing special to $\beta_1$
  here, and should hold for $\beta_k$ in general.}

\begin{question}
  What if $n_0 = w$ and $n>0 = 2w$? Intuitively, we would expect these numbers
  to not depend on the degree of the chain complex. Something similar should
  be true for any sequence of $(n_i)$ which converge.
\end{question}
\mc{Answer: If the sequence converges, it must be eventually constant and then $N_m = 0$.}

\begin{question}
  What does this formula say if there exists $N >0$ so that $n_i=0$ for 
  all $i > N$? Specifically, $\bP[\beta_i=0] = 1$ and $\bP[\beta_i > 0]= 0$ 
  for $i>N$.
\end{question}
\mc{Answer: same.}

Using the same logic as the previous theorem, I'm {\em pretty sure} we have
the following.

%\begin{conjecture}
%  \[
%    \bP[\rank A_m = r_m] = \sum_{i_{m-1}=0}^{n_{m-1}} P^{m-1}_{i_{m-1}}(r_m)
%    \sum_{i_{m-2} = 0}^{n_{m-2}} P^{m-2}_{i_{m-2}}(n_{m-1}- i_{m-1})
%    \cdots \sum_{i_1 = 0}^{n_2}P^1_{i_1}(n_2-i_2) P^0_{n_0}(n_1-i_1)
%  \]
%\end{conjecture}


\begin{question} What distribution does the random variable
  $\rank(A_m)$ follow? What are its moments?
\end{question}
\mc{Answer: we could answer this, we have answered the second question for $\beta_m$.}

\begin{conjecture}
  Attempt to answer all of the following in cases:
  (i) $n_i$ is eventually constant, (ii) $n_i \sim o(i)$.
  \begin{enumerate}
    \item $\bP[\beta_m = 0] \ra ?$ as $m \ra \infty$.
    \item $\bP[\beta_m = j] \ra ?$ as $m \ra \infty$, $j>0$.
    \item $\bE[\beta_m] = ?$ as $m \ra \infty$.
  \end{enumerate}
\end{conjecture}
\mc{Answer: done.}



\bibliography{master}
\bibliographystyle{plain}
  

\end{document}
