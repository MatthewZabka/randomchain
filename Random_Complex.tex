\mc{This section will be removed. Before we remove it, let's be 
certain we can't do anything with the final formula.}
In the previous section, we built a chain complex iteratively. That is, we
studied the properties of attaching an appropriate map to an already
constructed complex of length $m-1$. \mz{Rewrite this}

In this section, we assume we have already have a chain complex.  That is, we want to know the probabilistic properties of $H_m(A_\ast)$ knowing only that $A_{m-1}A_m=0$. \mz{Rewrite this too} That is, we wish to investigate
\[
\mathbb{P}[\rank(A_m) = r| A_{m}A_{m+1} = 0].
\] 

As in Section \ref{SecCondComp}, for $k\neq 0$ set
\[
P_k(r) :=  \mathbb{P}[\rank(A_{m+1})=r|A_{m}A_{m+1} = 0, \dim\ker (A_{m}) = k].
\]
and set $P_k(r) = 0$ whenever $r<k$.
Note that this does not depend on $m$. Then
	\begin{eqnarray*}
	&&\mathbb{P}[\rank(A_{m+1}) = r| A_{m}A_{m+1} = 0]\\
	&=& \sum_{k=0}^n \mathbb{P}[\dim\ker (A_{m}) = k|A_{m}A_{m+1} = 0]
    	\cdot P_k(r)  \textrm{ (law of tot. prob) }\\
	&=& \sum_{k=0}^n \mathbb{P}[\dim\ker (A_{m}) = k| \Fqn{n} \xrightarrow{A_{m+1}} \ker(A_{m})]
    	\cdot P_k(r)\\
	&=& \sum_{k=0}^n \mathbb{P}[\rank(A_{m}) = n-k | \Fqn{n} \xrightarrow{A_{m+1}} \ker(A_{m})]
    	\cdot P_k(r)\\
    &=& \sum_{k=0}^n \frac{\mathbb{P}[\rank(A_{m}) = n-k]
    \cdot \mathbb{P}(\Fqn{n} \xrightarrow{A_{m+1}} \ker(A_{m}) |\dim\ker(A_{m}) = k)}
    {\mathbb{P}(\Fqn{n} \xrightarrow{A_{m+1}} \ker(A_{m}))}
        \cdot P_k(r).
   \end{eqnarray*}
The last equality holds by Bayes' Rule. \mz{The magic should happen in the denominator of this fraction. It can be pulled out in front of the sum.  Everything in the numerator can be figured out using the lemmas in the counting section -- in fact, it seems to be constant over $m$.  See below:}

\[
  \frac{ \sum_{k=0}^n \mathbb{P}(\rank(A_{m}) = n-k) \cdot \mathbb{P}(\Fqn{n} \xrightarrow{A_{m+1}} \ker(A_{m}) |\dim\ker(A_{m}) = k)
    \cdot P_k(r)}
    {\mathbb{P}(\Fqn{n} \xrightarrow{A_{m+1}} \ker(A_{m}))}
\]

%\section{(Mike's probably wrong) Homology calculation}
%Let us step through the computation of the homology of a random
%chain complex degree by degree. In what follows, we assume a random
%chain complex $(\Fqn{n},A_*)$ exists.
%
%We ask for $\mathbb{P} [\beta_j= b_j]$, for each $j$. I'm kind of
%sloppy in sticking with either $\nul$ or $\rank$ in the following
%so I'm sure I use rank + nullity a lot without explicitly saying it.
%You've been warned...
%
%\subsection{$\beta_0$}
%
%\begin{align*}
%  \mathbb{P}[\beta_0 = b_0] &= \mathbb{P}[\rank(A_1) = n - b_0] \\
%   &= \frac{1}{q^{n^2}} \qbin{n}{n-b_0} \, (q^n-1) (q^n - q) \cdots (q^n - q^{n-b_0-1}) \, ,
%\end{align*}
%by Lemma~\ref{Num_mbyn_rankr}.
%
%\mc{Alternatively, we have $\bP[\beta_0=b_0] = 
%\bP[\beta_0=b_0 \, | \, \nul(A_0) = n]$, vacuously since
%$A_0$ is the zero map. But then
%$\bP[\beta_0 = b_0] = \bP_{n}(n-b_0)$, which may be obvious.
%Since it's late, I will only make this substitution in the 
%last conjecture.}
%
%  
%\subsection{$\beta_1$}  
%  
%\begin{align*}
%  	\mathbb{P}[\beta_1 = b_1] 
%  	&= \sum_{b=0}^n \mathbb{P} [ \beta_1 = b_1 \, | \, \beta_0=b] \cdot \mathbb{P}[\beta_0 = b] \\
%  	&= \sum_{b=0}^n \mathbb{P}[ \rank(A_2) = b-b_1 \, | \, \rank(A_1)=n-b] \cdot \mathbb{P}[ \rank(A_1) = n-b] \\
%  	&= \sum_{b=0}^n \mathbb{P}[ \rank(A_2) = b-b_1 \, | \, \nul(A_1)=b]  \cdot \mathbb{P}[ \rank(A_1) = n-b] \\
%  	&= \sum_{b=0}^n P_b(b-b_1) \bP[\beta_0=b] \\
%    &= \sum_{b=0}^n P_b(b-b_1) P_n(n-b)\, ,
%\end{align*}  
%where the last factor of the last equal sign is computed in the previous
%subsection.
%
%\subsection{$\beta_2$}
%
%\begin{align*}
%  	\bP[\beta_2 = b_2] 
%  	&= \sum_{c=0}^n \bP[\beta_2=b_2 \, | \, \nul(A_2) = c] \bP[\nul(A_2) = c] \\
%  	&= \sum_{c=0}^n \bP[\rank(A_3) = c-b_2 \, | \, \nul(A_2) = c] \bP[\nul(A_2) = c] \\
%  	&= \sum_{c=0}^n P_c(c-b_2) \bP[\rank(A_2) = n-c]
%\end{align*}
%
%Now we must compute this last factor $\bP[\rank(A_2) = n-c]$, so we 
%expand again:
%
%\begin{align*}
%  	\bP[\rank(A_2) = n-c] 
%    &= \sum_{b=0}^n \bP[\rank(A_2) = n-c \, | \,\nul(A_1) = b] \, \bP[\nul(A_1) = b] \\
%  	&=\sum_{b=0}^n P_b(n-c) \bP[\beta_0=b]
%\end{align*}
%
%Collecting these two, we get
%\[
%  \bP[\beta_2 = b_2] = \sum_{c=0}^n P_c(c-b_2) \sum_{b=0}^n P_b(n-c)
%  \bP[\beta_0=b]
%\]
%
%\subsection{$\beta_3$}
%So I think we understand the general story now. But just to be sure,
%let's do one more.
%
%\begin{align*}
%  	\bP[\beta_3 = b_3] 
%    &= \sum_{d=0}^n \bP[\beta_3=b_3 \, | \, \nul(A_3) = d] \, \bP[\nul(A_3) = d] \\
%  	&= \sum_{d=0}^n P_d(d-b_3) \bP[\rank(A_3) = n-d] \\
%  	&= \sum_{d=0}^n P_d(d-b_3) \, \sum_{c=0}^n P_c(n-d) \, \bP[\rank(A_2) = n-c] \\
%  	&= \sum_{d=0}^n P_d(d-b_3) \, \sum_{c=0}^n P_c(n-d) \, \sum_{b=0}^n P_b(n-c) \bP[\beta_0=b] \\
%\end{align*}
%
%\subsection{$\beta_j$}
%
%So here's the conjecture:
%
%\begin{align*}
%	\bP[\beta_j = b] 
%    &= \sum_{i_1=0}^n P_{i_1}(i_1-b) \sum_{i_2=0}^n P_{i_2}(n-i_1) \cdots \sum_{i_j=0}^n P_{i_j}(n-i_{j-1}) \bP[\beta_0=i_j] \\
% 	&= \sum_{i_1=0}^n P_{i_1}(i_1-b) \sum_{i_2=0}^n P_{i_2}(n-i_1) \cdots \sum_{i_j=0}^n P_{i_j}(n-i_{j-1}) P_n(n-i_j) \\
%\end{align*}
% 
%\mc{This formula may not be useful in any practical way. But it definitely
%`completes' the story in some sense. We should still do expectations
%and conditional probabilites.}
