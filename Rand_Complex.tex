Recall that a chain complex consists of a pair of sequences $(C_n, \delta_n)$, where the $C_n$ are appropriate spaces (vector spaces, groups, modules, etc.) and the $\delta_n$ are maps, $\delta_n: C_n \to C_{n-1}$ such that $\delta_{n-1}\delta_n = 0$.  

Let $q$ be a prime number and let $\Fq$ denote the field with $q$ elements. Let $n_m$ be a sequence of natural numbers. Consider the sequence of finite vector spaces $\Fqn{n_m}$ indexed by $m$.  We iteratively construct a sequence $A_m$ of random $(n_{m-1})\times (n_m)$ matrices.

Suppose we know $A_{m}$. Let $A_{m+1}$ be a random $(n_{m})\times (n_{m+1})$ matrix whose entries are chosen i.i.d. uniformly from $\Fq$, subject to the condition that $A_{m+1}A_{m} = 0$.

\begin{definition}
Let $n_m$ be a sequence of natural numbers. A \textbf{random chain complex} is a pair $(\Fqn{n_m},A_m)$, where $A_m$ is an iteratively defined sequence of random of $(n_{m-1})\times (n_m)$ matrices, as defined above.  
\end{definition}

We wish to investigate the probabilistic properties of the homology of a random chain complex.  We are primarily interested in the distribution of the Betti numbers $\beta_m := \dim(H_m(A_\ast; \Fqn{n}))$.  Recall that $H_m(A_\ast;\Fqn{n}) = \ker (A_{m})/ A_{m+1}(\Fqn{n})$, so if $k= \ker(A_m)$, then $\beta_m = k - \rank(A_{m+1})$.  We are therefore interested in the probabilistic properties of $\rank(A_{m+1})$ given that $A_{m}A_{m+1} = 0$ and that $\nul(A_{m}) = k$. 

\begin{remark}
As the $A_m$ are uniformly distributed, Definition \ref{defPkmr} immediately gives us that
\[
P^m_k(r) = \mathbb{P}\left[\beta_m = k-r | \nul(A_{m}) = k\right] \, .
\]
\end{remark}

%\mc{The following theorem is no longer true in general. It fails if $n_{m+1} < k$, but it is still true if $n_m = n_{m+1}$. One possible fix is to just impose $k \leq n_{m+1}$. This proposed fix may be   unnatural. In any case, there should be some condition on $k$ in the statement, since its a bound variable.}
\mz{Letting $k\leq n_{m+1}$ seems fine to me, but what about Definition \ref{defPkmr}?}


\begin{theorem}\label{Condptoinfty}
Let $\beta_m$ be the $m$-th Betti number of a random chain complex.  If $k\leq n_{m+1}$, then
\[
\mathbb{P}\left[\beta_m=0| A_{m}A_{m+1} = 0, \nul(A_{m}) = k \right] \to 1 \textrm{ as } q\to\infty.
\]
\end{theorem}

\begin{proof}
We have
	\begin{eqnarray*}
	&&\mathbb{P}\left[\beta_m = 0|A_{m}A_{m+1} = 0, \nul(A_{m}) = k \right]\\ 
    &=& P^m_k(k)\\
    &=& \frac{\prod_{j=0}^{k-1}(q^{n_{m+1}}-q^{j})
		\prod_{j=0}^{k-1}(q^k - q^j )}
		{q^{kn_{m+1}} \prod_{j=0}^{k-1} (q^k-q^j)}\\
		&=& \frac{\prod_{j=0}^{k-1}(q^{n_{m+1}} - q^j)}
		{q^{kn_{m+1}}} \\
		%&=& \frac{\prod_{j=0}^{k-1}(q^{n_{m+1}} - q^j)}{q^{n_{m+1}k}}\\
		%&=& \frac{q^{kn_{m+1}}}{q^{kn_{m+1}}}\prod_{j=0}^{k-1}(1-q^{j-n_{m+1}})\\
		&=& \prod_{j=0}^{k-1} (1-q^{j-n_{m+1}}),
	\end{eqnarray*}
which tends to 1 as $q\to\infty$.  
\end{proof}

The previous theorem immediately leads to two corollaries.  

\begin{corollary}\label{condtozero}
Let $b$ be a positive integer that is less than or equal to $k$. Then 
\[
\mathbb{P}[\beta_m = b| A_{m}A_{m+1} = 0, \nul(A_{m}) = k ] \to 0 \textrm{ as } q\to\infty.
\]
\end{corollary}
\begin{proof}
We have that
	\begin{eqnarray*}
	&&\mathbb{P}[\beta_m = b \st A_{m}A_{m+1} = 0, \nul(A_{m}) = k ]\\
    &=& 1 - \sum_{j\neq b}\mathbb{P}[\beta_0 = j \st A_{m}A_{m+1} = 0, \nul(A_{m}) = k ]\\
    &\leq& 1 - \mathbb{P}(\beta_m = 0|A_{m}A_{m+1} = 0, \dim\ker(A_{m}) = k ).
	\end{eqnarray*}
By Theorem~\ref{Condptoinfty}, this goes to $0$ as $q$ goes to infinity.
\end{proof}

\begin{corollary}
$\mathbb{E}[\beta_m | A_{m}A_{m+1} = 0, \dim\ker(A_{m}) = k ] \to n$ as $q\to\infty$.
\begin{proof}
The conditional expectation of the $m$-th Betti number is given by
	\begin{eqnarray*}
	& & \mathbb{E}[\beta_m | A_{m}A_{m+1} = 0, \dim\ker(A_{m}) = k ]\\
	&=& \sum_{b=1}^n b \mathbb{P}(\beta_m = b | A_{m}A_{m+1} = 0, \dim\ker(A_{m}) = k ) \, .
	\end{eqnarray*}
By Corollary \ref{condtozero}, all terms with $b< n$ in this sum tend to 
$0$ as $q$ goes to infinity. 

On the other hand, when $b=n$, 
\[
n\mathbb{P}(\beta_m=0| A_{m}A_{m+1} = 0, \dim\ker(A_{m}) = k ),
\]
which goes to $n$ has $q$ goes to infinity by Theorem~\ref{Condptoinfty}.
\end{proof}
\end{corollary}
