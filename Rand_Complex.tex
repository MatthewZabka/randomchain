% Recall that a chain complex consists of a pair of sequences $(C_n, \delta_n)$,
% where the $C_n$ are appropriate spaces (vector spaces, groups, modules, etc.)
% and the $\delta_n$ are maps, $\delta_n: C_n \to C_{n-1}$ such that
% $\delta_{n-1}\delta_n = 0$.  

% Let $q$ be a prime number and let $\Fq$ denote the field with $q$ elements. Let
% $n_m$ be a sequence of natural numbers. Consider the sequence of finite vector
% spaces $\Fqn{n_m}$ indexed by $m$.  We iteratively construct a sequence $A_m$
% of boundary maps from $\Fq^{n_m}$ into $\Fq^{n_{m-1}}$.

% \begin{definition} \label{defn:random_chain_cx} Let $n_m$ be a sequence of
%   natural numbers and let $A_0:\Fq^{n_0} \to 0$ be the zero map.  Suppose for
%   $m>0$ that $A_{m}:\Fq^{n_m} \to \Fq^{n_{m-1}}$ is known. Let $A_{m+1}$ be the
%   random linear transformation given by the random $(n_{m})\times (n_{m+1})$
%   matrix whose entries are chosen i.i.d. uniformly from $\Fq$, subject to the
%   condition that $A_{m+1}A_{m} = 0$.  A {\bf random chain complex} is a sequence
%   of pairs $(\Fqn{n_m},A_m)$ for $m \in \Z$.  
% \end{definition}

\mc{What do you think of the following more general definition? In particular,
what do you think of the wording?}

\begin{definition}
  Let $n_m$ be a sequence of natural numbers indexed by $m \in \Z$.  A {\bf
  model for a random chain complex over} $\Fqn{}$ consists of a probability
  distribution $\varphi$ on $\Fqn{}$, together with a sequence of random matrices
  $A_m: \Fqn{n_{m+1}} \ra \Fqn{n_m}$ whose entries are chosen with respect to
  $\varphi$, subject to $A_{m+1}A_m = 0$ for all $m \in \Z$.
\end{definition}


\begin{definition}
  The {\bf uniform random chain complex} is a model for a random chain complex
  over $\Fqn{}$ with the uniform distribution on $\Fqn{}$. 
  \label{defn:random_chain_cx}
\end{definition}

\mc{Re-write all this expository stuff}

% \begin{definition}
%   Let $n_m$ be a sequence of natural numbers indexed by $m \in \Z$. 
%   A {\bf random chain complex} is a sequence of pairs $(\Fqn{n_m},A_m)$
%   such that each $A_m: \Fqn{n_{m+1}} \ra \Fqn{n_m}$ is a 
%   random matrix whose entries are chosen i.i.d. uniformly from $\Fqn{}$,
%   subject to the condition $A_{m+1}A_m = 0$, for all $m \in \Z$.
% \end{definition}

% \mc{We could even get crazier, and we might want to for our CW complexes stuff.
%   Define a {\em model for a random chain complex} to be the above together with
%   a probability distribution on $\Fqn{}$. Then the above is the model with the
%   uniform distribution. This might be more interesting topologically since we
%   might want to do things like Bernoulli with a small parameter, so that our
%   spaces aren't uniformly complicated. Maybe we want to build spaces whose
%   attaching maps tends to have smaller degree, rather than uniform on $\Fqn{}$.
% Importantly, we wouldn't use anything other than the uniform distribution here,
% but still, we would be the first to make the definition.}

We wish to investigate the probabilistic properties of the homology of a random
chain complex.  We are primarily interested in the distribution of the Betti
numbers $\beta_m = \ker A_m - \im A_{m+1}$.
%:= \dim(H_m(A_\ast; \Fqn{n}))$.  Recall that
%$H_m(A_\ast;\Fqn{n}) = \ker (A_{m})/ A_{m+1}(\Fqn{n})$, so if $k= \ker(A_m)$,
%then $\beta_m = k - \rank(A_{m+1})$.  We are therefore interested in the
%probabilistic properties of $\rank(A_{m+1})$ given that $A_{m}A_{m+1} = 0$ and
%that $\nul(A_{m}) = k$. 


\begin{remark}
As the $A_m$ of a random chain complex are uniformly distributed, Definition
\ref{defPkmr} immediately gives us that
\[
  P^m_k(r) = \mathbb{P}\left[\beta_m = k-r \st \nul(A_{m}) = k\right].
\]
\end{remark}

\mz{If we know anything about $\lim_{q\to\infty} \Fq$, we should state it here.}


We are now ready to state our first theorem.


\begin{lemma}
  \label{lem:condqtoinfty}
Let $\beta_m$ be the $m$-th Betti number of a random chain complex.  If $k\leq n_{m+1}$, then
\[
\mathbb{P}\left[\beta_m= 0 \st \nul(A_{m}) = k \right] \to 1 \textrm{ as } q\to\infty.
\]
\end{lemma}
\mz{I had to fix a bunch of the indexing here.  Please double check!}
\begin{proof}
We have
	\begin{align*}
	\mathbb{P}\left[\beta_m = 0 \st A_{m}A_{m+1} = 0, \nul(A_{m}) = k \right]
    =& P^m_k(k)\\
    =& q^{-kn_{m+1}}\prod_{j=0}^{k-1} \frac{(q^{n_m+1}-q^j) (q^k - q^j)}{q^k - q^j}  \\
		=& \frac{1}{q^{kn_{m+1}}} \prod_{j=0}^{k-1}(q^{n_{m+1}} - q^j) \\
		%=& \frac{\prod_{j=0}^{k-1}(q^{n_{m+1}} - q^j)}{q^{n_{m+1}k}}\\
		=& \frac{q^{kn_{m+1}}}{q^{kn_{m+1}}}\prod_{j=0}^{k-1}(1-q^{j-n_{m+1}})\\
		=& \prod_{j=0}^{k-1} (1-q^{j-n_{m+1}}).
	\end{align*}
Since the index $j$ is less than $k$, and since $k$ is less than $n_{m+1}$, this product tends to $1$ as $q\to\infty$.  
\end{proof}

\mz{Exposition here regarding if $k > n_{m+1}$, then the smallest possible rank of $H_m$ is greater than $0$? Maybe an example?}

\begin{lemma}
  \label{lem:condqtoinfty2}
Let $\beta_m$ be the $m$-th Betti number of a random chain complex.  If $k > n_{m+1}$, then
\[
\mathbb{P}\left[\beta_m= k - n_{m+1} \st \nul(A_{m}) = k \right] \to 1 \textrm{ as } q\to\infty.
\]
\end{lemma}

\begin{proof}
Since $k$ must be less than or equal to $n_m$, we have that $n_{m+1} < k \leq n_m$.  So
	\begin{align*}
	\mathbb{P}\left[\beta_m = k - n_{m+1} \st A_{m}A_{m+1} = 0, \nul(A_{m}) = k \right]
    =& P^m_k(n_{m+1})\\
    =& q^{-kn_{m+1}}\prod_{j=0}^{n_{m+1}-1} \left(\frac{(q^{n_{m+1}} - q^j)(q^k - q^j)}{q^{n_{m+1}} - q^j}  \right)  \\
    =&  \prod_{j=0}^{n_{m+1}-1}(1-q^{j-k}).
	\end{align*}
Since the index $j$ is less than to $n_{m+1}$, which is less than $k$, this product goes to $1$ as $q$ goes to infinity.
\end{proof}

\mc{I've made this a separate lemma for now. Maybe we only need this, and not the previous
two lemmas at all. Or maybe not.}

\begin{lemma}
  Fix $m$ and $k$. Then 
  \[
    \lim_{q \ra \infty} P^m_k(r) = 
      \begin{cases}
        1 & \mbox{ if } r = \min(k,n_{m+1}) \, , \\
        0 & \mbox { else.}
      \end{cases}
    \]
\end{lemma}

\begin{proof}
  Reproduce the two previous proofs, starting from $P^m_k(k)$ and $P^m_k(n_{m+1})$.
\end{proof}

\begin{lemma}
  Fix $m$ and $r$. Then
  \[
    \lim_{q \ra \infty} P^m_k(r) = 
      \begin{cases}
        1 & \mbox{ if } r < \min(n_m,n_{m+1}) \, \mbox{ and } \, k = r \, , \\
        1 & \mbox{ if } r = \min(n_m,n_{m+1}) \, \mbox{ and }  \, k \geq r \, , \\
        0 & \mbox { else.}
      \end{cases}
    \]
\end{lemma}

\begin{proof}
  TBD
\end{proof}


We are now in a position to prove Theorem~\ref{thm:qtoinfty}.

\begin{proof}[Proof of Theorem~\ref{thm:qtoinfty}]
  Consider the law of total probability
  \begin{equation}
    \label{eqn:totalbm}
    \bP[\beta_m = j] = \sum_{k=0}^{n_m} \bP[\beta_m = j \st \nul(A_m) = k]
    \bP[\nul(A_m) = k] \, .
  \end{equation}
   Since $0 \leq \nul(A_m) \leq n_m$, we have that $\bP[\nul(A_m) = k]$ is
   finite for all $k$ and for all primes $q$.  It suffices to show that 
   $\bP[\beta_m =N_m \st \nul(A_m) = k] \ra 1$ as $q \ra \infty$.
   
   By Lemma~\ref{lem:condqtoinfty}, every
   term of Eq.~\eqref{eqn:totalbm} with $k\leq n_{m+1}$ tend to 0 as $q \ra \infty$.
   On the other hand, if $k>n_{m+1}$, there are two cases to consider.
   If $n_m \leq n_{m+1}$, then $n_m \leq n_{m+1} < k$, which is a contradiction.
   Otherwise, $n_m > n_{m+1}$, in which case $N_m = 0$. Hence, we have
   \begin{align*}
     \bP[\beta_m = 0 \st \nul(A_m) = k] &= q^{-k n_{m+1}} \prod_{j=0}^{k-1}q^{n_{m+1}}-q^j \\
     &= \prod_{j=0}^{k-1} 1 - q^{j-n_{m+1}} \, ,
   \end{align*}
   which tends to 1 as $q \ra \infty$.
 \end{proof}




  



The previous theorem immediately leads to two corollaries.  
\mc{Fix the following two Corollaries.}

\begin{corollary}\label{condtozero}
Let $\beta_m$ be the $m$-th Betti number of a random chain complex. Let $b$ be a positive integer that is less than or equal to $k$. Then 
\[
\mathbb{P}[\beta_m = b \st \nul(A_{m}) = k ] \to 0 \textrm{ as } q\to\infty.
\]
\end{corollary}
\begin{proof}
We have that
	\begin{eqnarray*}
	&&\mathbb{P}[\beta_m = b \st A_{m}A_{m+1} = 0, \nul(A_{m}) = k ]\\
    &=& 1 - \sum_{j\neq b}\mathbb{P}[\beta_0 = j \st A_{m}A_{m+1} = 0, \nul(A_{m}) = k ]\\
    &\leq& 1 - \mathbb{P}[\beta_m = 0 \st A_{m}A_{m+1} = 0, \dim\ker(A_{m}) = k ].
	\end{eqnarray*}
By Lemma~\ref{lem:condqtoinfty}, this goes to $0$ as $q$ goes to infinity.
\end{proof}

\begin{corollary}
Let $\beta_m$ be the $m$-th Betti number of a random chain complex.  Then 
\[
\mathbb{E}[\beta_m \st \nul(A_{m}) = k ] \to  N_m\textrm{ as } q\to\infty.
\]
\begin{proof}
The conditional expectation of the $m$-th Betti number is given by
	\begin{eqnarray*}
	& & \mathbb{E}[\beta_m \st A_{m}A_{m+1} = 0, \dim\ker(A_{m}) = k ]\\
	&=& \sum_{b=1}^{n_m} b \mathbb{P}[\beta_m = b \st A_{m}A_{m+1} = 0, \dim\ker(A_{m}) = k ] \, .
	\end{eqnarray*}
By Corollary \ref{condtozero}, all terms with $b< n$ in this sum tend to 
$0$ as $q$ tends to infinity. 

On the other hand, when $b=n$, 
\[
n\mathbb{P}(\beta_m=0\st A_{m}A_{m+1} = 0, \dim\ker(A_{m}) = k ),
\]
which goes to $n$ has $q$ tends to infinity by Lemma~\ref{lem:condqtoinfty}.
\end{proof}
\end{corollary}

We next derive explicit formulas for the distributions of the Betti numbers of a random chain complex. We first need a lemma.

\mz{Mike!  Please check the following lemma and proof}
\begin{lemma}\label{lemProbranks}
Let $(\Fq^{n_m}, A_m)$ be a random chain complex with $A_0:\Fq^{n_0}\to 0$.  Then
\begin{align*}
 &  \mathbb{P}\left[\rank(A_{m}) = n_{m} - k\right]\\
=& 	\sum_{i_{m-1}=0}^{n_{m-1}} P_{i_{m-1}}^{m-1}\left(n_{m} -k\right)
	\sum_{i_{m-2}=0}^{n_{m-2}} P_{i_{m-2}}^{m-2}\left(n_{m-2} - i_{m-2}\right)
		\cdots
	\sum_{i_1 = 0}^{n_1} P_{i_1}^1\left(n_2 - i_2\right) P_{n_0}^0 \left(n_1 - i_1\right) 
\end{align*}
\end{lemma}
\begin{proof}
The proof is by induction on $m$. For the basis step $m=1$, we have
\begin{align*}
\mathbb{P}\left[\rank(A_1) = n_1 - k\right]
	=& \sum_{i_0=0}^{n_0}\mathbb{P}\left [\rank(A_1)= n_1 - k\st\nul (A_0) = i_0\right]
		\mathbb{P}\left[\nul(A_0) = i_0 \right]\\
	=& \mathbb{P}\left [\rank(A_1)= n_1 - k\st\nul (A_0) = n_0\right]\\
	=& P_{n_0}^0(n_1 - k).
\end{align*}
The first equality follows by the Law of Total Probability, and the second equality follows because $A_0$ is the zero map.

For the inductive step, suppose that
\begin{align*}
 &  \mathbb{P}\left[\rank(A_{m-1}) = n_{m-1} - i_{m-1}\right]\\
=& 	\sum_{i_{m-2}=0}^{n_{m-2}} P_{i_{m-2}}^{m-2}\left(n_{m-1} -i_{m-1}\right)
	\sum_{i_{m-3}=0}^{n_{m-3}} P_{i_{m-3}}^{m-3}\left(n_{m-2} - i_{m-2}\right)
		\cdots
	\sum_{i_1 = 0}^{n_1} P_{i_1}^1\left(n_2 - i_2\right) P_{n_0}^0 \left(n_1 - i_1\right).
\end{align*}
We must show that
\begin{align*}
 &  \mathbb{P}\left[\rank(A_{m}) = n_{m} - k\right]\\
=& 	\sum_{i_{m-1}=0}^{n_{m-1}} P_{i_{m-1}}^{m-1}\left(n_{m} -k\right)
	\sum_{i_{m-2}=0}^{n_{m-2}} P_{i_{m-2}}^{m-2}\left(n_{m-1} - i_{m-1}\right)
		\cdots
	\sum_{i_1 = 0}^{n_1} P_{i_1}^1\left(n_2 - i_2\right) P_{n_0}^0 \left(n_1 - i_1\right) 
\end{align*}
Note that
\begin{align*}
 &	\mathbb{P}\left[\rank(A_{m}) = n_{m} - k\right]\\
=& 	\sum_{i_{m-1}}^{n_{m-1}} \mathbb{P}\left[ \rank(A_{m}) = n_{m} - k  
\st \nul(A_{m-1}) = i_{m-1}\right] \mathbb{P}\left[\nul(A_{m-1}) = i_{m-1}\right]\\
=&  \sum_{i_{m-1}}^{n_{m-1}} P_{i_{m-1}}^{n_{m-1}}(n_{m} - k)
		\mathbb{P}\left[n_{m-1} - \rank(A_{m-1}) = i_{m-1}\right]\\
=&	\sum_{i_{m-2}}^{n_{m-2}} P_{i_{m-2}}^{n_{m-2}}(n_{m} - k)
		\mathbb{P}\left[\rank(A_{m-1}) = n_{m-1}- i_{m-1}\right].
\end{align*}
The desired result now follows by the induction hypothesis.
\end{proof}

\mz{For consistency, I think we should change the index on $\beta$ to $m$.}

\begin{theorem} Let $\beta_j$ be the $j$-th Betti number of the random chain complex $(\Fq^{n} , A_m)$. Then
	\[    
    \bP[\beta_j = b] = \sum_{i_j=0}^{n_{j}} P_{i_j}^{j}(i_j-b)
    \sum_{i_{j-1}=0}^{n_{j-1}} P_{i_{j-1}}^{j-1}\left(n_{j} -i_j\right)
		\cdots
	\sum_{i_1 = 0}^{n_1} P_{i_1}^1\left(n_2 - i_2\right) P_{n_0}^0 \left(n_1 - i_1\right).
    \]
\end{theorem}

\begin{proof}
We have

\begin{align*}
  \mathbb{P}[\beta_{m} = b] 
   &= \mathbb{P}[\nul(A_{m}) - \rank(A_{m+1}) = b]\\
   &= \mathbb{P}[\rank(A_{m+1}) = \nul(A_{m}) - b]\\
   &= \sum_{k=0}^{n_{m}}\mathbb{P}[\rank(A_{m+1}) = k - b\st \nul(A_{m})=k] \mathbb{P}[\nul(A_{m}) = k]\\
   &= \sum_{k=0}^{n_{m}} P_{k}^{m}(k-b)\mathbb{P}[n_{m} - \rank(A_{m}) = k]\\
   &= \sum_{k=0}^{n_{m}} P_{k}^{m}(k-b)\mathbb{P}[\rank(A_{m}) = n_{m} - k]\\ 
\end{align*}
By Lemma \ref{lemProbranks} we have
\begin{align*}
 & \sum_{k=0}^{n_{m}} P_{k}^{m}(k-b)\mathbb{P}[\rank(A_{m}) = n_{m} - k] \\
=& \sum_{k=0}^{n_{m}} P_{k}^{m}(k-b)\sum_{i_{m-1}=0}^{n_{m-1}} P_{i_{m-1}}^{m-1}\left(n_{m} -k\right)
		\cdots
	\sum_{i_1 = 0}^{n_1} P_{i_1}^1\left(n_2 - i_2\right) P_{n_0}^0 \left(n_1 - i_1\right), 
\end{align*}
as desired.
\end{proof}

\mz{I believe the following lemma and theorem will prove what we want. We may want to change names the index from $j$ to $m$.}
\begin{lemma}
Let $I_m:= \{0,1,\ldots, n_m\}$.  Let $I^{(j)}:= I_1\times\cdots \times I_j$.  Then for every $j$ in $\mathbb{N}$, there exists exactly one $\mathbf{i}^\ast = (i_1^\ast,\ldots, i_j^\ast)$ in $I^{(j)}$ such that 
\[
P_{i_{j-1}^\ast}^{j-1}(n_j-i_j^\ast)\cdots P_{i_1^\ast}^1(n_2-i_2^\ast)P_{n_0}^0(n_1-i_1^\ast) \to 1
\]
as $q\to\infty$.
\end{lemma}

\begin{proof}
The proof is by induction on $j$.

Basis step: $j=1$.  Note that $P_{i_1}^1(n_2 - i_2)P_{n_0}^0 (n_1 - i_1) \to 1$ as $q\to\infty$ if and only if both $P_{i_1}^1(n_2 - i_2) \to 1$ and $P_{n_0}^0 (n_1 - i_1) \to 1$ as $q\to \infty$.  We have three cases to consider, and in each case, we shall find the desired $i_1^\ast$ and $i_2^\ast$.

Case 1: $n_0\leq n_1$ and $n_1\leq n_2$.  By Lemma \ref{lem:condqtoinfty} $P_{n_0}^0 (n_1 - i_1) \to 1$ as $q\to\infty$ if and only if $i_1 = n_1 - n_0$.  So in this case, $i_1^\ast = n_1 - n_0$.  Further, again by Lemma \ref{lem:condqtoinfty}, $P_{i_1^\ast}(n_2 - i_2) = P_{n_1 - n_0}(n_2 - i_2)  \to 1$ as $q\to\infty$ if and only if $i_2 = n_2 - n_1$.  So $i_2^\ast = n_2 - n_1$.

Case 2: $n_0 \leq n_1$ and $n_2 <n_1$. By Lemma \ref{lem:condqtoinfty} $P_{n_0}^0 (n_1 - i_1) \to 1$ as $q\to\infty$ if and only if $i_1 = n_1 - n_0$. So in this case, $i_1^\ast = n_1 - n_0$. We now have two different subcases to consider.
	\begin{itemize}
	\item Subcase 1: If $n_1 - n_0\leq n_2$, then by Lemma \ref{lem:condqtoinfty}, $P_{i_1^\ast}(n_2 - i_2) = P_{n_1-n_0}(n_2-i_2) \to 1$ as $q\to\infty$ if and only if $i_2 = n_2 - n_1$.  So in this subcase, $i_2^\ast = n_2 - n_1$.
	\item Subcase 2: On the other hand, if $n_1 - n_0 >n_2$, then by Lemma \ref{lem:condqtoinfty2}, $P_{i_1^\ast}(n_2 - i_2) = P_{n_1-n_0}(n_2-i_2) \to 1$ as $q\to\infty$ if and only if $i_2 = 0$.  So in this subcase, $i_2^\ast = 0$.
	\end{itemize}
	
Case 3: $n_0 > n_1$. By Lemma \ref{lem:condqtoinfty2}, we know that $P_{n_0}^0 (n_1 - i_1) \to 1$ as $q\to\infty$ if and only if $i_1 = 0$. So in this case, $i_1^\ast = 0$.  Further, no matter the relationship between $n_1$ and $n_2$, we have that Lemma \ref{lem:condqtoinfty2} gives us that that $P_{i_1^\ast}^1(n_2 - i_2) = P_{0}^1(n_2 - i_2)\to 1$ as $q\to\infty$ if and only if $i_2 = 0$.  So in this case, $i_2^\ast = 0$.

Inductive step: Assume there exists exactly one $(i_1^\ast,\ldots , i_{j-1}^\ast)$ in $I^{(j-1)}$ such that 
\[
P_{i_{j-2}}^{j-2}(n_{j-1} - i_{j-1}^\ast)P_{i_{j-3}}^{j-3}(n_{j-2} - i_{j-2}^\ast) \cdots P_{i_1^\ast}^1 (n_2 - i_2^\ast) P_{n_0}^0 (n_1 - i_1^\ast) \to 1
\]
as $q\to\infty$. If, on the one hand, $i_{j-1}^\ast \leq n_j$, set $i_j^\ast = n_j-i_{j-1}^\ast$.  Then by Lemma \ref{lem:condqtoinfty}, $P_{i_{j-1}^\ast}^{j-1}(n_j - i_j^\ast) = P_{i_{j-1}^\ast}^{j-1}(i_{j-1}^\ast) \to 1$ as $q\to\infty$. If, on the other hand, $i_{j-1}^\ast > n_j$, set $i_j^\ast = 0$. Then by Lemma \ref{lem:condqtoinfty2}, $P_{i_{j-1}^\ast}^{j-1}(n_j - i_j^\ast) = P_{i_{j-1}^\ast}^{j-1}(n_j ) \to 1$ as $q \to\infty$.  

So for $\mathbf{i} = (i_1^\ast,\ldots, i_{j-1}^\ast, i_j^\ast)$ in $I^{(j)}$, we have
\[
P_{i_{j-1}^\ast}^{j-1}(n_j-i_j^\ast)P_{i_{j-2}^\ast}^{j-2}(n_{j-1}-i_{j-1}^\ast)\cdots P_{i_1^\ast}^1(n_2-i_2^\ast)P_{n_0}^0(n_1-i_1^\ast) \to 1
\]
as $q\to\infty$, as desired.
\end{proof}

