In this section, we analyze Theorem~\ref{thm:bettinum} under the 
limit $q \ra \infty$.

\begin{proposition}
  \label{prop:oneseq}
Let $I_m:= \{0,1,\ldots, n_m\}$ and let $I^{(j)}:= I_1\times\cdots \times I_j$.
Then for every natural number $j$, there exists exactly one $\mathbf{i}^\ast =
(i_1^\ast,\ldots, i_j^\ast)$ in $I^{(j)}$ such that 
\[
  P_{i_{j-1}^\ast}^{j-1}(n_j-i_j^\ast)\cdots
  P_{i_1^\ast}^1(n_2-i_2^\ast)P_{n_0}^0(n_1-i_1^\ast) \to 1 \, ,
\]
as $q\to\infty$. In particular, if we set $i_0^\ast = n_0$, for each $\ell$ in
$\{1,2,\ldots, j\}$, we have $i_\ell^\ast = (n_\ell - i_{\ell - 1}^\ast)_+$.
\end{proposition}

\begin{proof}
The proof is by induction on $j$.

Base step ($j=1$). By Lemma~\ref{lem:Pqtoinfty}, we have $P_{n_0}^0 (n_1 -
i_1^\ast) \to 1$ as $q\to\infty$ if and only if $n_1 - i_1^\ast = \min(n_0,
n_1)$.  That is, $i_1^\ast = (n_1 - n_o)_+ = (n_1 - i_0^\ast)_+$.

Inductive step. Assume there exists exactly one $(i_1^\ast,\ldots ,
i_{j-1}^\ast)$ in $I^{(j-1)}$, with $i_\ell = (n_\ell - i_{\ell-1}^\ast)_+$ for
$\ell$ in $\{1,2,\ldots, j-1\}$, such that 
\[
P_{i_{j-2}^\ast}^{j-2}(n_{j-1} - i_{j-1}^\ast) %P_{i_{j-3}^\ast}^{j-3}(n_{j-2} - i_{j-2}^\ast) 
\cdots P_{i_1^\ast}^1 (n_2 - i_2^\ast) P_{n_0}^0 (n_1 - i_1^\ast) \to 1 \, ,
\]
as $q\to\infty$.  By Lemma~\ref{lem:Pqtoinfty}, $P_{i_{j-1}^\ast}^{j-1}(n_j -
i_j^\ast) \to 1$ as $q \to \infty$ if and only if $n_j - i_j^\ast =
\min(i_{j-1}^\ast, n_j)$. That is, $i_j^\ast = (n_j - i_{j-1}^\ast)_+$.  
For $\mathbf{i} = (i_1^\ast,\ldots, i_{j-1}^\ast, i_j^\ast)$ in $I^{(j)}$, we
have
\[
P_{i_{j-1}^\ast}^{j-1}(n_j-i_j^\ast)P_{i_{j-2}^\ast}^{j-2}(n_{j-1}-i_{j-1}^\ast)\cdots P_{i_1^\ast}^1(n_2-i_2^\ast)P_{n_0}^0(n_1-i_1^\ast) \to 1
\]
as $q\to\infty$, as desired.
\end{proof}

\mc{As a corollary, we could put what happens to the rank as $q \ra \infty$. This 
is what I think is true. Let me know what you think.}

Proposition~\ref{prop:oneseq} has a number of immediate consequences.

\begin{corollary}
  Let $(\Fqn{n_m},A_m)$ be a uniform random chain complex. Then
  \[
    \bP[\rank(A_m) = n_m-(n_{m-1} -(n_{m-2} -( \cdots -(n_1-n_0)_+ \cdots )_+ )_+ )_+]
    \ra 1 \,
  \]
  as $q \ra \infty$.
\end{corollary}

\begin{proof}
  This follows immediately from Proposition~\ref{prop:oneseq} and Theorem~\ref{thm:qtoinfty}.
\end{proof}

\begin{proof}[Proof of Theorem~\ref{thm:qtoinfty}]
  By Theorem~\ref{thm:bettinum}, it is sufficient to show
  \[
    P^m_{i_m^\ast}(i_m^\ast-b) P_{i^\ast_{m-1}}^{m-1}\left(n_{m} -i_m^\ast\right) \cdots
    P_{i_1^\ast}^1\left(n_2 - i_2^\ast\right) P_{n_0}^0 \left(n_1 - i_1^\ast\right) \to 1 
  \]
  as $q \ra \infty$ for a single sequence $\mathbf{i}^\ast=(i_0^\ast, \ldots,
  i_m^\ast)$ and a single
  value of $b$. After choosing $\mathbf{i}^\ast$ as in Proposition~\ref{prop:oneseq},
  the value of $b$ is easily determined from Lemma~\ref{lem:Pqtoinfty} to be
	\begin{align*}
	b 	&= i_m^\ast - \min(i_m^\ast, n_{m+1})\\
		&= (-n_{m+1} + i_m^\ast)_+\\
		&= (-n_{m+1} + (n_m -i_{m-1}^\ast)_+)_+\\
		&= (-n_{m+1} + (n_m - (n_{m-1} - (\cdots (n_1 - n_0)_+ \cdots)_+)_+)_+ )_+\\
        &= B_m \, . \qedhere
	\end{align*}
\end{proof}

\mz{Do we want a corollary here for the case where $n_m$ is monotone increasing?}
\mc{I added the following corollary and a short proof. What do you think?}

\begin{corollary}
  \label{cor:inc}
  If $\{n_m\}$ is a monotone increasing sequence, then 
  \[
    \lim_{q \ra \infty} \bP[\beta_m = 0] = 1 \, .
  \]
\end{corollary}

\begin{proof}
  By direct inspection, we have
  \[
    (n_m - (n_{m-1} - ( \cdots (n_1 - n_0)_+ \cdots )_+)_+)_+ \leq n_m \, ,
  \]
  and hence $B_m = 0$.
\end{proof}

\begin{corollary}
  The $t$-th moments of the random variable $\beta_m$ satisfy
  \[
    \lim_{q \ra \infty} \mathbb{E}\left[\beta_m^t \right] = B_m^t \, .
  \]
\end{corollary}
% \begin{proof}
	% \begin{align*}
	% \lim_{q \ra \infty} \mathbb{E}\left[\beta_m^t \right]
	% &= \lim_{q \ra \infty} \sum_{b=0}^{n_m} b^t \bP\left[\beta_m = b \right]\\
	% &= \sum_{b=0}^{n_m} b^t \lim_{q \ra \infty} \bP\left[\beta_m = b \right]\\
	% &= B_m^t
	% \end{align*}
% \end{proof}


