In this section, we analyze Theorem~\ref{thm:bettinum} under the 
limit $q \ra \infty$.

\begin{proposition}
  \label{prop:oneseq}
Let $I_m:= \{0,1,\ldots, n_m\}$.  Let $I^{(j)}:= I_1\times\cdots \times I_j$.
Then for every $j$ in $\mathbb{N}$, there exists exactly one $\mathbf{i}^\ast =
(i_1^\ast,\ldots, i_j^\ast)$ in $I^{(j)}$ such that 
\[
  P_{i_{j-1}^\ast}^{j-1}(n_j-i_j^\ast)\cdots
  P_{i_1^\ast}^1(n_2-i_2^\ast)P_{n_0}^0(n_1-i_1^\ast) \to 1 \, ,
\]
as $q\to\infty$.
\end{proposition}

\begin{proof}
The proof is by induction on $j$.

\mz{The case $j=1$ is easy. Show this.}

Base step ($j=2$): Since each $P^m_k(r)$ is a probability, the product $P_{i_1}^1(n_2 -
i_2)P_{n_0}^0 (n_1 - i_1) \to 1$ as $q\to\infty$ if and only if both
$P_{i_1}^1(n_2 - i_2) \to 1$ and $P_{n_0}^0 (n_1 - i_1) \to 1$ as $q\to
\infty$.  We have three cases to consider, and in each case, we shall find the
desired $i_1^\ast$ and $i_2^\ast$.

Case 1: $n_0\leq n_1$ and $n_1\leq n_2$.  By Lemma~\ref{lem:Pqtoinfty},
$P_{n_0}^0 (n_1 - i_1) \to 1$ as $q\to\infty$ if and only if $i_1 = n_1 - n_0$.
Again by Lemma~\ref{lem:Pqtoinfty}, $P_{i_1^\ast}(n_2 - i_2) = P_{n_1 - n_0}(n_2 - i_2)
\to 1$ as $q\to\infty$ if and only if $i_2 = n_2 - n_1 + n_0$.  So $i_1^\ast = n_1 - n_0$
and $i_2^\ast = n_2 - n_1 + n_0$.

Case 2: $n_0 \leq n_1$ and $n_2 <n_1$. By Lemma~\ref{lem:Pqtoinfty},
$P_{n_0}^0 (n_1 - i_1) \to 1$ as $q\to\infty$ if and only if $i_1 = n_1 - n_0$.
So in this case, $i_1^\ast = n_1 - n_0$. We now have two different subcases to
consider.  
\begin{itemize} 
  \item Subcase 1: If $n_1 - n_0\leq n_2$, then by
    Lemma \ref{lem:Pqtoinfty}, $P_{i_1^\ast}^1(n_2 - i_2) =
    P_{n_1-n_0}^1(n_2-i_2) \to 1$ as $q\to\infty$ if and only if $i_2 = n_2 -
    n_1 + n_0$.  So in this subcase, $i_2^\ast = n_2 - n_1 + n_0$.  
  \item Subcase 2: On the other hand, if $n_1 - n_0 >n_2$, then by 
  Lemma~\ref{lem:Pqtoinfty}, $P_{i_1^\ast}(n_2 - i_2) = P_{n_1-n_0}(n_2-i_2)
    \to 1$ as $q\to\infty$ if and only if $i_2 = 0$.  So in this subcase,
    $i_2^\ast = 0$.  
\end{itemize}

Case 3: $n_0 > n_1$. By Lemma~\ref{lem:Pqtoinfty}, we know that $P_{n_0}^0
(n_1 - i_1) \to 1$ as $q\to\infty$ if and only if $i_1 = 0$. 
Further, regardless of the values of $n_1$ and $n_2$,
we have that Lemma~\ref{lem:Pqtoinfty} gives 
$P_{i_1^\ast}^1(n_2 - i_2) = P_{0}^1(n_2 - i_2)\to 1$ as $q\to\infty$ if and
only if $i_2 = 0$.  So in this case, $i_1^\ast = 0$ and $i_2^\ast = 0$.

Inductive step: Assume there exists exactly one $(i_1^\ast,\ldots , i_{j-1}^\ast)$ in $I^{(j-1)}$ such that 
\[
P_{i_{j-2}}^{j-2}(n_{j-1} - i_{j-1}^\ast)P_{i_{j-3}}^{j-3}(n_{j-2} - i_{j-2}^\ast) \cdots P_{i_1^\ast}^1 (n_2 - i_2^\ast) P_{n_0}^0 (n_1 - i_1^\ast) \to 1
\]
as $q\to\infty$. If, on the one hand, $i_{j-1}^\ast \leq n_j$, set $i_j^\ast =
n_j-i_{j-1}^\ast$.  Then by Lemma~\ref{lem:Pqtoinfty},
$P_{i_{j-1}^\ast}^{j-1}(n_j - i_j^\ast) = P_{i_{j-1}^\ast}^{j-1}(i_{j-1}^\ast)
\to 1$ as $q\to\infty$. If, on the other hand, $i_{j-1}^\ast > n_j$, set
$i_j^\ast = 0$. Then by Lemma~\ref{lem:Pqtoinfty},
$P_{i_{j-1}^\ast}^{j-1}(n_j - i_j^\ast) = P_{i_{j-1}^\ast}^{j-1}(n_j ) \to 1$
as $q \to\infty$.  

So for $\mathbf{i} = (i_1^\ast,\ldots, i_{j-1}^\ast, i_j^\ast)$ in $I^{(j)}$, we have
\[
P_{i_{j-1}^\ast}^{j-1}(n_j-i_j^\ast)P_{i_{j-2}^\ast}^{j-2}(n_{j-1}-i_{j-1}^\ast)\cdots P_{i_1^\ast}^1(n_2-i_2^\ast)P_{n_0}^0(n_1-i_1^\ast) \to 1
\]
as $q\to\infty$, as desired.
\end{proof}

\begin{proof}[Proof of Theorem~\ref{thm:qtoinfty}]
  (sketch of proof) By Proposition~\ref{prop:oneseq}, there is a single $\mathbf{i}^*$
  such that
  \[
    P_{i_{j-1}^\ast}^{j-1}(n_j-i_j^\ast)\cdots
    P_{i_1^\ast}^1(n_2-i_2^\ast)P_{n_0}^0(n_1-i_1^\ast) \to 1 \, .
  \]
  By Lemma~\ref{lem:Pqtoinfty} there is one value of 
\end{proof}

