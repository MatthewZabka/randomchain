\mc{I think this section will be completely removed now with no conjectures.
I still point out what has and hasn't been done.}

This section consists of a bunch of miscellaneous facts and conjectures that
either we should include or not, but either way, I think we should understand
them. In particular, I think these will be useful things to know to get a good
grasp on what's really going on. 

\begin{conjecture}
 Consider the following.
  \begin{enumerate}
    \item Fix $r$. As a function of $k$, $P^m_k(r)$ is decreasing on its support.
    \item Fix $k$. As a function of $r$, $P^m_k(r)$ is increasing on its support.
    \item $P^m_k(r)$ has a maximum when $k = r$, as either variable changes.
    \item $P^m_{r+1}(r) > P^m_r(r-1)$.
  \end{enumerate}
\end{conjecture}

\mc{Answer: We haven't answered this although its probably pretty easy to prove.}


\begin{question} 
  What is the max of $\bP[\beta_1=b]$, as a function of $b$? For
  what $b$ is the max attained? The same questions for $\bP[\beta_2=b]$.
\end{question}
  \mc{Answer: I think our stuff shows that $\bP[\beta_1=b]$ attains its maximum
    at $b = N_m$. Of course, this is in the limit as $q \ra \infty$, but maybe
    there's an argument that says for fixed $q$, $\bP[\beta_1=b]$ is a 
    decreasing function of $b$, and therefore it attains its maximum at
    the minimal value of $b$. And there should be nothing special to $\beta_1$
  here, and should hold for $\beta_k$ in general.}

\begin{question}
  What if $n_0 = w$ and $n>0 = 2w$? Intuitively, we would expect these numbers
  to not depend on the degree of the chain complex. Something similar should
  be true for any sequence of $(n_i)$ which converge.
\end{question}
\mc{Answer: If the sequence converges, it must be eventually constant and then $N_m = 0$.}

\begin{question}
  What does this formula say if there exists $N >0$ so that $n_i=0$ for 
  all $i > N$? Specifically, $\bP[\beta_i=0] = 1$ and $\bP[\beta_i > 0]= 0$ 
  for $i>N$.
\end{question}
\mc{Answer: same.}

Using the same logic as the previous theorem, I'm {\em pretty sure} we have
the following.

%\begin{conjecture}
%  \[
%    \bP[\rank A_m = r_m] = \sum_{i_{m-1}=0}^{n_{m-1}} P^{m-1}_{i_{m-1}}(r_m)
%    \sum_{i_{m-2} = 0}^{n_{m-2}} P^{m-2}_{i_{m-2}}(n_{m-1}- i_{m-1})
%    \cdots \sum_{i_1 = 0}^{n_2}P^1_{i_1}(n_2-i_2) P^0_{n_0}(n_1-i_1)
%  \]
%\end{conjecture}


\begin{question} What distribution does the random variable
  $\rank(A_m)$ follow? What are its moments?
\end{question}
\mc{Answer: we could answer this, we have answered the second question for $\beta_m$.}

\begin{conjecture}
  Attempt to answer all of the following in cases:
  (i) $n_i$ is eventually constant, (ii) $n_i \sim o(i)$.
  \begin{enumerate}
    \item $\bP[\beta_m = 0] \ra ?$ as $m \ra \infty$.
    \item $\bP[\beta_m = j] \ra ?$ as $m \ra \infty$, $j>0$.
    \item $\bE[\beta_m] = ?$ as $m \ra \infty$.
  \end{enumerate}
\end{conjecture}
\mc{Answer: done.}

